\documentclass[english]{article}
\newcommand{\G}{\overline{C_{2k-1}}}
\usepackage[latin9]{inputenc}
\usepackage{amsmath}
\usepackage{amssymb}
\usepackage{lmodern}
\usepackage{mathtools}
\usepackage{enumitem}
%\usepackage{natbib}
%\bibliographystyle{plainnat}
%\setcitestyle{authoryear,open={(},close={)}}
\let\avec=\vec
\renewcommand\vec{\mathbf}
\renewcommand{\d}[1]{\ensuremath{\operatorname{d}\!{#1}}}
\newcommand{\pydx}[2]{\frac{\partial #1}{\partial #2}}
\newcommand{\dydx}[2]{\frac{\d #1}{\d #2}}
\newcommand{\ddx}[1]{\frac{\d{}}{\d{#1}}}
\newcommand{\hk}{\hat{K}}
\newcommand{\hl}{\hat{\lambda}}
\newcommand{\ol}{\overline{\lambda}}
\newcommand{\om}{\overline{\mu}}
\newcommand{\all}{\text{all }}
\newcommand{\valph}{\vec{\alpha}}
\newcommand{\vbet}{\vec{\beta}}
\newcommand{\vT}{\vec{T}}
\newcommand{\vN}{\vec{N}}
\newcommand{\vB}{\vec{B}}
\newcommand{\vX}{\vec{X}}
\newcommand{\vx}{\vec {x}}
\newcommand{\vn}{\vec{n}}
\newcommand{\vxs}{\vec {x}^*}
\newcommand{\vV}{\vec{V}}
\newcommand{\vTa}{\vec{T}_\alpha}
\newcommand{\vNa}{\vec{N}_\alpha}
\newcommand{\vBa}{\vec{B}_\alpha}
\newcommand{\vTb}{\vec{T}_\beta}
\newcommand{\vNb}{\vec{N}_\beta}
\newcommand{\vBb}{\vec{B}_\beta}
\newcommand{\bvT}{\bar{\vT}}
\newcommand{\ka}{\kappa_\alpha}
\newcommand{\ta}{\tau_\alpha}
\newcommand{\kb}{\kappa_\beta}
\newcommand{\tb}{\tau_\beta}
\newcommand{\hth}{\hat{\theta}}
\newcommand{\evat}[3]{\left. #1\right|_{#2}^{#3}}
\newcommand{\prompt}[1]{\begin{prompt*}
		#1
\end{prompt*}}
\newcommand{\vy}{\vec{y}}
\DeclareMathOperator{\sech}{sech}
\DeclarePairedDelimiter\abs{\lvert}{\rvert}%
\DeclarePairedDelimiter\norm{\lVert}{\rVert}%
\newcommand{\dis}[1]{\begin{align}
	#1
	\end{align}}
\newcommand{\LL}{\mathcal{L}}
\newcommand{\RR}{\mathbb{R}}
\newcommand{\NN}{\mathbb{N}}
\newcommand{\ZZ}{\mathbb{Z}}
\newcommand{\QQ}{\mathbb{Q}}
\newcommand{\Ss}{\mathcal{S}}
\newcommand{\BB}{\mathcal{B}}
\usepackage{graphicx}
% Swap the definition of \abs* and \norm*, so that \abs
% and \norm resizes the size of the brackets, and the 
% starred version does not.
%\makeatletter
%\let\oldabs\abs
%\def\abs{\@ifstar{\oldabs}{\oldabs*}}
%
%\let\oldnorm\norm
%\def\norm{\@ifstar{\oldnorm}{\oldnorm*}}
%\makeatother
\newenvironment{subproof}[1][\proofname]{%
	\renewcommand{\qedsymbol}{$\blacksquare$}%
	\begin{proof}[#1]%
	}{%
	\end{proof}%
}

\usepackage{centernot}
\usepackage{dirtytalk}
\usepackage{calc}
\newcommand{\prob}[1]{\setcounter{section}{#1-1}\section{}}


\newcommand{\prt}[1]{\setcounter{subsection}{#1-1}\subsection{}}
\newcommand{\pprt}[1]{{\textit{{#1}.)}}\newline}
\renewcommand\thesubsection{\alph{subsection}}
\usepackage[sl,bf,compact]{titlesec}
\titlelabel{\thetitle.)\quad}
\DeclarePairedDelimiter\floor{\lfloor}{\rfloor}
\makeatletter

\newcommand*\pFqskip{8mu}
\catcode`,\active
\newcommand*\pFq{\begingroup
	\catcode`\,\active
	\def ,{\mskip\pFqskip\relax}%
	\dopFq
}
\catcode`\,12
\def\dopFq#1#2#3#4#5{%
	{}_{#1}F_{#2}\biggl(\genfrac..{0pt}{}{#3}{#4}|#5\biggr
	)%
	\endgroup
}
\def\res{\mathop{Res}\limits}
% Symbols \wedge and \vee from mathabx
% \DeclareFontFamily{U}{matha}{\hyphenchar\font45}
% \DeclareFontShape{U}{matha}{m}{n}{
%       <5> <6> <7> <8> <9> <10> gen * matha
%       <10.95> matha10 <12> <14.4> <17.28> <20.74> <24.88> matha12
%       }{}
% \DeclareSymbolFont{matha}{U}{matha}{m}{n}
% \DeclareMathSymbol{\wedge}         {2}{matha}{"5E}
% \DeclareMathSymbol{\vee}           {2}{matha}{"5F}
% \makeatother

%\titlelabel{(\thesubsection)}
%\titlelabel{(\thesubsection)\quad}
\usepackage{listings}
\lstloadlanguages{[5.2]Mathematica}
\usepackage{babel}
\newcommand{\ffac}[2]{{(#1)}^{\underline{#2}}}
\usepackage{color}
\usepackage{amsthm}
\newtheorem{theorem}{Theorem}[section]
%\newtheorem*{theorem*}{Theorem}[section]
\newtheorem{conj}[theorem]{Conjecture}
\newtheorem{corollary}[theorem]{Corollary}
\newtheorem{example}[theorem]{Example}
\newtheorem{lemma}[theorem]{Lemma}
\newtheorem*{lemma*}{Lemma}
\newtheorem{problem}[theorem]{Problem}
\newtheorem{proposition}[theorem]{Proposition}
\newtheorem*{proposition*}{Proposition}
\newtheorem*{corollary*}{Corollary}
\newtheorem{fact}[theorem]{Fact}
\newtheorem*{prompt*}{Prompt}
\newtheorem*{claim*}{Claim}
\newcommand{\claim}[1]{\begin{claim*} #1\end{claim*}}
%organizing theorem environments by style--by the way, should we really have definitions (and notations I guess) in proposition style? it makes SO much of our text italicized, which is weird.
\theoremstyle{remark}
\newtheorem{remark}{Remark}[section]

\theoremstyle{definition}
\newtheorem{definition}[theorem]{Definition}
\newtheorem{notation}[theorem]{Notation}
\newtheorem*{notation*}{Notation}
%FINAL
\newcommand{\due}{2 October 2017} 
\RequirePackage{geometry}
\geometry{margin=.7in}
\usepackage{todonotes}
\title{MATH 8301 Homework IV}
\author{David DeMark}
\date{\due}
\usepackage{fancyhdr}
\pagestyle{fancy}
\fancyhf{}
\rhead{David DeMark}
\chead{\due}
\lhead{MATH 8301}
\cfoot{\thepage}
% %%
%%
%%
%DRAFT

%\usepackage[left=1cm,right=4.5cm,top=2cm,bottom=1.5cm,marginparwidth=4cm]{geometry}
%\usepackage{todonotes}
% \title{MATH 8669 Homework 4-DRAFT}
% \usepackage{fancyhdr}
% \pagestyle{fancy}
% \fancyhf{}
% \rhead{David DeMark}
% \lhead{MATH 8669-Homework 4-DRAFT}
% \cfoot{\thepage}

%PROBLEM SPEFICIC

\newcommand{\lint}{\underline{\int}}
\newcommand{\uint}{\overline{\int}}
\newcommand{\hfi}{\hat{f}^{-1}}
\newcommand{\tfi}{\tilde{f}^{-1}}
\newcommand{\tsi}{\tilde{f}^{-1}}
\newcommand{\PP}{\mathcal{P}}
\newcommand{\nin}{\centernot\in}
\newcommand{\seq}[1]{({#1}_n)_{n\geq 1}}
\newcommand{\Tt}{\mathcal{T}}
\newcommand{\card}{\mathrm{card}}
\newcommand{\setc}[2]{\{ #1\::\:#2 \}}
\newcommand{\Fcal}{\mathcal{F}}
\newcommand{\cbal}{\overline{B}}
\newcommand{\Ccal}{\mathcal{C}}
\newcommand{\cl}{\overline}
\newcommand{\id}{\mathrm{id}}
\newcommand{\intr}{\mathrm{int}}
\begin{document}
	\maketitle
\prob{1}
\prt{1} \prompt{For $\abs{(V,\Fcal)}$ a compact connected surface, find a relation between $e,f$}
\begin{proof}[Response]
As $\abs{(V,\Fcal)}$ is compact, we may assume each edge of each 2-simplex is identified with an edge of a distinct 2-simplex. We let $S=\{(\Delta, \epsilon)\in \Fcal^2:\abs{\Delta}=3,\;\abs{\epsilon}=2,\;\epsilon\subset \Delta\}$. As each 2-simplex $\Delta_0$ has boundary composed of three edges $\epsilon_1,\epsilon_2,\epsilon_3$, we have that $\abs{S}=3f$. We also have that each edge is boundary to two 2-simplices and thus $\abs{S}=2e$. This gives us the relation $3f=2e$
\end{proof}
\prt{2}\prompt{Find formulas for $f,e$ in terms of $\chi,v$.}
	\begin{proof}[Response]
We have that $\chi=v-e+f$. Substituting $f=\frac{2}{3}e$, we have $\chi=v-\frac{e}{3}\implies e=3(v-\chi)$. Now, substituting $e=\frac{3}{2}f$, we have $\frac{3}{2}f=3(v-\chi)\implies f=2(v-\chi)$.
	\end{proof}

\prt{3}\begin{proposition*}
	Show for any triangulation of a compact surface $S=\abs{(V,\Fcal)}$ with $\chi(S)=0$ that $v(S)=v\geq 7$.
\end{proposition*}
\begin{proof}
	We have that any triangulation of a surface must have at least one vertex as else the simplicial complex $(V,\Fcal)$ is necessarily empty. Furthermore, as the 1-skeleton of $(V,\Fcal)$ must be a graph in the classical sense (that is, with no loops or double edges), we have that $3v=e\leq{v\choose 2}$. Hence, we must have that $v$ satisfies the solution $3v\leq \frac{v^2-v}{2}$, or simplifying, $v^2-7v\geq 0$, that is $v\leq 0$ or $v\geq 7$. However, as we have already noted that $v\geq 1$, we must have that $v\geq 7$. 
\end{proof}
\prob{2}
\prt{1}
\begin{proposition*}
	Let $P$ be star-shaped with respect to $p$. Then, $P$ is contractible.
\end{proposition*}
\begin{proof}
	We let $H:P\times [0,1]\to P$ be given by $H(q,t)=(1-t)q+tp$. It is well-known and obvious that $H$ is continuous should it be well-defined, and one may view the definition of star-shaped precisely as $H(q_0,t_0)\in P$ for any $q_0\in P$, $t_0\in [0,1]$. Thus, $H$ is well defined and as $H(q,1)=p$ for any $q\in P$, we have that it is a deformation retract and hence a detraction.
\end{proof}
\prt{2}
\begin{proposition*}
	If $P$ is a polygon star-shaped w/r/t $p\in \intr P$, then $f:P\setminus \{p\}\to S^1$ given by $f(q)=\frac{q-p}{\abs{q-p}}$ is a homotopy equivalence.
\end{proposition*}
\begin{proof}
As $p\in \intr P$, there exists some $\delta>0$ such that $\cl{B}_\delta(p)\subset P$. We let $S=\partial \cl{B}_{\delta}(p)$. We then let $g:S^1\to P$ be given by $g(x)=\delta x+p$ and note $\mathrm{im}\,g=S$. Then, $fg(x)=\mathrm{id}_{S^1}$, as $fg(x)=f(\delta x+p)=\frac{(\delta x+p)-p}{\abs{(\delta x+ p)-p}}=\frac{x}{\abs{x}}=x$ as $\abs{x}=1$. On the other hand, we have that $gf(q)=g\left(\frac{q-p}{\abs{q-p}}\right)=\delta\frac{(q-p)}{\abs{q-p}}+p$--in particular, as $gf(q)$ is of the form $p+s(q-p)$, we have that for any $q\not \in B_\delta(p)$, $gf(q)$ is on the line $\cl{pq}$. On the other hand, for $q\in B_\delta(q)$, $gf(q)$ is still colinear to $p,q$, so we have that $q\in \cl{p(gf(q))}$. Hence, we may take $H:P\times[0,1]\to P$ by $(q,t)\mapsto (1-t)q+tgf(q)$ to be our homotopy given our equivalence
\end{proof}
\prt{3}
\begin{proposition*}
	$T^2\setminus\{p\}\approxeq S^1\# S^1$
\end{proposition*}\begin{proof}
We begin with a lemma.
\begin{lemma*}
	We let $X$ be a topological space, $A\subset X$ and $\sim$ an equivalence relation on $X$ such that for all $x\in X\setminus A$, $[x]$ is a singleton. Then, if $f:X\to A$ is a homotopy equivalence with homotopy inverse the inclusion $\iota$ such that $\iota f\approxeq_A \id_X$, $X/\sim$ is in homotopy equivalence with $A/\sim$.
	\end{lemma*}
\begin{subproof} We let the natural maps $X\to X/\sim$ and $A\to A/\sim$ be denoted $q_X$ and $q_A$ respectively.
	We let $H:X\times [0,1]\to X$ give a homotopy relative to $A$ from $\id_X$ to $\iota f$. 
	We define $\tilde f:X/\sim\to A/\sim$ by $\tilde f([x])=q_A\circ f(x)$, with $\tilde \iota$ defined similarly, and $\tilde H$ given by $\tilde H([x],t)=q_X\circ H(x,t)$. As our hypothesis forces $f$, $\iota$ and $H(-,t)$ to restrict to the identity (composed or precomposed with the inclusion $\iota$ when appropriate) on $A$ and $[x]$ is a singleton for $x\in X\setminus A$ we have that each of $f,\iota$ and $H$ respect $\sim$ and hence $\tilde f, \tilde \iota $ and $\tilde H$ well defined. We note that $\tilde{f}\tilde{\iota}=\id_{A/\sim}$ as $f\iota=\id_A$, and hence need only show that $\tilde H$ gives a homotopy between $\tilde \iota \tilde f$ and $\id_{X/\sim}$. As $\tilde H(-,1)=\id_{X/\sim}$ by construction, we need only show that $\tilde H(-,0)=\tilde \iota \tilde f$. We have that $\tilde H(x,0)=(q_X\circ \iota \circ f)(x)$, and note that $q_X\circ \iota=\tilde\iota \circ q_A$ as $q_X$ restricts to $ q_A$ on $A$. Further, we defined $\tilde f([x])=q_A\circ f(x)$. Hence, $\tilde H(x,0)=\tilde \iota \tilde f$ and hence gives the desired homotopy. 
\end{subproof}
By the lemma, we now need only show that $I^2\setminus \{p\}$ where $I$ is the unit interval and $p\in \intr I^2$, is homotopy equivalent to $\partial I^2$ relative to $\partial I^2$. As $I^2$ is convex, we have that for any $q\in I^2$, the line segment $\cl{pq}$ is colinear to precisely one point $d_q\in \partial I^2$. We let $f(q)=d_q$, that is we let $f$ be the projection-to-boundary map. By convexity, it is clear that $f$ is continuous, and as $I^2$ is star-shaped with respect to any point of its interior, we have that the linear homotopy $H$ from $\id_{I^2\setminus\{p\}}$ to $f$ by $(q,t)\mapsto (1-t)q+td_q$ is well-defined and continuous. Further, as $q=d_q$ for any $q\in \partial I^2$, we have that $H$ is a homotopy relative to $\partial I^2$. Furthermore, as $f\iota=\id_{\partial I^2}$, $f$ is a homotopy equivalence relative to $\partial I^2$. Thus, letting $\sim$ be the standard toral side-identification, we have that $T^2\setminus \{[p]\}=I^2\setminus \{p\}/\sim\approxeq \partial I^2/\sim=S^1\#S^1$ \end{proof} 
\end{document}
