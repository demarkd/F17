\documentclass[english]{article}
\newcommand{\G}{\overline{C_{2k-1}}}
\usepackage[latin9]{inputenc}
\usepackage{amsmath}
\usepackage{amssymb}
\usepackage{lmodern}
\usepackage{mathtools}
\usepackage{enumitem}
\usepackage{relsize}

%\usepackage{natbib}
%\bibliographystyle{plainnat}
%\setcitestyle{authoryear,open={(},close={)}}
\let\avec=\vec
\renewcommand\vec{\mathbf}
\renewcommand{\d}[1]{\ensuremath{\operatorname{d}\!{#1}}}
\newcommand{\pydx}[2]{\frac{\partial #1}{\partial #2}}
\newcommand{\dydx}[2]{\frac{\d #1}{\d #2}}
\newcommand{\ddx}[1]{\frac{\d{}}{\d{#1}}}
\newcommand{\hk}{\hat{K}}
\newcommand{\hl}{\hat{\lambda}}
\newcommand{\ol}{\overline{\lambda}}
\newcommand{\om}{\overline{\mu}}
\newcommand{\all}{\text{all }}
\newcommand{\valph}{\vec{\alpha}}
\newcommand{\vbet}{\vec{\beta}}
\newcommand{\vT}{\vec{T}}
\newcommand{\vN}{\vec{N}}
\newcommand{\vB}{\vec{B}}
\newcommand{\vX}{\vec{X}}
\newcommand{\vx}{\vec {x}}
\newcommand{\vn}{\vec{n}}
\newcommand{\vxs}{\vec {x}^*}
\newcommand{\vV}{\vec{V}}
\newcommand{\vTa}{\vec{T}_\alpha}
\newcommand{\vNa}{\vec{N}_\alpha}
\newcommand{\vBa}{\vec{B}_\alpha}
\newcommand{\vTb}{\vec{T}_\beta}
\newcommand{\vNb}{\vec{N}_\beta}
\newcommand{\vBb}{\vec{B}_\beta}
\newcommand{\bvT}{\bar{\vT}}
\newcommand{\ka}{\kappa_\alpha}
\newcommand{\ta}{\tau_\alpha}
\newcommand{\kb}{\kappa_\beta}
\newcommand{\tb}{\tau_\beta}
\newcommand{\hth}{\hat{\theta}}
\newcommand{\evat}[3]{\left. #1\right|_{#2}^{#3}}
\newcommand{\restr}[2]{\evat{#1}{#2}{}}
\newcommand{\prompt}[1]{\begin{prompt*}
		#1
\end{prompt*}}
\newcommand{\vy}{\vec{y}}
\DeclareMathOperator{\sech}{sech}
\DeclarePairedDelimiter\abs{\lvert}{\rvert}%
\DeclarePairedDelimiter\norm{\lVert}{\rVert}%
\newcommand{\dis}[1]{\begin{align}
	#1
	\end{align}}
\newcommand{\LL}{\mathcal{L}}
\newcommand{\RR}{\mathbb{R}}
\newcommand{\CC}{\mathbb{C}}
\newcommand{\NN}{\mathbb{N}}
\newcommand{\ZZ}{\mathbb{Z}}
\newcommand{\QQ}{\mathbb{Q}}
\newcommand{\Ss}{\mathcal{S}}
\newcommand{\BB}{\mathcal{B}}
\usepackage{graphicx}
% Swap the definition of \abs* and \norm*, so that \abs
% and \norm resizes the size of the brackets, and the 
% starred version does not.
%\makeatletter
%\let\oldabs\abs
%\def\abs{\@ifstar{\oldabs}{\oldabs*}}
%
%\let\oldnorm\norm
%\def\norm{\@ifstar{\oldnorm}{\oldnorm*}}
%\makeatother
\newenvironment{subproof}[1][\proofname]{%
	\renewcommand{\qedsymbol}{$\blacksquare$}%
	\begin{proof}[#1]%
	}{%
	\end{proof}%
}

\usepackage{centernot}
\usepackage{dirtytalk}
\usepackage{calc}
\newcommand{\prob}[1]{\setcounter{section}{#1-1}\section{}}


\newcommand{\prt}[1]{\setcounter{subsection}{#1-1}\subsection{}}
\newcommand{\pprt}[1]{{\textit{{#1}.)}}\newline}
\renewcommand\thesubsection{\alph{subsection}}
\usepackage[sl,bf,compact]{titlesec}
\titlelabel{\thetitle.)\quad}
\DeclarePairedDelimiter\floor{\lfloor}{\rfloor}
\makeatletter

\newcommand*\pFqskip{8mu}
\catcode`,\active
\newcommand*\pFq{\begingroup
	\catcode`\,\active
	\def ,{\mskip\pFqskip\relax}%
	\dopFq
}
\catcode`\,12
\def\dopFq#1#2#3#4#5{%
	{}_{#1}F_{#2}\biggl(\genfrac..{0pt}{}{#3}{#4}|#5\biggr
	)%
	\endgroup
}
\def\res{\mathop{Res}\limits}
% Symbols \wedge and \vee from mathabx
% \DeclareFontFamily{U}{matha}{\hyphenchar\font45}
% \DeclareFontShape{U}{matha}{m}{n}{
%       <5> <6> <7> <8> <9> <10> gen * matha
%       <10.95> matha10 <12> <14.4> <17.28> <20.74> <24.88> matha12
%       }{}
% \DeclareSymbolFont{matha}{U}{matha}{m}{n}
% \DeclareMathSymbol{\wedge}         {2}{matha}{"5E}
% \DeclareMathSymbol{\vee}           {2}{matha}{"5F}
% \makeatother

%\titlelabel{(\thesubsection)}
%\titlelabel{(\thesubsection)\quad}
\usepackage{listings}
\lstloadlanguages{[5.2]Mathematica}
\usepackage{babel}
\newcommand{\ffac}[2]{{(#1)}^{\underline{#2}}}
\usepackage{color}
\usepackage{amsthm}
\newtheorem{theorem}{Theorem}[section]
\newtheorem*{theorem*}{Theorem}
\newtheorem{conj}[theorem]{Conjecture}
\newtheorem{corollary}[theorem]{Corollary}
\newtheorem{example}[theorem]{Example}
\newtheorem{lemma}[theorem]{Lemma}
\newtheorem*{lemma*}{Lemma}
\newtheorem{problem}[theorem]{Problem}
\newtheorem{proposition}[theorem]{Proposition}
\newtheorem*{proposition*}{Proposition}
\newtheorem*{corollary*}{Corollary}
\newtheorem{fact}[theorem]{Fact}
\newtheorem*{prompt*}{Prompt}
\newtheorem*{claim*}{Claim}
\newtheorem{claim}{Claim}
%\newcommand{\claim}[1]{\begin{claim*} #1\end{claim*}}
%organizing theorem environments by style--by the way, should we really have definitions (and notations I guess) in proposition style? it makes SO much of our text italicized, which is weird.
\theoremstyle{remark}
\newtheorem{remark}{Remark}[section]

\theoremstyle{definition}
\newtheorem{definition}[theorem]{Definition}
\newtheorem*{definition*}{Definition}
\newtheorem{notation}[theorem]{Notation}
\newtheorem*{notation*}{Notation}
%FINAL
\newcommand{\due}{19 November 2017} 
\RequirePackage{geometry}
\geometry{margin=.7in}
\usepackage{todonotes}
\title{MATH 8301 Homework IX}
\author{David DeMark}
\date{\due}
\usepackage{fancyhdr}
\pagestyle{fancy}
\fancyhf{}
\rhead{David DeMark}
\chead{\due}
\lhead{MATH 8301}
\cfoot{\thepage}
% %%
%%
%%
%DRAFT

%\usepackage[left=1cm,right=4.5cm,top=2cm,bottom=1.5cm,marginparwidth=4cm]{geometry}
%\usepackage{todonotes}
% \title{MATH 8669 Homework 4-DRAFT}
% \usepackage{fancyhdr}
% \pagestyle{fancy}
% \fancyhf{}
% \rhead{David DeMark}
% \lhead{MATH 8669-Homework 4-DRAFT}
% \cfoot{\thepage}

%PROBLEM SPEFICIC

\newcommand{\lint}{\underline{\int}}
\newcommand{\uint}{\overline{\int}}
\newcommand{\hfi}{\hat{f}^{-1}}
\newcommand{\tfi}{\tilde{f}^{-1}}
\newcommand{\tsi}{\tilde{f}^{-1}}
\newcommand{\PP}{\mathcal{P}}
\newcommand{\nin}{\centernot\in}
\newcommand{\seq}[1]{({#1}_n)_{n\geq 1}}
\newcommand{\Tt}{\mathcal{T}}
\newcommand{\card}{\mathrm{card}}
\newcommand{\setc}[2]{\{ #1\::\:#2 \}}
\newcommand{\Fcal}{\mathcal{F}}
\newcommand{\cbal}{\overline{B}}
\newcommand{\Ccal}{\mathcal{C}}
\newcommand{\Dcal}{\mathcal{D}}
\newcommand{\cl}{\overline}
\newcommand{\id}{\mathrm{id}}
\newcommand{\intr}{\mathrm{int}}
\renewcommand{\hom}{\mathrm{Hom}}
\newcommand{\vect}{\mathrm{Vect}}
\newcommand{\Top}{\mathrm{Top}}
\renewcommand{\top}{\Top}
\newcommand{\hTop}{\mathrm{hTop}}
\newcommand{\set}{\mathrm{Set}}
\newcommand{\frp}{\mathop{\large {\mathlarger{\star}}}}
\newcommand{\ondt}{1_{\cdot}}
\newcommand{\onst}{1_{\star}}
\newcommand{\bdy}{\partial}
\newcommand{\im}{\mathrm{im}}
\newcommand{\re}{\mathrm{re}}
\newcommand{\oX}{\overline{X}}
\begin{document}
\maketitle
\prob{1} We let $X$ be a path-connected topological space with $x_0\in N\subset X$ where $N$ is contractible, $f:X\to X$ a continuous map fixing $x_0$, and $M_f$ its mapping torus with basepoint $m_0$ the image of $(x_0,1/2)$ in $M_f$.  
\prt{1}
\begin{proposition*}
	$M_f$ is path-connected. 
\end{proposition*}
\begin{proof}
	We note that as $X$ and $I$ are path-connected, $X\times I$ is path-connected as the product of paths is itself a path. Thus, $M_f$ is path-connected, as the quotient of a path is also itself a path.
\end{proof}
\prt{2}\begin{proposition*}
	We let $U$ be the image of $X\times (0,1)$. Then, $\pi_{1}(U,m_0)=\pi_1(X,x_0)$. 
\end{proposition*}
\begin{proof}
	We note that each point in $U$ is in a singleton equivalence class in the quotient defining $M_f$. Thus, $U\cong X\times (0,1)$, so $\pi_1(U,m_0)=\pi_1(X,x_0)\times \pi_1((0,1),1/2)=\pi_1(X,x_0)$.
\end{proof}
\prt{3} \begin{proposition*}
	We let $V=X\times ([0,1/3)\cup (2/3,1])\cup N\times I$. Then, $\pi_1(V,m_0)=\pi_1(X,x_0)\frp \ZZ$. 
\end{proposition*}
\begin{proof}
	We let $W$ be a contractible neighborhood around $X_0$ such that $f(W)\subset N$ and $f(W)$ is contractible. We let $A=X\times ([0,1/3)\cup (2/3,1])$, and let $B=(f(W)\times [0,1/3))\cup (W\times (2/3,1] )\cup N\times (1/5,4/5)$. We let $m_0'$ be the image of $(x_0,1)$. Then, $B$ deformation-retracts to $S_1$ by applying the appropriate contraction-retraction to each cross-section, so $\pi_1(B,m_0')\cong\ZZ$, and $A$ deformation retracts to $X\times {0}$, which we show by constructing a map $H:A\times I\to A$ defined by \begin{equation*}
		H((x,t),s):=\begin{cases}
		(x,(1-s)t) & 0<t<1/3\\
		(x,(1-s)t+s)& 2/3<t<1\\
		(x,t)&t=0
		\end{cases}
	\end{equation*} 
	Thus, $\pi_1(A,m_0')\cong\pi_1(X,x_0)$. Finally, as $A\cap B$ deformation-retracts to $S_1\setminus \{e^{i\theta}\;:\;1/3\leq \theta \leq2/3\}$, we have that $\pi_1(A\cap B,m_0')$ is trivial. Thus Siefert-von Kampen yields the statement of the proposition. 
\end{proof}
\prt{4} \begin{proposition*}
	$\pi_1(U\cap V,m_0)\cong G\frp \tilde{G}$ where $G\cong \pi_1(X,m_0)\cong \tilde{G}$.
\end{proposition*}
\begin{proof}
	We let $A'=(X\times (2/3,1))\cup N\times (1/3,4/5) $ and $B'=(X\times (0,1/3))\cup N\times (1/5,2/3) $. Then, by contracting $(1/3,1)$ to $\{4/5\}$ in the second coordinate, we see that $A'$ deformation-retracts to a space homeomorphic to $X$, as does $B'$ by a symmetric argument. Finally, $A'\cup B'$ is simply $N\times (1/3,2/3)$, which is contractible, proving the proposition by an application of SvK. 
\end{proof}
\prt{5}\begin{proposition*}
	$H:=\pi_1(M_f,m_0)=(G\frp \ZZ)/\langle tf_*(\gamma)t^{-1}f_*(\gamma^{-1})\;:\; \gamma \in G\rangle$.
\end{proposition*}\begin{proof}
We have from Siefert-von Kampen that $H=(G\frp \ZZ\frp \hat{G})/M$ where $\hat{G}$ is again a distinct isomorphic copy of $G$ and $M$ is some normal subgroup. We let $\iota_U$ and $\iota_V$ be the inclusions of $U\cap V$ to their indices. We claim that $(\iota_U)_*$ is simply the \say{forgetful} map, which evaluates elements of $G\frp \tilde{G}$ by identifying $G$ and $\tilde{G}$. This can be seen by composing the inclusion $\iota_U$ with the deformation retract we used to compute $\pi_1(U,m_0)$. We claim as well that $(\iota_V)_*$ acts by mapping $\gamma\mapsto \gamma$ for $\gamma\in G$ and $\tilde{\gamma}\mapsto t^{-1}f_*(\gamma)t$. To see this, we note that if $\tilde{\gamma}\in \tilde{G}$ (corresponding to $X\times(2/3,1]$), we may \say{push $\gamma$ though} $X\times[0]$ so that after the homotopy, $\gamma$ is a path $\alpha$ from $m_0$ \say{the long way} to $(x_0,1/4)$ concatenated with $f_*(\gamma)$ lying in $X\times \{1/4\}$, then concatenated with $\alpha^{-1}$. We let $\beta$ be a path from $(x_0,1/4)$ to $m_0$ \say{the short way} and \say{stretch} the image of the path we've constructed to $\alpha^{-1}\beta^{-1}f_*(\gamma) \beta\alpha$. Noting $\beta \alpha=t\in \ZZ$ shows our claim. Then, quotienting by $(\iota_U)_*(\iota_V)_*^{-1}$ identifies $G$ and $\hat{G}$ and induces the relation of the statement of the proposition. 
\end{proof}
\prt{6}
\begin{proposition}
	$\ZZ\ltimes G\cong \ZZ\frp G/\langle tgt^{-1}\phi(g^{-1})\rangle$
\end{proposition}
\begin{proof}
	We note that $\ZZ\ltimes G$ is clearly a quotient of the free product $\ZZ\frp G$ as it is generated by the same set $G\sqcup \ZZ$, identifying $\ZZ$ with $(\ZZ,[1])$ and identifying $G$ with $(0,G)$. As all commutation relations between elements of $G$ or between elements of $\ZZ$ are predetermined respectively by the group structure, what is left is to determine commutation relations between elements of $G$ and $\ZZ$. From the definition of $\ZZ\ltimes G$, we have that $gt=(0,g)(t,[1])=(t,\phi(g))=(t,[1])(0,\phi(g))=t\phi(g)$. Thus, $t^{-1}gt=\phi(g)$.  
\end{proof}
\prt{7} \begin{proposition*} We suppose $f$ is a homotopy equivalence. Then,
$\pi_1(M_f,m_0)\cong \ZZ\ltimes G$
\end{proposition*}
\begin{proof}
Follows immediately from the presentation of $\pi_1(M_f,m_0)$ given above as well as the observation that if $f$ is a homotopy equivalence, then $f_*$ is an automorphism. \end{proof}

\prob{2}
\begin{proposition*}
	Any map $f:\RR P^2\to S^1$ is nullhomotopic.
\end{proposition*}
\begin{proof}
	We note that $\pi_1(\RR P^2,x_0)\cong\ZZ/2$, which is a torsion group and hence has no non-zero quotient isomorphic to a subgroup of $\ZZ\cong \pi_1(S_1,0)$. Thus, $f_*$ is the zero map for any such map. Hence, by the lifting criterion, there exists a lift $\tilde{f}:\RR P^2\to \RR$. However, moreso than simply connected, $\RR$ is contractible. We let $\tilde{H}$ be that contraction and let $H:\RR P^2 \times I\to S^1$ be given by $H:=p\circ \tilde H\circ \tilde f$. Then, $H$ is a nullhomotopy of $f=p\circ \tilde f$. 
\end{proof}
\end{document}
