\documentclass{article}
\usepackage{amsmath}
\usepackage{tikz}
\usepackage{amsfonts}
\usepackage{amsthm}
\usepackage{mathtools}
\usepackage[top=1in, bottom=1in, left=1in, right=1in]{geometry}
\usepackage{fancyhdr}

\thispagestyle{fancyplain} \lhead{Cutting and Pasting Example}
\chead{}
 \rhead{Katie Gedeon}

\usetikzlibrary{calc}
\usetikzlibrary{backgrounds}
\usepgflibrary{shapes}
\usetikzlibrary{through}

\pgfdeclarelayer{background}
\pgfdeclarelayer{foreground}
\pgfsetlayers{background,main,foreground}
\usetikzlibrary{arrows, decorations.pathreplacing, decorations.markings}


\begin{document}

\tikzset{
  % style to apply some styles to each segment of a path
  on each segment/.style={
    decorate,
    decoration={
      show path construction,
      moveto code={},
      lineto code={
        \path [#1]
        (\tikzinputsegmentfirst) -- (\tikzinputsegmentlast);
      },
      curveto code={
        \path [#1] (\tikzinputsegmentfirst)
        .. controls
        (\tikzinputsegmentsupporta) and (\tikzinputsegmentsupportb)
        ..
        (\tikzinputsegmentlast);
      },
      closepath code={
        \path [#1]
        (\tikzinputsegmentfirst) -- (\tikzinputsegmentlast);
      },
    },
  },
  % style to add an arrow in the middle of a path
  mid arrow/.style={postaction={decorate,decoration={
        markings,
        mark=at position .5 with {\arrow[#1]{stealth}}
      }}},
}




\tikzstyle{main node}=[circle, draw, fill=black,
                        inner sep=0pt, minimum width=2pt]

\tikzstyle{vecArrow} = [thick, decoration={markings,mark=at position
   1 with {\arrow[semithick]{open triangle 60}}},
   double distance=1.4pt, shorten >= 5.5pt,
   preaction = {decorate},
   postaction = {draw,line width=1.4pt, white,shorten >= 4.5pt}]

\tikzstyle{innerWhite} = [semithick, white,line width=3pt, shorten >= 4.5pt]











\begin{enumerate}
\item Show that the connected sum of a torus $T$ and the projective plane $\mathbb{R}\mathbf{P}^{2}$ is homeomorphic to the connected sum of three copies of $\mathbb{R}\mathbf{P}^2$.
\begin{proof}
Let $T \# \mathbb{R}\mathbf{P}^{2}$ be given in the first picture below. Then by a chain of cuttings and gluings:
%\input{trp2}
We see that $T \# \mathbb{R}\mathbf{P}^{2} \simeq \mathbb{R}\mathbf{P}^{2} \# \mathbb{R}\mathbf{P}^{2}\# \mathbb{R}\mathbf{P}^{2}$ as desired.
\end{proof}


\item Let $X$ be a surface obtained by pasting edges of an $8$-sided polygon with labeling scheme
\[
a_1a_2a_3a_4a_1a_4^{-1}a_3a_2^{-1}.
\] 
To which standard surface is $X$ homeomorphic?

\begin{proof}
 We consider the polygon with the given labeling scheme, and follow a chain of cuttings and gluings:
%\input{rp2}
Hence, $X$ is homeomorphic to the connected sum of three copies of $\mathbb{R}\mathbf{P}^{2}$.

\end{proof}


\item I'm not sure what problem I was working on this code for, but here it is anyway.
%\input{other}


\end{enumerate}
\end{document}