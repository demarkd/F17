\documentclass[english]{article}
\newcommand{\G}{\overline{C_{2k-1}}}
\usepackage[latin9]{inputenc}
\usepackage{amsmath}
\usepackage{amssymb}
\usepackage{lmodern}
\usepackage{mathtools}
\usepackage{enumitem}
%\usepackage{natbib}
%\bibliographystyle{plainnat}
%\setcitestyle{authoryear,open={(},close={)}}
\let\avec=\vec
\renewcommand\vec{\mathbf}
\renewcommand{\d}[1]{\ensuremath{\operatorname{d}\!{#1}}}
\newcommand{\pydx}[2]{\frac{\partial #1}{\partial #2}}
\newcommand{\dydx}[2]{\frac{\d #1}{\d #2}}
\newcommand{\ddx}[1]{\frac{\d{}}{\d{#1}}}
\newcommand{\hk}{\hat{K}}
\newcommand{\hl}{\hat{\lambda}}
\newcommand{\ol}{\overline{\lambda}}
\newcommand{\om}{\overline{\mu}}
\newcommand{\all}{\text{all }}
\newcommand{\valph}{\vec{\alpha}}
\newcommand{\vbet}{\vec{\beta}}
\newcommand{\vT}{\vec{T}}
\newcommand{\vN}{\vec{N}}
\newcommand{\vB}{\vec{B}}
\newcommand{\vX}{\vec{X}}
\newcommand{\vx}{\vec {x}}
\newcommand{\vn}{\vec{n}}
\newcommand{\vxs}{\vec {x}^*}
\newcommand{\vV}{\vec{V}}
\newcommand{\vTa}{\vec{T}_\alpha}
\newcommand{\vNa}{\vec{N}_\alpha}
\newcommand{\vBa}{\vec{B}_\alpha}
\newcommand{\vTb}{\vec{T}_\beta}
\newcommand{\vNb}{\vec{N}_\beta}
\newcommand{\vBb}{\vec{B}_\beta}
\newcommand{\bvT}{\bar{\vT}}
\newcommand{\ka}{\kappa_\alpha}
\newcommand{\ta}{\tau_\alpha}
\newcommand{\kb}{\kappa_\beta}
\newcommand{\tb}{\tau_\beta}
\newcommand{\hth}{\hat{\theta}}
\newcommand{\evat}[3]{\left. #1\right|_{#2}^{#3}}
\newcommand{\prompt}[1]{\begin{prompt*}
		#1
\end{prompt*}}
\newcommand{\vy}{\vec{y}}
\DeclareMathOperator{\sech}{sech}
\DeclarePairedDelimiter\abs{\lvert}{\rvert}%
\DeclarePairedDelimiter\norm{\lVert}{\rVert}%
\newcommand{\dis}[1]{\begin{align}
	#1
	\end{align}}
\newcommand{\LL}{\mathcal{L}}
\newcommand{\RR}{\mathbb{R}}
\newcommand{\NN}{\mathbb{N}}
\newcommand{\ZZ}{\mathbb{Z}}
\newcommand{\QQ}{\mathbb{Q}}
\newcommand{\Ss}{\mathcal{S}}
\newcommand{\BB}{\mathcal{B}}
\usepackage{graphicx}
% Swap the definition of \abs* and \norm*, so that \abs
% and \norm resizes the size of the brackets, and the 
% starred version does not.
%\makeatletter
%\let\oldabs\abs
%\def\abs{\@ifstar{\oldabs}{\oldabs*}}
%
%\let\oldnorm\norm
%\def\norm{\@ifstar{\oldnorm}{\oldnorm*}}
%\makeatother
\newenvironment{subproof}[1][\proofname]{%
	\renewcommand{\qedsymbol}{$\blacksquare$}%
	\begin{proof}[#1]%
	}{%
	\end{proof}%
}

\usepackage{centernot}
\usepackage{dirtytalk}
\usepackage{calc}
\newcommand{\prob}[1]{\setcounter{section}{#1-1}\section{}}


\newcommand{\prt}[1]{\setcounter{subsection}{#1-1}\subsection{}}
\newcommand{\pprt}[1]{{\textit{{#1}.)}}\newline}
\renewcommand\thesubsection{\alph{subsection}}
\usepackage[sl,bf,compact]{titlesec}
\titlelabel{\thetitle.)\quad}
\DeclarePairedDelimiter\floor{\lfloor}{\rfloor}
\makeatletter

\newcommand*\pFqskip{8mu}
\catcode`,\active
\newcommand*\pFq{\begingroup
	\catcode`\,\active
	\def ,{\mskip\pFqskip\relax}%
	\dopFq
}
\catcode`\,12
\def\dopFq#1#2#3#4#5{%
	{}_{#1}F_{#2}\biggl(\genfrac..{0pt}{}{#3}{#4}|#5\biggr
	)%
	\endgroup
}
\def\res{\mathop{Res}\limits}
% Symbols \wedge and \vee from mathabx
% \DeclareFontFamily{U}{matha}{\hyphenchar\font45}
% \DeclareFontShape{U}{matha}{m}{n}{
%       <5> <6> <7> <8> <9> <10> gen * matha
%       <10.95> matha10 <12> <14.4> <17.28> <20.74> <24.88> matha12
%       }{}
% \DeclareSymbolFont{matha}{U}{matha}{m}{n}
% \DeclareMathSymbol{\wedge}         {2}{matha}{"5E}
% \DeclareMathSymbol{\vee}           {2}{matha}{"5F}
% \makeatother

%\titlelabel{(\thesubsection)}
%\titlelabel{(\thesubsection)\quad}
\usepackage{listings}
\lstloadlanguages{[5.2]Mathematica}
\usepackage{babel}
\newcommand{\ffac}[2]{{(#1)}^{\underline{#2}}}
\usepackage{color}
\usepackage{amsthm}
\newtheorem{theorem}{Theorem}[section]
%\newtheorem*{theorem*}{Theorem}[section]
\newtheorem{conj}[theorem]{Conjecture}
\newtheorem{corollary}[theorem]{Corollary}
\newtheorem{example}[theorem]{Example}
\newtheorem{lemma}[theorem]{Lemma}
\newtheorem*{lemma*}{Lemma}
\newtheorem{problem}[theorem]{Problem}
\newtheorem{proposition}[theorem]{Proposition}
\newtheorem*{proposition*}{Proposition}
\newtheorem*{corollary*}{Corollary}
\newtheorem{fact}[theorem]{Fact}
\newtheorem*{prompt*}{Prompt}
\newtheorem*{claim*}{Claim}
\newcommand{\claim}[1]{\begin{claim*} #1\end{claim*}}
%organizing theorem environments by style--by the way, should we really have definitions (and notations I guess) in proposition style? it makes SO much of our text italicized, which is weird.
\theoremstyle{remark}
\newtheorem{remark}{Remark}[section]

\theoremstyle{definition}
\newtheorem{definition}[theorem]{Definition}
\newtheorem{notation}[theorem]{Notation}
\newtheorem*{notation*}{Notation}
%FINAL
\newcommand{\due}{9 October 2017} 
\RequirePackage{geometry}
\geometry{margin=.7in}
\usepackage{todonotes}
\title{MATH 8301 Homework V}
\author{David DeMark}
\date{\due}
\usepackage{fancyhdr}
\pagestyle{fancy}
\fancyhf{}
\rhead{David DeMark}
\chead{\due}
\lhead{MATH 8301}
\cfoot{\thepage}
% %%
%%
%%
%DRAFT

%\usepackage[left=1cm,right=4.5cm,top=2cm,bottom=1.5cm,marginparwidth=4cm]{geometry}
%\usepackage{todonotes}
% \title{MATH 8669 Homework 4-DRAFT}
% \usepackage{fancyhdr}
% \pagestyle{fancy}
% \fancyhf{}
% \rhead{David DeMark}
% \lhead{MATH 8669-Homework 4-DRAFT}
% \cfoot{\thepage}

%PROBLEM SPEFICIC

\newcommand{\lint}{\underline{\int}}
\newcommand{\uint}{\overline{\int}}
\newcommand{\hfi}{\hat{f}^{-1}}
\newcommand{\tfi}{\tilde{f}^{-1}}
\newcommand{\tsi}{\tilde{f}^{-1}}
\newcommand{\PP}{\mathcal{P}}
\newcommand{\nin}{\centernot\in}
\newcommand{\seq}[1]{({#1}_n)_{n\geq 1}}
\newcommand{\Tt}{\mathcal{T}}
\newcommand{\card}{\mathrm{card}}
\newcommand{\setc}[2]{\{ #1\::\:#2 \}}
\newcommand{\Fcal}{\mathcal{F}}
\newcommand{\cbal}{\overline{B}}
\newcommand{\Ccal}{\mathcal{C}}
\newcommand{\Dcal}{\mathcal{D}}
\newcommand{\cl}{\overline}
\newcommand{\id}{\mathrm{id}}
\newcommand{\intr}{\mathrm{int}}
\renewcommand{\hom}{\mathrm{Hom}}
\newcommand{\vect}{\mathrm{Vect}}
\newcommand{\Top}{\mathrm{Top}}
\renewcommand{\top}{\Top}
\newcommand{\hTop}{\mathrm{hTop}}
\newcommand{\set}{\mathrm{Set}}
\begin{document}
	\maketitle
\prob{1} We let $\Ccal$ be the category with a single object $*$ and morphisms $\hom(*,*)=G$.\prt{1} \begin{proposition*}
For functor $F:\Ccal\to \mathrm{Vect}_k$, show that $V:=F(*)$ is a representation of $V$. 
\end{proposition*}
\begin{proof}
We define a representation of a group $G$ as a pair $(V,\phi)$ where $V$ is a vector space and $\phi:G\to GL(V)$ is a homomorphism. We note that the morphisms of $\vect_k$ are linear maps, and further, as $F$ takes morphisms $A\to B$ to morphsims $F(A)\to F(B)$, we have that the only morphisms in the image of $F$ lie in $\hom(F(*),F(*)$. Further, we note that for all $g\in \hom_\Ccal(*,*)$, there exists some $g^{-1}$ with $g\circ g^{-1}=g^{-1}\circ g=\id_{*}$; as $F$ preserves identity and respects composition, we have that $F(g)\circ F(g^{-1})=F(g^{-1})\circ F(g)=\id_{V})$, that is, $F(g)$ is an invertible linear transformation. Thus, $F$ induces a map $\hom(*,*)\to GL(V)$, and as $F$ respects composition and preserves identity, $F$ may be viewed as a homomorphism. This completes our proof.
\end{proof}
\prt{2}
\begin{prompt*}
	Conversely, show a functor can be constructed from a representation $(V,\phi)$.
\end{prompt*}
\begin{proof}
	We let $F:\Ccal\to \vect_k$ be our functor and let $F(*)=V$, with $F:\hom_\Ccal(*,*)\to \hom_{\vect_k}(V,V)$ being given by $F(g)=\phi(g)$. Then, as $\phi$ respects the group operation, it respects composition, and as $\phi$ respects the group identity, $F$ respects the identity morphism and is hence functorial.
\end{proof}
\prob{2}
%\todo[inline]{rewrite all this to be two-sided, stupid.}
\prt{1}\begin{proposition*}
	We let $f:A\to B$ be a morphism in the category $\set$. Then, $f$ is a [categorical] isomorphism if and only if $f$ is a bijection.
\end{proposition*}
\begin{proof}
	We suppose that $a,b\in A$ with $f(a)=f(b)$. Then, $\id_A(a)=g(f(a))=g(f(b))=\id_A(b)$, so $a=b$ and $f$ is injective. On the other hand, we suppose there exists some $b\in B$ such that $b\not \in \mathrm{im}f$. Then $f(g(b))\neq b$, so we have established a contradiction and $f$ is surjective and thus bijective.
	
	On the other hand, we suppose $f$ is a bijection. Then, $f$ is invertible, with unique two-sided inverse $f^{-1}$, with $ff^{-1}=\id_A$ and $f^{-1}f=\id_B$. This completes our proof.
\end{proof}
\prt{2}\begin{proposition*}
	We let $f:A\to B$ be a morphism in the category $\top$. $f$ is an isomorphism if and only if it is a homeomorphism.
\end{proposition*}
\begin{proof}
	We have that $\top$ is a concrete category and hence a subcategory of $\set$. Thus, the isomorphisms of $\set$ are bijections, and as the morphisms of $\set$ are continuous maps, we have that $f,g$ are continuous bijections and thus homeomorphisms. 
	
	On the other hand, we suppose $f$ is a homeomorphism. Then, by definition, there exists a unique continuous two-sided inverse $f^{-1}$ such that $ff^{-1}=\id_A$ and $f^{-1}f=\id_B$, so $f$ is an isomorphism.
\end{proof}
\prt{3}\begin{proposition*}
	We let $f:A\to B$ be a morphism in the category $\hTop$. $f$ is an isomorphism if and only if it is a homotopy equivalence.
\end{proposition*}
\begin{proof}
	We suppose $f$ is an isomorphism. Then, there exists some map $g:B\to A$ such that $fg\approxeq \id_B$ and $gf\approxeq \id_A$. Then, by definition, $f$ is a homotopy equivalence. The converse follows just as obviously.
\end{proof}
\prt{4}
\begin{proposition*}
	All maps of the category defined in problem 1 are isomorphisms.
\end{proposition*}
\begin{proof}
	By definition, for all $g\in G$, there exists some element $g^{-1}$ such that $gg^{-1}=g^{-1}g=\id_G$. Then, viewing $g,g^{-1}$ as morphisms in $\hom_\Ccal(*,*)$, $g$ is an isomorphism.
\end{proof}
\prt{5}
\begin{proposition*}
	For $F:\Ccal\to\Dcal$ a functor, $f:X\to Y$ an isomorphism in $\Ccal$, $F(f)$ is an isomorphism in $\Dcal$. 
\end{proposition*}
\begin{proof}
	We have that $F(\id_\Ccal)=\id_\Dcal$, and $F(\beta\circ \alpha)=F(\beta)\circ F(\alpha)$. Thus, if $f:A\to B$ is an isomorphism in $\Ccal$ such that $g\circ f=\id_A$ and $f\circ g=\id_B$, we have that $F(f)\circ F(g)=F(f\circ g)=F(\id_B)=\id_{F(B)}$, and  $F(g)\circ F(f)=F(g\circ f)=F(\id_A)=\id_{F(A)}$. Thus, $F(f)$ is an isomorphism in $\Dcal$
\end{proof}
\prob{3}
\prt{1}Let $X$ and $Y$ be topological spaces. \begin{proposition*}
	For $f:X\to Y$ continuous, the induced map $\phi_0(f):[x]\mapsto[f(x)]$ is well-defined.
\end{proposition*}
\begin{proof}
	We wish to show that for any $x~y$, $f(x)=f(y)$. We let $\phi:[0,1]\to X$ be a path from $x$ to $y$. Then, $(f\circ\phi):[0,1]\to Y$ is continuous with $(f\circ\phi)(0)=f(x)$ and $(f\circ \phi)(1)=f(y)$. Hence, $f\circ \phi$ is a path connection from $f(x)$ to $f(y)$, so $f(x)\circ f(y)$ and $\pi_0(f)$ is well-defined.
\end{proof}
\prt{2}\begin{proposition*}
	$\pi_0$ defines a functor $\pi_0:\top\to\set$.
\end{proposition*}
\begin{proof}
	There are two things to show: \emph{(i)} for any $X\in \top$, $\pi_0(\id_X)=\id_{\pi_0(x)}$, and \emph{(ii)} for $f\in \hom_\top(A,B)$, $g\in \hom_{\top}(B\to C)$, $\pi_0(g\circ f)=\pi_0(g)\circ \pi_0(f)$
	\begin{enumerate}[label=\emph{(\roman*)}]
		\item $\pi_0(\id_X):[x]\mapsto [\id_X(x)]=[x]$ simply sends an equivalence class to itself, and is hence the identity morphism on $\pi_0(X)$. 
		\item $(\pi_0(g)\circ \pi_0(f)):x\mapsto [f(x)]\mapsto \pi_0(g)[f(x)]=[g(f(x))]=\pi_0*(f\circ g)[x]$
	\end{enumerate}
This completes our proof.
\end{proof}
\prt{3} \begin{corollary*}
	$\pi_0$ defines a functor $\pi_0:\hTop\to\set$.
\end{corollary*}
\begin{proof}
	We have already shown the functoriality of $\pi_0$ as a functor from $\top$ to $\set$, so what is left to show is that $\pi_0$ respects equivalence classes of morphisms, that is, for $f,g\in \hom_{\top}(A,B)$ with $f\approxeq g$, $\pi_0(f)=\pi_0(g)$. We let $H:I\times A\to B$ be a homotopy from $f$ to $g$, with $H(0,x)=f(x)$ and $H(1,x)=g(x)$. Then, for any $x_0\in A$, $h:I\to B$ given by $t\mapsto H(t,x_0)$ gives a continuous map with $h(0)=f(x_0)$ and $h(1)=g(x_0)$ and is hence a path-connection from $f(x_0)$ to $g(x_0)$, so $f(x_0)\sim g(x_0)$ and $\pi_0(f)=\pi_0(g)$. Thus, $\pi_0$ is well-defined on morphisms of $\hTop$, and is hence functorial as inherited from $\top$.
\end{proof}
\prt{4}\begin{proposition*}
	If $X\approxeq Y$, then $\#\pi_0(X)=\#\pi_0(Y)$
\end{proposition*}
\begin{proof}
If $X\approxeq Y$, then there exists some isomorphism (homotopy equivalence) $f\in \hom_{\hTop}(X,Y)$. Then, by question 2(e), $\pi_0(f)\in \hom_{\set}(\pi_0(X),\pi_0(Y))$ is an isomorphism in $\set$, and by problem 2(b), $\pi_0(f)$ is a bijection. This completes our proof.
\end{proof}
\end{document}
