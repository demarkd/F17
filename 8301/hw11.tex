\documentclass[english]{article}
\newcommand{\G}{\overline{C_{2k-1}}}
\usepackage[latin9]{inputenc}
\usepackage{amsmath}
\usepackage{amssymb}
\usepackage{lmodern}
\usepackage{mathtools}
\usepackage{enumitem}
\usepackage{relsize}

%\usepackage{natbib}
%\bibliographystyle{plainnat}
%\setcitestyle{authoryear,open={(},close={)}}
\let\avec=\vec
\renewcommand\vec{\mathbf}
\renewcommand{\d}[1]{\ensuremath{\operatorname{d}\!{#1}}}
\newcommand{\pydx}[2]{\frac{\partial #1}{\partial #2}}
\newcommand{\dydx}[2]{\frac{\d #1}{\d #2}}
\newcommand{\ddx}[1]{\frac{\d{}}{\d{#1}}}
\newcommand{\hk}{\hat{K}}
\newcommand{\hl}{\hat{\lambda}}
\newcommand{\ol}{\overline{\lambda}}
\newcommand{\om}{\overline{\mu}}
\newcommand{\all}{\text{all }}
\newcommand{\valph}{\vec{\alpha}}
\newcommand{\vbet}{\vec{\beta}}
\newcommand{\vT}{\vec{T}}
\newcommand{\vN}{\vec{N}}
\newcommand{\vB}{\vec{B}}
\newcommand{\vX}{\vec{X}}
\newcommand{\vx}{\vec {x}}
\newcommand{\vn}{\vec{n}}
\newcommand{\vxs}{\vec {x}^*}
\newcommand{\vV}{\vec{V}}
\newcommand{\vTa}{\vec{T}_\alpha}
\newcommand{\vNa}{\vec{N}_\alpha}
\newcommand{\vBa}{\vec{B}_\alpha}
\newcommand{\vTb}{\vec{T}_\beta}
\newcommand{\vNb}{\vec{N}_\beta}
\newcommand{\vBb}{\vec{B}_\beta}
\newcommand{\bvT}{\bar{\vT}}
\newcommand{\ka}{\kappa_\alpha}
\newcommand{\ta}{\tau_\alpha}
\newcommand{\kb}{\kappa_\beta}
\newcommand{\tb}{\tau_\beta}
\newcommand{\hth}{\hat{\theta}}
\newcommand{\evat}[3]{\left. #1\right|_{#2}^{#3}}
\newcommand{\restr}[2]{\evat{#1}{#2}{}}
\newcommand{\prompt}[1]{\begin{prompt*}
		#1
\end{prompt*}}
\newcommand{\vy}{\vec{y}}
\DeclareMathOperator{\sech}{sech}
\DeclarePairedDelimiter\abs{\lvert}{\rvert}%
\DeclarePairedDelimiter\norm{\lVert}{\rVert}%
\newcommand{\dis}[1]{\begin{align}
	#1
	\end{align}}
\newcommand{\LL}{\mathcal{L}}
\newcommand{\RR}{\mathbb{R}}
\newcommand{\CC}{\mathbb{C}}
\newcommand{\NN}{\mathbb{N}}
\newcommand{\ZZ}{\mathbb{Z}}
\newcommand{\QQ}{\mathbb{Q}}
\newcommand{\Ss}{\mathcal{S}}
\newcommand{\BB}{\mathcal{B}}
\usepackage{graphicx}
% Swap the definition of \abs* and \norm*, so that \abs
% and \norm resizes the size of the brackets, and the 
% starred version does not.
%\makeatletter
%\let\oldabs\abs
%\def\abs{\@ifstar{\oldabs}{\oldabs*}}
%
%\let\oldnorm\norm
%\def\norm{\@ifstar{\oldnorm}{\oldnorm*}}
%\makeatother
\newenvironment{subproof}[1][\proofname]{%
	\renewcommand{\qedsymbol}{$\blacksquare$}%
	\begin{proof}[#1]%
	}{%
	\end{proof}%
}

\usepackage{centernot}
\usepackage{dirtytalk}
\usepackage{calc}
\newcommand{\prob}[1]{\setcounter{section}{#1-1}\section{}}


\newcommand{\prt}[1]{\setcounter{subsection}{#1-1}\subsection{}}
\newcommand{\pprt}[1]{{\textit{{#1}.)}}\newline}
\renewcommand\thesubsection{\alph{subsection}}
\usepackage[sl,bf,compact]{titlesec}
\titlelabel{\thetitle.)\quad}
\DeclarePairedDelimiter\floor{\lfloor}{\rfloor}
\makeatletter

\newcommand*\pFqskip{8mu}
\catcode`,\active
\newcommand*\pFq{\begingroup
	\catcode`\,\active
	\def ,{\mskip\pFqskip\relax}%
	\dopFq
}
\catcode`\,12
\def\dopFq#1#2#3#4#5{%
	{}_{#1}F_{#2}\biggl(\genfrac..{0pt}{}{#3}{#4}|#5\biggr
	)%
	\endgroup
}
\def\res{\mathop{Res}\limits}
% Symbols \wedge and \vee from mathabx
% \DeclareFontFamily{U}{matha}{\hyphenchar\font45}
% \DeclareFontShape{U}{matha}{m}{n}{
%       <5> <6> <7> <8> <9> <10> gen * matha
%       <10.95> matha10 <12> <14.4> <17.28> <20.74> <24.88> matha12
%       }{}
% \DeclareSymbolFont{matha}{U}{matha}{m}{n}
% \DeclareMathSymbol{\wedge}         {2}{matha}{"5E}
% \DeclareMathSymbol{\vee}           {2}{matha}{"5F}
% \makeatother

%\titlelabel{(\thesubsection)}
%\titlelabel{(\thesubsection)\quad}
\usepackage{listings}
\lstloadlanguages{[5.2]Mathematica}
\usepackage{babel}
\newcommand{\ffac}[2]{{(#1)}^{\underline{#2}}}
\usepackage{color}
\usepackage{amsthm}
\newtheorem{theorem}{Theorem}[section]
\newtheorem*{theorem*}{Theorem}
\newtheorem{conj}[theorem]{Conjecture}
\newtheorem{corollary}[theorem]{Corollary}
\newtheorem{example}[theorem]{Example}
\newtheorem{lemma}[theorem]{Lemma}
\newtheorem*{lemma*}{Lemma}
\newtheorem{problem}[theorem]{Problem}
\newtheorem{proposition}[theorem]{Proposition}
\newtheorem*{proposition*}{Proposition}
\newtheorem*{corollary*}{Corollary}
\newtheorem{fact}[theorem]{Fact}
\newtheorem*{prompt*}{Prompt}
\newtheorem*{claim*}{Claim}
\newtheorem{claim}{Claim}
%\newcommand{\claim}[1]{\begin{claim*} #1\end{claim*}}
%organizing theorem environments by style--by the way, should we really have definitions (and notations I guess) in proposition style? it makes SO much of our text italicized, which is weird.
\theoremstyle{remark}
\newtheorem{remark}{Remark}[section]

\theoremstyle{definition}
\newtheorem{definition}[theorem]{Definition}
\newtheorem*{definition*}{Definition}
\newtheorem{notation}[theorem]{Notation}
\newtheorem*{notation*}{Notation}
%FINAL
\newcommand{\due}{6 December 2017} 
\RequirePackage{geometry}
\geometry{margin=.7in}
\usepackage{todonotes}
\title{MATH 8301 Homework XI}
\author{David DeMark}
\date{\due}
\usepackage{fancyhdr}
\pagestyle{fancy}
\fancyhf{}
\rhead{David DeMark}
\chead{\due}
\lhead{MATH 8301}
\cfoot{\thepage}
% %%
%%
%%
%DRAFT

%\usepackage[left=1cm,right=4.5cm,top=2cm,bottom=1.5cm,marginparwidth=4cm]{geometry}
%\usepackage{todonotes}
% \title{MATH 8669 Homework 4-DRAFT}
% \usepackage{fancyhdr}
% \pagestyle{fancy}
% \fancyhf{}
% \rhead{David DeMark}
% \lhead{MATH 8669-Homework 4-DRAFT}
% \cfoot{\thepage}

%PROBLEM SPEFICIC

\newcommand{\lint}{\underline{\int}}
\newcommand{\uint}{\overline{\int}}
\newcommand{\hfi}{\hat{f}^{-1}}
\newcommand{\tfi}{\tilde{f}^{-1}}
\newcommand{\tsi}{\tilde{f}^{-1}}
\newcommand{\PP}{\mathcal{P}}
\newcommand{\nin}{\centernot\in}
\newcommand{\seq}[1]{({#1}_n)_{n\geq 1}}
\newcommand{\Tt}{\mathcal{T}}
\newcommand{\card}{\mathrm{card}}
\newcommand{\setc}[2]{\{ #1\::\:#2 \}}
\newcommand{\Fcal}{\mathcal{F}}
\newcommand{\cbal}{\overline{B}}
\newcommand{\Ccal}{\mathcal{C}}
\newcommand{\Dcal}{\mathcal{D}}
\newcommand{\cl}{\overline}
\newcommand{\id}{\mathrm{id}}
\newcommand{\intr}{\mathrm{int}}
\renewcommand{\hom}{\mathrm{Hom}}
\newcommand{\vect}{\mathrm{Vect}}
\newcommand{\Top}{\mathrm{Top}}
\renewcommand{\top}{\Top}
\newcommand{\hTop}{\mathrm{hTop}}
\newcommand{\set}{\mathrm{Set}}
\newcommand{\frp}{\mathop{\large {\mathlarger{*}}}}
\newcommand{\ondt}{1_{\cdot}}
\newcommand{\onst}{1_{\star}}
\newcommand{\bdy}{\partial}
\newcommand{\im}{\mathrm{im}}
\newcommand{\re}{\mathrm{re}}
\newcommand{\oX}{\overline{X}}
\newcommand{\ox}{\overline{x}}
\newcommand{\tX}{\tilde{X}}
\newcommand{\tx}{\tilde{x}}
\newcommand{\hX}{\hat{X}}
\newcommand{\hx}{\hat{x}}
\newcommand{\aut}{\mathrm{Aut}}
\newcommand{\del}{\partial}
\begin{document}
\maketitle
\emph{Collaborators: Sarah Brauner.}

\noindent\textbf{All references to theorems come from Allen Hatcher's \emph{Algebraic Topology} unless otherwise stated.}
\prob{1}
We let $\oX$ be path connected, $G$ a group action on $X$, and $p:\oX\to X:=\oX/G$ a covering map
\prt{1}
\begin{proposition*}
	For covering space maps $\oX\to \tX\to X$, $\tX\cong \oX/H$ for some $H<G$. 
\end{proposition*}
\begin{proof}
	We let $\hX\to X$ be the universal cover of $X$ and first restrict our attention to the case $\oX=\hX$. Then, proposition 1.39 implies that $F:=\aut(\hX/X)\cong \pi_1(X,x_0)$. We note that for any $H<F$, we may construct a covering map $\tX:=\hX/H\to X$, which is unique up to based isomorphism by the observation that for $\tX_1:=\hX/H_1$ and $\tX_2:=\hX/H_2$, we then have that $\aut(\hX/\tX_1)=\aut(\hX/\tX_2)$ if and only if $H_1=H_2$. Then, proposition 1.38 shows our special case, as we have shown that covering maps $\hX/H\to X$ are in bijection with subgroups of $F$, and proposition 1.38 associates subgroups of $F$ with covering maps $\oX\to X$. Proposition 1.39 then states that for $\tX=\hX/H$, $\pi_1(\tX,\tx_0)=H$, so indeed our association and that of proposition 1.38 one and the same.\footnote{Well, technically to bridge that gap we also need the fact that for $p:\tX\to X$ a covering map, $p_*:\pi_1(\tX,\tx_0)\to \pi_1(X,x_0)$ is an injection, which is given by proposition 1.31.} 
	
	We now consider the case of a general covering map $\oX\to X:=\oX/G$ with intermediate covering maps $\oX\to\tX\to X$. As before, we let $\hX\to X$ be a universal cover with $F:=\aut(\hX/X)\cong \pi_1(X,x_0)$. Then, our previous work shows $\oX=\hX/N$ for some $N<F$ with $N\cong \pi_1(\oX,\ox_0)$. We have by proposition 1.40a that $\oX\to X$ is normal and $G\cong F/N$. Then, as the subgroup poset of $F$ is dual to the covering space poset of $X$ (where $\tX\geq\oX$ iff there exists a covering $\tX\to \oX$)\footnote{We haven't quite justified this, but it is an immediate corollary of proposition 1.38 in conjunction with proposition 1.31.}, we have by our special case that intermediate coverings $\oX\to\tX\to X$ are in bijection with subgroups $\hat{H}$ such that $N<\hat{H}<F$, which are in bijection with subgroups $H<G=F/N$ by the map $\hat{H}\mapsto \hat{H}/N$. As each of the covering spaces $\oX/H\to X$ exist uniquely by the same argument of above, our proof is now complete. 
\end{proof}
\prt{2}
\begin{proposition}
	$\oX/H_1\cong \oX/H_2$ if and only if $H_1$ and $H_2$ are conjugate
\end{proposition}
\begin{proof}
Proposition 1.38 states that for covering spaces $p_1:\tX_1\to X$ and $p_2:\tX_2\to X$, $\tX_1\cong \tX_2$ if and only if $\pi_1(\tX_1)$ and $\pi_1(\tX_2)$ are conjugate. We showed that $\tX_i=\hX/\pi_1(\tX_i):=H_i$, and it is the case that all covering spaces $\hX/H$ exist. Thus, the proposition follows trivially from part a.
\end{proof}
\prt{3}
\begin{proposition}
	$\tX:=\oX/H\to X$ is normal if and only if $H\unlhd G$ in which case $\aut(\tX/X)=G/H$. 
\end{proposition}
\begin{proof}
	We have that $\tX=\oX/H=\hX/\hat{H}$ for some $\hat{H}<F$ with $N\unlhd \hat{H}\cong \pi_1(\tX,\tx_0)$ and $H=\hat{H}/N$. We have by proposition 1.39 that $\tX\to X$ is normal if and only if $p_*(\pi_1(\tX,\tx_0)\cong \hat{H}$ is normal in $\pi_1(X,x_0)$. Finally, we note that $H\unlhd G$ if and only if $\hat{H}\unlhd F$. This completes our proof.
\end{proof}
\prob{2}
\begin{proposition*}
	For any group $G$ and $N\unlhd G$, there exists a normal covering space $\oX\to X$ with $\pi_1(X)\cong G$, $\pi_1(\oX)\cong N$ and $\aut(\oX/X)\cong G/N$.
\end{proposition*}
\begin{proof}
We let $G=\langle g_\alpha \;\mid \;r_\beta \rangle$ with generators $g_\alpha$ indexed over $A$ and relations $r_\beta$ indexed over $B$. We then recall Hatcher's construction of a space $X_G$ with $\pi_1(X_G,x_0)=G$: we let $Y=\left(\bigwedge_{\alpha\in A} S^1\right) \sqcup \left(\bigsqcup_{\beta\in B} D^2\right)$, and define an equivalence relation $\sim$ wich associates the boundary of the copy of $D^2$ indexed by $\beta$ to the copies of $S^1$ associated to the letters in the word $r_\beta$. Then, $X:=Y/\sim$ fulfills the desired property, and the universal cover $\hX$ of $X$ has $\aut(\hX/X)=G$. Our argument for question 1 then ensures the existence of $\oX$ and gives its fundamental group and $\aut(\oX/X)$.
\end{proof}
\prob{3}	We let $C_\bullet $ and $D_\bullet$ be chain complexes and $f,g,h:C_\bullet\to D_{\bullet}$ chain maps. If there exists a homotopy map $P:C_n\to D_{n+1}$ such that $P\del+\del P=f-g$, we say $f\sim g$.
\begin{proposition*}
	Chain homotopy ($\sim$) is an equivalence relation.
\end{proposition*}


\begin{proof}	
	\emph{Reflexivity:} We wish to show $f\sim f$. We let $P:C_n\to D_{n+1}$ be the $0$ map. Then, $P\del+\del P=0=f-f$, so $f\sim f$.
	
	\emph{Symmetry} We suppose $f\sim g$ and let $P:C_n\to D_n$ be a chain homotopy between them. We let $P'=-P$ and then have that $P'\del+\del P'=-(P\del +\del P)=-(f-g)=g-f$ so $g\sim f$. 
	
	\emph{Transitivity} We suppose $f\sim g$ and $g\sim h$. We let $P_i$ be such that $P_1\del+\del P_1=f-g$ and $P_2\del +\del P_2=g-h$. We let $P=P_1+P_2$ and then have that $P\del+\del P=(P_1+P_2)\del+\del (P_1+P_2)=(P_1\del +\del P_1) +(P_2\del +\del P_2) =(f-g)+(g-h)=f-h$ so $f\sim h$.
\end{proof}
\prob{4}
\begin{proposition*}
We suppose $A\subset X$ with inclusion map $\iota:A\to X$ and that $X$ retracts onto $A$. Then, $\iota_*:H_*(A)\to H_*(X)$ is an injection. 
\end{proposition*}

\begin{proof} We suppose
	$X$ retracts onto $A$ via the map $\tau$, i.e. $\tau:X\to A$ is such that $\tau\circ \iota=\id_A$. We recall that $H_*(-)$ is \emph{functorial,} that is, it preserves composition and the identity map. Thus, $\id_{H_*(A)}=H_*(\id_A)=H_*(\tau\circ \iota)=\tau_*\circ \iota_*$. We recall the following category-theoretic fact with obvious proof: if $\phi\in \hom(A,B)$, $\psi\in \hom(B,C)$ and $\psi\circ \phi$ is an isomorphism, then $\phi$ is a monomorphism and $\psi$ is an epimorphism. Hence, $\iota_*$ is an injection, and though we have already completed our proof, we may as well note that $\tau_*$ is surjective. 
\end{proof}
\prob{5} \begin{prompt*}
	Let $A$ be a finitely generated Abelian group. Construct a chain complex $C_\bullet$ such that $H_0(C_\bullet)=A$ but $H_i(C_\bullet)=0$ for $i\neq 0$.
\end{prompt*}
\begin{proof}[Response]
	By the structure theorem for Abelian groups, $A=\bigoplus_{k=1}^n \ZZ/{m_k}$ for some $m_k\geq 0$. We let $N=\langle (0,\hdots,m_k,\hdots,0)\;\mid\; k\in [n] \rangle $, so that $A=\ZZ^n/N$. Then, we claim that the following chain complex fulfils the desired properties: we let \begin{equation}
		C_i=\begin{cases}
			N     & i=1              \\
			\ZZ^n & i=0              \\
			0     & \text{otherwise}
		\end{cases}
	\end{equation}
Now,  we have that $\del_1$ is injective (the inclusion $N\to \ZZ^n$), so $H_1(C_*)=0/0=0$. $\im \del_1=N$ and $\ker \del_0=\ZZ^n$ as $\im\del_0=0$. Thus, $H_0(C_*)=\ZZ^n/N=A$. For all other $i$, we have that $C_i=0$ so $\del_i$ is the $0$ map and hence $H_i(C_*)=0$. 
\end{proof}

\end{document}
