\documentclass[english]{article}
\newcommand{\G}{\overline{C_{2k-1}}}
\usepackage[latin9]{inputenc}
\usepackage{amsmath,calligra,mathrsfs}
\usepackage{amssymb}
\usepackage{lmodern}
\usepackage{mathtools}
\usepackage{enumitem}
\usepackage{pgf}
\usepackage{tikz}
\usepackage{tikz-cd}
\usepackage{relsize}

\usetikzlibrary{arrows, matrix}
%\usepackage{natbib}
%\bibliographystyle{plainnat}
%\setcitestyle{authoryear,open={(},close={)}}
\let\avec=\vec
\renewcommand\vec{\mathbf}
\renewcommand{\d}[1]{\ensuremath{\operatorname{d}\!{#1}}}
\newcommand{\pydx}[2]{\frac{\partial #1}{\partial #2}}
\newcommand{\dydx}[2]{\frac{\d #1}{\d #2}}
\newcommand{\ddx}[1]{\frac{\d{}}{\d{#1}}}
\newcommand{\hk}{\hat{K}}
\newcommand{\hl}{\hat{\lambda}}
\newcommand{\ol}{\overline{\lambda}}
\newcommand{\om}{\overline{\mu}}
\newcommand{\all}{\text{all }}
\newcommand{\valph}{\vec{\alpha}}
\newcommand{\vbet}{\vec{\beta}}
\newcommand{\vT}{\vec{T}}
\newcommand{\vN}{\vec{N}}
\newcommand{\vB}{\vec{B}}
\newcommand{\vX}{\vec{X}}
\newcommand{\vx}{\vec {x}}
\newcommand{\vn}{\vec{n}}
\newcommand{\vxs}{\vec {x}^*}
\newcommand{\vV}{\vec{V}}
\newcommand{\vTa}{\vec{T}_\alpha}
\newcommand{\vNa}{\vec{N}_\alpha}
\newcommand{\vBa}{\vec{B}_\alpha}
\newcommand{\vTb}{\vec{T}_\beta}
\newcommand{\vNb}{\vec{N}_\beta}
\newcommand{\vBb}{\vec{B}_\beta}
\newcommand{\bvT}{\bar{\vT}}
\newcommand{\ka}{\kappa_\alpha}
\newcommand{\ta}{\tau_\alpha}
\newcommand{\kb}{\kappa_\beta}
\newcommand{\tb}{\tau_\beta}
\newcommand{\hth}{\hat{\theta}}
\newcommand{\evat}[3]{\left. #1\right|_{#2}^{#3}}
\newcommand{\prompt}[1]{\begin{prompt*}
		#1
\end{prompt*}}
\newcommand{\vy}{\vec{y}}
\DeclareMathOperator{\sech}{sech}
\DeclareMathOperator{\Spec}{Spec}
\DeclareMathOperator{\spec}{Spec}
\DeclareMathOperator{\spm}{Spm}
\DeclareMathOperator{\rad}{rad}
\newcommand{\mor}{\mathrm{Mor}}
\newcommand{\obj}{\mathrm{Obj}~}
\DeclarePairedDelimiter\abs{\lvert}{\rvert}%
\DeclarePairedDelimiter\norm{\lVert}{\rVert}%
\newcommand{\dis}[1]{\begin{align}
	#1
	\end{align}}
\newcommand{\Aa}{\mathbb{A}}
\newcommand{\LL}{\mathcal{L}}
\newcommand{\CC}{\mathbb{C}}
\newcommand{\DD}{\mathbb{D}}
\newcommand{\RR}{\mathbb{R}}
\newcommand{\NN}{\mathbb{N}}
\newcommand{\ZZ}{\mathbb{Z}}
\newcommand{\QQ}{\mathbb{Q}}
\newcommand{\Ss}{\mathcal{S}}
\newcommand{\OO}{\mathcal{O}}
\newcommand{\BB}{\mathcal{B}}
\newcommand{\Pcal}{\mathcal{P}}
\newcommand{\FF}{\mathscr{F}}
\newcommand{\GG}{\mathscr{G}}
\newcommand{\Fcal}{\mathcal{F}}
\newcommand{\afr}{\mathfrak{a}}
\newcommand{\bfr}{\mathfrak{b}}
\newcommand{\cfr}{\mathfrak{c}}
\newcommand{\dfr}{\mathfrak{d}}
\newcommand{\efr}{\mathfrak{e}}
\newcommand{\ffr}{\mathfrak{f}}
\newcommand{\gfr}{\mathfrak{g}}
\newcommand{\hfr}{\mathfrak{h}}
\newcommand{\ifr}{\mathfrak{i}}
\newcommand{\jfr}{\mathfrak{j}}
\newcommand{\kfr}{\mathfrak{k}}
\newcommand{\lfr}{\mathfrak{l}}
\newcommand{\mfr}{\mathfrak{m}}
\newcommand{\nfr}{\mathfrak{n}}
\newcommand{\ofr}{\mathfrak{o}}
\newcommand{\pfr}{\mathfrak{p}}
\newcommand{\qfr}{\mathfrak{q}}
\newcommand{\rfr}{\mathfrak{r}}
\newcommand{\sfr}{\mathfrak{s}}
\newcommand{\tfr}{\mathfrak{t}}
\newcommand{\ufr}{\mathfrak{u}}
\newcommand{\vfr}{\mathfrak{v}}
\newcommand{\wfr}{\mathfrak{w}}
\newcommand{\xfr}{\mathfrak{x}}
\newcommand{\yfr}{\mathfrak{y}}
\newcommand{\zfr}{\mathfrak{z}}
\newcommand{\Dcal}{\mathcal{D}}
\newcommand{\Ccal}{\mathcal{C}}
\usepackage{graphicx}
% Swap the definition of \abs* and \norm*, so that \abs
% and \norm resizes the size of the brackets, and the 
% starred version does not.
%\makeatletter
%\let\oldabs\abs
%\def\abs{\@ifstar{\oldabs}{\oldabs*}}
%
%\let\oldnorm\norm
%\def\norm{\@ifstar{\oldnorm}{\oldnorm*}}
%\makeatother
\newenvironment{subproof}[1][\proofname]{%
	\renewcommand{\qedsymbol}{$\blacksquare$}%
	\begin{proof}[#1]%
	}{%
	\end{proof}%
}

\usepackage{centernot}
\usepackage{dirtytalk}
\usepackage{calc}
\newcommand{\prob}[1]{\setcounter{section}{#1-1}\section{}}


\newcommand{\prt}[1]{\setcounter{subsection}{#1-1}\subsection{}}
\newcommand{\pprt}[1]{{\textit{{#1}.)}}\newline}
\renewcommand\thesubsection{\alph{subsection}}
\usepackage[sl,bf,compact]{titlesec}
\titlelabel{\thetitle.)\quad}
\DeclarePairedDelimiter\floor{\lfloor}{\rfloor}
\makeatletter

\newcommand*\pFqskip{8mu}
\catcode`,\active
\newcommand*\pFq{\begingroup
	\catcode`\,\active
	\def ,{\mskip\pFqskip\relax}%
	\dopFq
}
\catcode`\,12
\def\dopFq#1#2#3#4#5{%
	{}_{#1}F_{#2}\biggl(\genfrac..{0pt}{}{#3}{#4}|#5\biggr
	)%
	\endgroup
}
\def\res{\mathop{Res}\limits}
% Symbols \wedge and \vee from mathabx
% \DeclareFontFamily{U}{matha}{\hyphenchar\font45}
% \DeclareFontShape{U}{matha}{m}{n}{
%       <5> <6> <7> <8> <9> <10> gen * matha
%       <10.95> matha10 <12> <14.4> <17.28> <20.74> <24.88> matha12
%       }{}
% \DeclareSymbolFont{matha}{U}{matha}{m}{n}
% \DeclareMathSymbol{\wedge}         {2}{matha}{"5E}
% \DeclareMathSymbol{\vee}           {2}{matha}{"5F}
% \makeatother

%\titlelabel{(\thesubsection)}
%\titlelabel{(\thesubsection)\quad}
\usepackage{listings}
\lstloadlanguages{[5.2]Mathematica}
\usepackage{babel}
\newcommand{\ffac}[2]{{(#1)}^{\underline{#2}}}
\usepackage{color}
\usepackage{amsthm}
\newtheorem{theorem}{Theorem}[section]
\newtheorem*{thm*}{Theorem}
\newtheorem{conj}[theorem]{Conjecture}
\newtheorem{corollary}[theorem]{Corollary}
\newtheorem{example}[theorem]{Example}
\newtheorem{lemma}[theorem]{Lemma}
\newtheorem*{lemma*}{Lemma}
\newtheorem{problem}[theorem]{Problem}
\newtheorem{proposition}[theorem]{Proposition}
\newtheorem*{proposition*}{Proposition}
\newtheorem*{corollary*}{Corollary}
\newtheorem{fact}[theorem]{Fact}
\newtheorem*{prompt*}{Prompt}
\newtheorem*{claim*}{Claim}
\newcommand{\claim}[1]{\begin{claim*} #1\end{claim*}}
%organizing theorem environments by style--by the way, should we really have definitions (and notations I guess) in proposition style? it makes SO much of our text italicized, which is weird.
\theoremstyle{remark}
\newtheorem{remark}{Remark}[section]

\theoremstyle{definition}
\newtheorem{definition}[theorem]{Definition}
\newtheorem*{definition*}{Definition}
\newtheorem{notation}[theorem]{Notation}
\newtheorem*{notation*}{Notation}
%FINAL
\newcommand{\due}{30 October 2017} 
\RequirePackage{geometry}
\geometry{margin=.7in}
\usepackage{todonotes}
\title{MATH 8253 Homework III}
\author{David DeMark}
\date{\due}
\usepackage{fancyhdr}
\pagestyle{fancy}
\fancyhf{}
\rhead{David DeMark}
\chead{\due}
\lhead{MATH 8253}
\cfoot{\thepage}
% %%
%%
%%
%DRAFT

%\usepackage[left=1cm,right=4.5cm,top=2cm,bottom=1.5cm,marginparwidth=4cm]{geometry}
%\usepackage{todonotes}
% \title{MATH 8669 Homework 4-DRAFT}
% \usepackage{fancyhdr}
% \pagestyle{fancy}
% \fancyhf{}
% \rhead{David DeMark}
% \lhead{MATH 8669-Homework 4-DRAFT}
% \cfoot{\thepage}

%PROBLEM SPEFICIC

\newcommand{\lint}{\underline{\int}}
\newcommand{\uint}{\overline{\int}}
\newcommand{\hfi}{\hat{f}^{-1}}
\newcommand{\tfi}{\tilde{f}^{-1}}
\newcommand{\tsi}{\tilde{f}^{-1}}
\newcommand{\PP}{\mathcal{P}}
\newcommand{\nin}{\centernot\in}
\newcommand{\seq}[1]{({#1}_n)_{n\geq 1}}
\newcommand{\Tt}{\mathcal{T}}
\newcommand{\card}{\mathrm{card}}
\newcommand{\setc}[2]{\{ #1\::\:#2 \}}
\newcommand{\idl}[1]{\langle #1 \rangle}
\newcommand{\cl}{\overline}
\newcommand{\id}{\mathrm{id} }
\newcommand{\im}{\mathrm{Im}}
\newcommand{\cat}[1]{{\mathrm{\bf{#1}}}}
%\usepackage[backend=biber,style=alphabetic]{biblatex}
%\addbibresource{algeo.bib}
\newcommand{\colim}{\varinjlim}
\newcommand{\clim}{\varprojlim}
\newcommand{\frp}{\mathop{\large {\mathlarger{\star}}}}
\newcommand{\restr}[2]{\evat{#1}{#2}{}}
\begin{document}
\maketitle
\prob{1}
\emph{We save this problem for after problem 3 so that we may use its result}
\prob{2}
\begin{proposition*}
	$\colim_{G_i,f_{ij}}G_i=G:=\frp_{i\in I} G_i/N$ where $N$ is generated by elements of the form $a_ia_j^{-1}$ where there is some $k$ such that $f_{ik}(a_i)=f_{ij}(a_j)$ for all $i\leq j$ and $\frp$ is taken to be the free product.
\end{proposition*}
\begin{proof}
	We let $\rho_j:G_j\to G$ be the inclusion map $G_j\to \frp_{i\in I}G_i$ composed with the quotient map $\frp_{i\in I}G_i\to \frp_{i\in I} G_i/N$. Then, $(G,\rho_j)$ is a co-cone as $\rho_j(f_{ij}(a))\rho_i(f_{ii}(a))^{-1}\in N\implies \rho_j(f_{ij}(a))=\rho_i(a)$. To see that it is universal, we let $(C,\sigma_j)$ be another co-cone and first note that there is a unique map $\frp_j G_j\to C$ commuting with $\sigma_j$ by the universal property of the coproduct; hence, we may consider the $\sigma_j$ maps fully determined by the map $\psi: \frp_j G_j\to C$. We then note that $\sigma_j(a)=\sigma_i(b)$ for all $(a,b)\in G_j\times G_i$ such that $f_{jk}(a)=f_{ik}(b)$ for all $i,j$ as $C$ is a co-cone. Hence, $N\trianglelefteq \ker \psi$, so there is a unique map $G\to G/(\ker \psi/N)$ commuting with $\psi$ by the universal property of the quotient map. This completes our proof.
\end{proof}
\prob{3} \begin{proposition*}
 We let $A$ be an integral domain. For any open set $U\subset X$, $\OO(U)$ is canonically isomorphic to $\bigcap_{x\in U}A_x$ (viewing $A_x$ as a subset of the ring $\mathrm{frac}(A)$). 
\end{proposition*}
\begin{proof}
	We break into two subclaims:
	\begin{claim*}
		For any \emph{basic} open set $U:=D(f)$ with $f\in A$, $\OO(U)\cong\bigcap_{x\in U}A_x$ canonically.
	\end{claim*}
\begin{subproof}
	We have that $\OO(U)=A_{S(f)}$ where $S(f)=\{g\in A\;:\;D(f)\subset D(g)\}$. We note that $\bigcap_{x\in U} A_x=\{\frac{a}{b}\in \mathrm{frac}(A)\;:\;b\notin \bigcup_{x\in U} \pfr_x\}=A[S^{-1}]$ where $S:=\left(\bigcup_{x\in U}\pfr_x\right)^c$. We let $g\in S$. Then, for all $\pfr$ such that $f\not \in \pfr$ (i.e. $x_\pfr\in D(f)=U$),$g\not \in \pfr\implies D(g)\supset D(f)\implies S\subset S(f)$. Moreover, for any $g\in S(f)$, we have that $g\notin \pfr_x$ for any $x\in U$, so $S(f)=S$. Thus, $\bigcap_{x\in U} A_x\cong A_{S(f)}$, and as $A_{S(f)}$ is the limit of all $\OO(D(g))\subset \OO(D(f))$, we have that there is a canonical isomorphism $\bigcap_{x\in U}A_x\to A_{S(f)}$ commuting with restriction maps.
\end{subproof}
\begin{claim*} For any $U\subset X$ open, $\OO(U)\cong\bigcap_{D(f)\subset U}\OO(D(f))$
	\end{claim*}
\begin{subproof}
	We have that $\OO(U)=\clim_{D(f)\subset U}\OO(D(f))=\clim_{D(f)\subset U}A_f$. As $A$ is a domain, each restriction map $A_f\to A_g$ where $D(g)\subset D(f)$ is an injection. Thus, for any $V,W\subset U$ basic open with $V\cap W\neq \emptyset$, we must have for any $s\in \OO(U)$ that $\restr{s}{V}=\restr{s}{W}$ and hence $s\in \OO(V)\cap \OO(W)$ viewed as a subset of $\mathrm{frac}(A)$. We recall that $\spec A$ is connected\footnote{Brief proof of this fact: suppose there exist $U:=D(I_1),W:=D(I_2)$ open such that $U\cap W=\emptyset$ and $U\cup W=X$. Then, $V(I_1)\cup V(I_2)=V(I_1I_2)=X$, so $I_1I_2\subset \bigcap_{\pfr\triangleleft A}\pfr$. But as $A$ is a domain, this last ideal is the zero ideal! Hence, one of $I_1$ or $I_2$ is the zero ideal, so either $U=\emptyset$ or $V=\emptyset$.}. Considering the open cover of $U$ by all basic open sets it contains, by taking a \say{walk} from $D(f)$ to $D(g)$ by a sequence $D(f)=D(f_0),D(f_1),\hdots,D(f_n)$ where $D(f_i)\cap D(f_{i+1})\neq \emptyset$, we have for any $D(f),D(g)\subset U$, any $s\in \OO(U)$ has $s\in D(f)\cap D(g)$. Thus, $\OO(U)\subset \bigcap_{D(f)\subset U}\OO(D(f))$. To show the reverse containment, we note that $\bigcap_{D(f)\subset U}\OO(D(f))$ has obvious inclusion maps commuting with restriction to each $D(f)\subset U$, thus making it a cone. As the largest possible cone, we have that it must indeed be a universal cone and thus $\OO(U)\cong\bigcap_{D(f)\subset U}\OO(D(f))$ canonically.
\end{subproof}
These two claims together combine to prove the proposition.
\end{proof}
\prob{1}\begin{proposition*}
	For $X=\spec \ZZ$, $\OO_X(U)=\ZZ[\{p_1,p_2,\hdots,p^n\}^{-1}]$ where $U=D(p_1p_2\hdots p_n)$\footnote{that all open sets are of this form is a rephrasing of the hint}, with restriction maps all inclusions.
\end{proposition*}
\begin{proof}
	We have that $\OO_X(U)=\bigcap_{x_p\in U} \ZZ_p$\footnote{That is, $\ZZ$ localized at the ideal $p$, not the $p$-adic integers--we shall not make reference to the completion of $\ZZ_p$ in this problem set.} by the result of the previous problem. Viewed as a subset of $\QQ$, this is $\{\frac{a}{b}\::\:\gcd(a,b)=1;\;b\notin p\ZZ\text{ for any} x_p\in U\}$, which can be rephrased $\ZZ[p_1p_2\hdots p_n^{-1}]$. Then, as each ring $\OO_X(U)$ is an integral domain, all restriction maps are injective and hence inclusions.
\end{proof}

\prob{4}\begin{corollary*}
	Any ring $A$ is canonically isomorphic to the projective limit of all its localizations $A_f$ for $f\in A$.
\end{corollary*}
\begin{proof}
	Follows immediately from the fact that for \emph{any\footnote{as opposed to just those which are not basic}} open set $U\subset X=\spec A$, $\OO_X(U)=\clim_{D(f)\subset U} \OO_X(D(f))=\clim_{D(f)\subset U}A_f$ by letting $U=X$. 
\end{proof}
\prob{5}
\begin{proposition*}
	$D(2,t)\subset \AA_\ZZ^1$ is not affine.
\end{proposition*}
\begin{proof}
We note that as $\langle 2, t\rangle $ is a maximum ideal, $U:=D(2,t)=\AA_\ZZ^1\setminus \{x_{\langle 2,t\rangle}\}$. By the result of problem 3, $$\OO(U)=\bigcup_{x\in U}\ZZ[t]_{\pfr_x}=\bigcup_{\langle 2,t\rangle \neq \pfr\triangleleft \ZZ[t]}\ZZ[t]_{\pfr}$$. As $\ZZ[t]$ is a(n?)\footnote{Is UFD pronounced \say{unique factorization domain} or \say{yoo-eff-dee?}} UFD, we may rewrite this as $$\OO(U)=\left\{\frac{f(t)}{g(t)}\in \ZZ(t)\;\:\;g(t)=1 \text{ or }\langle 2,t\rangle \text{ is the \emph{only} ideal } \pfr \text{ such that }g(t)\in \pfr\right\}$$. However, as $\ZZ[t]$ is a UFD, all elements are contained in a \emph{principal} prime ideal, which $\langle 2,t\rangle $ is not. Thus, $g(t)$ may be assumed to be $1$, so $\OO(U)=\ZZ[t]$. We suppose for the sake of contradiction that $U$ is affine. Then, $U\cong \spec\OO(U)$, so $V$ induces a bijection between radical ideals of $\OO(U)$ and varieties in $\spec \OO(U)$. However, $\langle 2,t\rangle  $ is a radical ideal in $\OO(U)$, but $V(2,t)=\emptyset$. This completes our proof.
	\end{proof}
\prob{6}

\begin{proposition*}
	We let $\OO(U)$ be the algebra of holomorphic functions on $U\subset \CC$ open.
	\begin{enumerate}[label=\emph{\roman*)}]
	\item  This presheaf is indeed a sheaf on $\CC$
	\item The stalk $\OO_0$ may be identified with $\left\{f(z)=\sum_{n=0}^\infty a_nz^n \;:\;f(z)\text{ converges for some }z\in \CC\setminus \{0 \}\right\}$
	
	\end{enumerate}
\end{proposition*}
\begin{proof}
	\begin{enumerate}[label=\emph{\roman*)}]
\item Locality follows trivially; indeed if a holomorphic function is zero everywhere in any open set, it is a basic theorem of complex analysis that it is zero everywhere. Gluing follows nearly as trivially. We let $(u_i)_{i\in I}$ be an open cover of $\CC$ and $(f_i)_{i\in I}$ a compatible $I$-touple of functions. Then, $$F(x)=\begin{cases}f_i(x)&x\in u_i
\end{cases}$$ is well-defined as $f_i(x)=f_j(x)$ for any $x$ in any $u_i\cap u_j$. We claim $F$ is holomorphic. For any $x$ in $\CC$, we let $N_x$ be a neighborhood of $x$ such that $N_x\subset u_i$ for some $u_i$ and have that as $F(x)=f_i(x)$ has all its complex derivatives in that neighborhood, it is holomorphic at $x$. As $x$ was arbitrary, this shows that $F\in \OO(\CC)$.
\item We claim that $\OO_0=\{f(z)=\sum_{n=0}^\infty a_nz^n\::\:f(z)\text{ convergent in some open }U\ni 0\}$. Indeed, as any holomorphic function on an open set is equal to its Taylor series centered at any point in that set, we may identify each element of each $\OO_X(U)$ with its Maclaurin series. Then, as the construction from problem 1 generalizes immediately to rings or algebras, we identify $f\in \OO_X(U)$ with $g\in \OO_X(V)$ if $f=g$ on $U\cap V$, which occurs if and only if their Maclaurin series coincide. This proves our first claim. We then note that all functions of $\OO_0$ converge \emph{somewhere} and if a Maclaurin series converges at $z_0\in \CC$, it then converges in the open ball centered at zero with radius $\norm{z_0}$. This identifies the two interpretations of $\OO_0$ and proves the proposition.
\end{enumerate}

\end{proof}
\prob{7}\begin{proposition*}
	We let $\FF$ be a presheaf on topological space $X$, and $\FF^+$ its sheafification with natural sheafification map $\FF\to \FF^+$. Then, there is a canonical isomorphism between stalks $\FF_p\to \FF^+_p$.
\end{proposition*}
\begin{proof}
	We recall that for $U\subset X$ open, $$\FF^+(U)=\{(s_x\in \FF_x)\;:\; s_x\in \FF_x;\;\forall \;x,\;\exists x\in V\subset U \text{ s.t. } \exists s\in \FF(V)\text{ s.t. } s_y=\restr{s}{y}\;\forall y\in V\}.$$ There is naturally a unique set\footnote{I'm ignoring some set-theoretic issues here; replacement with `class' does not effect my argument.} of morphisms $\rho_{U,x}:\FF(U)\to \FF^+_x$ for all $U\subset X$ open, $x\in U$ commuting with the restriction morphisms, that is $\FF(U)\to\FF^+(U)\to \FF^+_p$, the composition of the sheafification map and the restriction to stalk map. Thus, $\FF^+_x$ is a co-cone for $\FF$, so there is a unique morphism $\phi_x:\FF_x\to \FF^+_x$ commuting with the restriction maps for $\FF$ (and as a byproduct, the sheafification maps). On the other hand, there is an obvious set of maps $\sigma_{U,x:}\FF^+(U)\to \FF_x$ for any $x\in U\subset X$ open, that is $(s_y\in \FF_y)_{y\in U}\mapsto s_x\in \FF_x$. As for any $V\subset U$, the restriction $\FF^+(U)\to \FF^+(V)$ is the \say{tautological restriction map} $(s_y\in \FF_y)_{y\in U}\mapsto (s_y\in \FF_y)_{y\in V}$, it is clear that $\sigma_{U,x}$ forms a co-cone. Hence, there is a unique map $\psi_x:\FF^+_x\to \FF_x$ commuting with $\sigma_{U,x}$. Then, $\phi_x\circ \psi_x:\FF_x^+\to \FF_x^+$ gives a morphism commuting with the restriction maps of $\FF^+$ and hence must be the identity by the universal mapping property of the colimit, with a mirrored statement holding for $\psi_x\circ \phi_x:\FF_x\to \FF_x$. Thus, as $\phi_x$ and $\psi_x$ were unique and are inverse isomorphisms, our proof is complete.
\end{proof}
\prob{8}\prt{1} \begin{proposition*}We consider a morphism of sheaves (with concrete target category\footnote{Following Heartshorne's lead\textellipsis}) $\phi:\FF\to \GG$ with maps $\phi_U:\FF(U)\to \GG(U)$ where both are sheaves over $X$. For any $p\in X$:
	\begin{enumerate}[label=\emph{\roman*)}]
	\item $(\ker\phi)_p=\ker(\phi_p)$
	\item $(\im\phi)_p=\im(\phi_p)$
	\end{enumerate}
\begin{proof}Proof by awful element-chasing:
	\begin{enumerate}[label=\emph{\roman*)}]
		\item We recall that in a concrete category, $\FF_p=\bigsqcup_{U\ni p} \FF(U)/\sim$, where $\sim$ is a relation (quotient by ideal, normal subgroup, etc.) allowing commutation. Then, the induced map $\phi_p: [x]\mapsto [\phi_U(x)]$ for some $U\ni x$ is well-defined. We suppose $[x]\in \ker(\phi_p)$. Then, $\phi_U(x)=0\in \GG(U)$ for some $U\ni p$ and indeed any $p\in V\subset U$, so $x$ has a representative in $(\ker\phi)(U)$ for any such $V$. Then, $[x]\in (\ker\phi)_p$, so $(\ker\phi)_p\supseteq\ker(\phi_p)$ On the other hand, we suppose $[x]\in (\ker\phi)_p$. Then, there exists some $U\ni p$ such that $\phi_U(x)=0$, so $[\phi_U(x)]=\phi_p([x])=[0]$. Thus, $(\ker\phi)_p=\ker(\phi_p)$.
\item We suppose $[x]\in (\im\phi)_p$. Then, there is some representative $x\in \GG(U)$ where $p\in U$ such that $x=\phi_U(y)$ for some $y\in \FF(U)$, so $\phi_p([y])=[x]$, and $[x]\in \im\phi_p$, so $(\im\phi)_p\subseteq \im\phi_p$. On the other hand, we let $[x]\in \im\phi_p$ Then, there is some $[y]\in \FF_p$ such that $\phi_p([y])=[x]$, so there is some $U\ni p$ such that $[y]$ has a representative $y\in \FF(U)$ where $\phi_U(y)=x\in[x]$. Thus, $x\in (\im\phi)(U)$ implying $[x]\in(\im\phi)_p$ so $\im\phi_p= \im\phi_p$.
	\end{enumerate}\end{proof}\end{proposition*}
\prt{2}\begin{corollary*}
	$\phi$ is injective (resp. surjective) if and only if $\phi_p$ is injective (resp. surjective)
\end{corollary*}\begin{proof}
We claim that for a sheaf $\mathscr{A}$, $A=0$ if and only if $A_p=0$ for all $p\in X$. Indeed, the locality axiom ensures this is true. Then, $\phi$ is injective if and only if $\ker\phi$ is the $0$ sheaf, if and only if $(\ker\phi)_p=0$, if and only if $\ker\phi_p=0$ for all $p$ by the previous result.  The same follows for surjectivity by simply replacing $\ker$ with $\mathrm{coker}$.\end{proof}

\prt{3}\begin{corollary*}
	Let $\Fcal:=\hdots\displaystyle{\mathrel{\mathop{\leftarrow}^{\mathrm{\phi^{i-1}}}}}\FF_{i-1}\displaystyle{\mathrel{\mathop{\leftarrow}^{\mathrm{\phi^i}}}}\FF_i\displaystyle{\mathrel{\mathop{\leftarrow}^{\mathrm{\phi^{i+1}}}}}\hdots$ be a sequence of morphisms and sheaves. Then, $\Fcal$ is exact if and only if the induced sequence $\Fcal_p$ is exact for all $p\in X$.
\end{corollary*}
\begin{proof}
	We suppose $\Fcal$ is exact. Then, $\ker\phi^i=\im\phi^{i+1}$ for all $i$, so by part a, we have $\ker(\phi^i_p)=\im(\phi^{i+1}_p)$ for all $p$. Thus, for any $p\in X$, $\Fcal_p$ is exact. On the other hand, as sheaves may be recovered uniquely from stalks, if the sequence $\Fcal_p$ is exact at every point $p$, we may reconstruct the sheaves $\ker(\phi^i)$ and $\im(\phi^{i+1})$ from their stalks and have by exactness at stalks that $\Fcal$ is exact once again from part a.
\end{proof}
\end{document}