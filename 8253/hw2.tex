\documentclass[english]{article}
\newcommand{\G}{\overline{C_{2k-1}}}
\usepackage[latin9]{inputenc}
\usepackage{amsmath}
\usepackage{amssymb}
\usepackage{lmodern}
\usepackage{mathtools}
\usepackage{enumitem}
\usepackage{pgf}
\usepackage{tikz}
\usepackage{tikz-cd}
\usetikzlibrary{arrows, matrix}
%\usepackage{natbib}
%\bibliographystyle{plainnat}
%\setcitestyle{authoryear,open={(},close={)}}
\let\avec=\vec
\renewcommand\vec{\mathbf}
\renewcommand{\d}[1]{\ensuremath{\operatorname{d}\!{#1}}}
\newcommand{\pydx}[2]{\frac{\partial #1}{\partial #2}}
\newcommand{\dydx}[2]{\frac{\d #1}{\d #2}}
\newcommand{\ddx}[1]{\frac{\d{}}{\d{#1}}}
\newcommand{\hk}{\hat{K}}
\newcommand{\hl}{\hat{\lambda}}
\newcommand{\ol}{\overline{\lambda}}
\newcommand{\om}{\overline{\mu}}
\newcommand{\all}{\text{all }}
\newcommand{\valph}{\vec{\alpha}}
\newcommand{\vbet}{\vec{\beta}}
\newcommand{\vT}{\vec{T}}
\newcommand{\vN}{\vec{N}}
\newcommand{\vB}{\vec{B}}
\newcommand{\vX}{\vec{X}}
\newcommand{\vx}{\vec {x}}
\newcommand{\vn}{\vec{n}}
\newcommand{\vxs}{\vec {x}^*}
\newcommand{\vV}{\vec{V}}
\newcommand{\vTa}{\vec{T}_\alpha}
\newcommand{\vNa}{\vec{N}_\alpha}
\newcommand{\vBa}{\vec{B}_\alpha}
\newcommand{\vTb}{\vec{T}_\beta}
\newcommand{\vNb}{\vec{N}_\beta}
\newcommand{\vBb}{\vec{B}_\beta}
\newcommand{\bvT}{\bar{\vT}}
\newcommand{\ka}{\kappa_\alpha}
\newcommand{\ta}{\tau_\alpha}
\newcommand{\kb}{\kappa_\beta}
\newcommand{\tb}{\tau_\beta}
\newcommand{\hth}{\hat{\theta}}
\newcommand{\evat}[3]{\left. #1\right|_{#2}^{#3}}
\newcommand{\prompt}[1]{\begin{prompt*}
		#1
\end{prompt*}}
\newcommand{\vy}{\vec{y}}
\DeclareMathOperator{\sech}{sech}
\DeclareMathOperator{\Spec}{Spec}
\DeclareMathOperator{\spec}{Spec}
\DeclareMathOperator{\spm}{Spm}
\DeclareMathOperator{\rad}{rad}
\newcommand{\mor}{\mathrm{Mor}}
\newcommand{\obj}{\mathrm{Obj}~}
\DeclarePairedDelimiter\abs{\lvert}{\rvert}%
\DeclarePairedDelimiter\norm{\lVert}{\rVert}%
\newcommand{\dis}[1]{\begin{align}
	#1
	\end{align}}
\newcommand{\Aa}{\mathbb{A}}
\newcommand{\LL}{\mathcal{L}}
\newcommand{\CC}{\mathbb{C}}
\newcommand{\DD}{\mathbb{D}}
\newcommand{\RR}{\mathbb{R}}
\newcommand{\NN}{\mathbb{N}}
\newcommand{\ZZ}{\mathbb{Z}}
\newcommand{\QQ}{\mathbb{Q}}
\newcommand{\Ss}{\mathcal{S}}
\newcommand{\OO}{\mathcal{O}}
\newcommand{\BB}{\mathcal{B}}
\newcommand{\Pcal}{\mathcal{P}}
\newcommand{\afr}{\mathfrak{a}}
\newcommand{\bfr}{\mathfrak{b}}
\newcommand{\cfr}{\mathfrak{c}}
\newcommand{\dfr}{\mathfrak{d}}
\newcommand{\efr}{\mathfrak{e}}
\newcommand{\ffr}{\mathfrak{f}}
\newcommand{\gfr}{\mathfrak{g}}
\newcommand{\hfr}{\mathfrak{h}}
\newcommand{\ifr}{\mathfrak{i}}
\newcommand{\jfr}{\mathfrak{j}}
\newcommand{\kfr}{\mathfrak{k}}
\newcommand{\lfr}{\mathfrak{l}}
\newcommand{\mfr}{\mathfrak{m}}
\newcommand{\nfr}{\mathfrak{n}}
\newcommand{\ofr}{\mathfrak{o}}
\newcommand{\pfr}{\mathfrak{p}}
\newcommand{\qfr}{\mathfrak{q}}
\newcommand{\rfr}{\mathfrak{r}}
\newcommand{\sfr}{\mathfrak{s}}
\newcommand{\tfr}{\mathfrak{t}}
\newcommand{\ufr}{\mathfrak{u}}
\newcommand{\vfr}{\mathfrak{v}}
\newcommand{\wfr}{\mathfrak{w}}
\newcommand{\xfr}{\mathfrak{x}}
\newcommand{\yfr}{\mathfrak{y}}
\newcommand{\zfr}{\mathfrak{z}}
\newcommand{\Dcal}{\mathcal{D}}
\newcommand{\Ccal}{\mathcal{C}}
\usepackage{graphicx}
% Swap the definition of \abs* and \norm*, so that \abs
% and \norm resizes the size of the brackets, and the 
% starred version does not.
%\makeatletter
%\let\oldabs\abs
%\def\abs{\@ifstar{\oldabs}{\oldabs*}}
%
%\let\oldnorm\norm
%\def\norm{\@ifstar{\oldnorm}{\oldnorm*}}
%\makeatother
\newenvironment{subproof}[1][\proofname]{%
	\renewcommand{\qedsymbol}{$\blacksquare$}%
	\begin{proof}[#1]%
	}{%
	\end{proof}%
}

\usepackage{centernot}
\usepackage{dirtytalk}
\usepackage{calc}
\newcommand{\prob}[1]{\setcounter{section}{#1-1}\section{}}


\newcommand{\prt}[1]{\setcounter{subsection}{#1-1}\subsection{}}
\newcommand{\pprt}[1]{{\textit{{#1}.)}}\newline}
\renewcommand\thesubsection{\alph{subsection}}
\usepackage[sl,bf,compact]{titlesec}
\titlelabel{\thetitle.)\quad}
\DeclarePairedDelimiter\floor{\lfloor}{\rfloor}
\makeatletter

\newcommand*\pFqskip{8mu}
\catcode`,\active
\newcommand*\pFq{\begingroup
	\catcode`\,\active
	\def ,{\mskip\pFqskip\relax}%
	\dopFq
}
\catcode`\,12
\def\dopFq#1#2#3#4#5{%
	{}_{#1}F_{#2}\biggl(\genfrac..{0pt}{}{#3}{#4}|#5\biggr
	)%
	\endgroup
}
\def\res{\mathop{Res}\limits}
% Symbols \wedge and \vee from mathabx
% \DeclareFontFamily{U}{matha}{\hyphenchar\font45}
% \DeclareFontShape{U}{matha}{m}{n}{
%       <5> <6> <7> <8> <9> <10> gen * matha
%       <10.95> matha10 <12> <14.4> <17.28> <20.74> <24.88> matha12
%       }{}
% \DeclareSymbolFont{matha}{U}{matha}{m}{n}
% \DeclareMathSymbol{\wedge}         {2}{matha}{"5E}
% \DeclareMathSymbol{\vee}           {2}{matha}{"5F}
% \makeatother

%\titlelabel{(\thesubsection)}
%\titlelabel{(\thesubsection)\quad}
\usepackage{listings}
\lstloadlanguages{[5.2]Mathematica}
\usepackage{babel}
\newcommand{\ffac}[2]{{(#1)}^{\underline{#2}}}
\usepackage{color}
\usepackage{amsthm}
\newtheorem{theorem}{Theorem}[section]
\newtheorem*{thm*}{Theorem}
\newtheorem{conj}[theorem]{Conjecture}
\newtheorem{corollary}[theorem]{Corollary}
\newtheorem{example}[theorem]{Example}
\newtheorem{lemma}[theorem]{Lemma}
\newtheorem*{lemma*}{Lemma}
\newtheorem{problem}[theorem]{Problem}
\newtheorem{proposition}[theorem]{Proposition}
\newtheorem*{proposition*}{Proposition}
\newtheorem*{corollary*}{Corollary}
\newtheorem{fact}[theorem]{Fact}
\newtheorem*{prompt*}{Prompt}
\newtheorem*{claim*}{Claim}
\newcommand{\claim}[1]{\begin{claim*} #1\end{claim*}}
%organizing theorem environments by style--by the way, should we really have definitions (and notations I guess) in proposition style? it makes SO much of our text italicized, which is weird.
\theoremstyle{remark}
\newtheorem{remark}{Remark}[section]

\theoremstyle{definition}
\newtheorem{definition}[theorem]{Definition}
\newtheorem*{definition*}{Definition}
\newtheorem{notation}[theorem]{Notation}
\newtheorem*{notation*}{Notation}
%FINAL
\newcommand{\due}{27 September 2017} 
\RequirePackage{geometry}
\geometry{margin=.7in}
\usepackage{todonotes}
\title{MATH 8253 Homework I}
\author{David DeMark}
\date{\due}
\usepackage{fancyhdr}
\pagestyle{fancy}
\fancyhf{}
\rhead{David DeMark}
\chead{\due}
\lhead{MATH 8253}
\cfoot{\thepage}
% %%
%%
%%
%DRAFT

%\usepackage[left=1cm,right=4.5cm,top=2cm,bottom=1.5cm,marginparwidth=4cm]{geometry}
%\usepackage{todonotes}
% \title{MATH 8669 Homework 4-DRAFT}
% \usepackage{fancyhdr}
% \pagestyle{fancy}
% \fancyhf{}
% \rhead{David DeMark}
% \lhead{MATH 8669-Homework 4-DRAFT}
% \cfoot{\thepage}

%PROBLEM SPEFICIC

\newcommand{\lint}{\underline{\int}}
\newcommand{\uint}{\overline{\int}}
\newcommand{\hfi}{\hat{f}^{-1}}
\newcommand{\tfi}{\tilde{f}^{-1}}
\newcommand{\tsi}{\tilde{f}^{-1}}
\newcommand{\PP}{\mathcal{P}}
\newcommand{\nin}{\centernot\in}
\newcommand{\seq}[1]{({#1}_n)_{n\geq 1}}
\newcommand{\Tt}{\mathcal{T}}
\newcommand{\card}{\mathrm{card}}
\newcommand{\setc}[2]{\{ #1\::\:#2 \}}
\newcommand{\idl}[1]{\langle #1 \rangle}
\newcommand{\cl}{\overline}
\newcommand{\id}{\mathrm{id}}
\newcommand{\im}{\mathrm{Im}}
\newcommand{\cat}[1]{{\mathrm{\bf{#1}}}}
%\usepackage[backend=biber,style=alphabetic]{biblatex}
%\addbibresource{algeo.bib}

\begin{document}
\maketitle\prob{1} \begin{proposition*}
	Let $A$ be a ring, $S\subset A$ a multiplicative system, and $\tau:A\to A_S$ the localization homomorphism. Then, the induced map $$\tau^*:\spec A_S\to \bigcap_{f\in S} D(f)\subset\spec A$$ is a homeomorphism.
\end{proposition*}
\begin{proof}
	\begin{itemize}
		\item \emph{Well-Defined} It is well-known that for any ring homomorphism, the pre-image of a prime ideal is a prime ideal. We let $\pfr\triangleleft A_S$. We note that $\tau^{-1}(\pfr)\neq \emptyset$ as $(\im \tau)^c\subset A^\times$. Then, $\tau^{-1}(\pfr)\cap S\neq \emptyset$, as else $\pfr$ contains a unit. Hence, $\tau^*(x_\pfr)=x_{\tau^{-1}(\pfr)}\in \bigcap_{f\in S}D(f)$.
		\item \emph{Injectivity} From our previous observation that $(\im \tau)^c\subset A_S^\times$, we have that for any proper ideal $I\triangleleft A_S$, $I\subset \im \tau$. Thus, for any two prime ideals $\pfr_1,\pfr_2$, we have that $\tau^{-1}(\pfr_1\Delta \pfr_2)\neq\emptyset$ and hence $\tau^{-1}(\pfr_1)$ and $\tau^{-1}(\pfr_2)$ are distinct. It follows $\tau^*$ is injective. 
		\item \emph{Surjectivity} We let $x_\qfr\in \bigcap_{f\in S}D(f)$ and let $\qfr=\langle x_1,\hdots\rangle $. We claim that $\tau^{-1}\langle\tau \qfr\rangle=\qfr$. That $\qfr\subset \tau^{-1}\langle\tau \qfr\rangle$ is obvious. We suppose $r\in \tau^{-1}\tau \qfr$. Then, $\frac{r}{1}\in \langle\tau \qfr\rangle=\qfr[S^{-1}]$, so $\frac{r}{1}=\frac{q}{s}$ for some $q\in \qfr$, $s\in S$. Then, $s'(rs-q)=0$ for some $s'\in S$. However, we must have that $(rs-q)\in \qfr$ as $s\notin \qfr$, which then implies $rs\in \qfr$ which in turn implies by primality $r\in \qfr$ as $s\notin \qfr$. Thus, $\tau^{-1}\langle \tau \qfr\rangle=\qfr$, so $\tau^{*}x_{\langle \tau \qfr\rangle}=x_\qfr$ and $\tau^{*}$ is surjective.
		\item \emph{Bicontinuity} We let $E\subset A$. Then, $(\tau^*)^{-1}D(E)=D_S(\tau E)$, which is open, so $\tau^*$ is continuous. 
		
		To show continuity of $(\tau^*)^{-1}$, we let $E\subset A_S$ and claim $\tau^*(V_S(E))=V(\tau^{-1}(E))$. We note that if $E\subset \pfr\triangleleft A_S$ then certainly $\tau^{-1}(E)\subset \tau^{-1}(\pfr)$, so $\tau^*V_S(E)\subset V(\tau^{-1}(E))$. On the other hand, for $\tau^{-1}(E)\subset\qfr\triangleleft A$, we have that $E\subset \tau\tau^{-1}(E)\subset \langle\tau(\qfr)\rangle=(\tau^*)^{-1}(\qfr)$, so $V(\tau^{-1}(E))\subset \tau^*(V_S(E))$ and our proof is complete.
		%On the other hand, letting $E\subset A_S$, $\tau^*(D_S(E))=\tau^*(\{x_{\pfr}:E\not \subset \pfr\})=\{x_{\tau^{-1}\pfr}:E\not \subset \pfr\}$. We note that if $E\not \subset\pfr$, then there exists some $\frac{e}{s}\in A_S\setminus \pfr$ for $e\in A$, then implying that $\frac{e}{1}\not \in \pfr$ and thus $e\in\tau^{-1}(\frac{e}{1})$
	\end{itemize}
	
	\end{proof}

\prob{2}
\begin{proposition*}
	Let $X$ be an irreducible topological space such that there is some $x_0\in X$ with $\cl{\{x_0\}}=X$. Then, for $\phi:X\to Y$ continuous, \emph{(i)}, $\cl{\mathrm{Im}(\phi)}$ is irreduicble, and \emph{(ii)} $\cl{\{\phi(x)\}}=\cl{\mathrm{Im}(\phi)}$ 
\end{proposition*}
\begin{subproof}
	We begin with a quick lemma.
	\begin{lemma*}
		With the hypotheses of the proposition, for any $u\subset \cl{\im(\phi)}$ open, $\phi^{-1}(u)= \emptyset$ if and only if $u=\emptyset$.
	\end{lemma*}
\begin{subproof} One direction is obvious, we now show the other.
We suppose $\phi^{-1}(u)=\emptyset$. Then, $u\cap \im(\phi)=\emptyset$, so $\cl{\im(\phi)}\setminus u$ is a closed subset of $Y$ containing the image of $\phi$ and hence its closure, that is $\cl{\im(\phi)}\subseteq \cl{\im(\phi)}\setminus u$. Thus, $u=\emptyset$.
\end{subproof}
Now, onward to the proposition. \begin{enumerate}[label=\emph{(\roman*)}]
\item We let $\emptyset\neq u,v\subset \cl{\im(\phi)}$ be open. Then, $\phi^{-1}(u),\phi^{-1}(v)$ are nonempty open sets in $X$ by the lemma, and hence $\emptyset\neq \phi^{-1}(u)\cap \phi^{-1}(v)=\phi^{-1}(u\cap v)$. Thus, for any two non-empty open sets in $\cl{\im(\phi)}$, their intersection is non-empty and hence $\cl{\im(\phi)}$ is irreducible.
\item We let $\phi(x)\notin v\subset \cl{\im(\phi)}$ where $v$ is open. Then, $x\notin \phi^{-1}(v)$, so $X\setminus \phi^{-1}(v)$ is a closed subset of $X$ containing $x$ and is hence the full set $X$. Thus, $\phi^{-1}(v)$ is empty, so by the lemma $v=\emptyset$. Hence, $\cl{\{\phi(x)\}}=\cl{\im(\phi)}$
\end{enumerate}
\end{subproof}\begin{corollary*}
Let $\phi:A\to A'$ be a ring homomorphism and $\phi^*:\spec A'\to \spec A$ the corresponding map of spectra. If $\spec A'$
is irreducible with $\xi$ its generic point, then the closure $\cl{\phi^*(Spec A')}$ is irreducible and
 $\phi^*(\xi)$ is a generic point of it.
\end{corollary*}
\begin{proof}
	Follows from the proposition, replacing $\spec A'$ with $X$ and $\xi$ with $x$.

\end{proof}
\prob{3}
\begin{proposition*}
	For $R$ a principal ideal domain with natural injection $\iota:R\to R[t]$ and induced map $\pi:\spec R[t]\to \spec R$, $\pi^{-1}(s)\approx \mathbb{A}_{k(s)}^1$.
\end{proposition*}
\begin{proof}
	We let $s\in \spec{R}$ correspond to ideal $\pfr_s\triangleleft R$. Then, $K(s)=\left(R/\pfr_s\right)_{\pfr_s}=R/\pfr_s$ as in a PID, primality and maximality are equivalent. Thus, $\Aa_{k(s)}^1=\spec{k(s)[t]}=\spec{(R/\pfr_s)[t]}=\spec{(R[t]/\pfr_s[t])}\approx V(\pfr_s[t])\subset \Aa_{R}^1$. Thus, we need only show $V(\pfr_s[t])=\pi^{-1}(s)$. We note for $x\in \Aa_{R}^1$, $\pi(x)=s$ if and only if $\pfr_s[t]\subset \qfr_x\subset R[t]$, as $\pfr_s$ is maximal in $R$. This is precisely the set $V(\pfr_s[t])$. This completes our proof 
\end{proof}
\prob{4}
We let $A$ be a ring
\prt{1}
\begin{proposition*}
	For $f\in A$, we let $S(f)=\{g\in A\;:\;D(f)\subset D(g)\}$. Then $S(f)$ is a multiplicative system containing $\{f,f^2,f^3,\hdots\}$.
\end{proposition*}
\begin{proof}
	We let $g,h\in S(f)$. Then, $D(gh)=D(g)\cap D(h)\supseteq D(f)$, so $gh\in S(f)$. We also note that as $A\to k(s)$ is a ring homomorphism, for $f\neq 0$ in $k(s)$, we have $f^n\neq 0$ as otherwise $k(s)$ would contain zero-devisors. Thus, for any $s\in D(f)$, $s\in D(f^n)$ for all $n$. 
\end{proof}
\prt{2}
\begin{proposition*}
The localization map $\sigma: A_f\to A_{S(f)}$ is an isomorphism.
\end{proposition*}
\begin{proof}
 We shall state and prove a lemma then show its equivalence to the proposition.
 \begin{lemma*}
 	For $D(f)\subset D(g)$, and $\rho_f:A\to A_f$ the localization map, $\rho_f(g)$ is a unit in $A_f$.
 \end{lemma*}
\begin{subproof}
We rewrite our hypothesis as $V(f)\supset V(g)$ and have by the Nullstellensatz that $\rad(\langle f \rangle)=I(V(F))\subset I(V(g))=\rad(\idl{g})$. In particular, we have that $f\in \rad(\idl{g})$, so for some $n$, $f^n=gr$ with $r\in A$. Applying $\rho_f$ to that equality yields $\frac{f^n}{1}=\frac{gr}{1}$, so dividing through by $f^n$ gives the relation $\rho_f(g)\frac{r}{f^n}=1$.
\end{subproof}
We have the below commutative diagram

	\[ 
	\begin{tikzcd}
	&A \arrow{ld}{\rho_f} \arrow{d}{\rho_{S(f)}} &  \\
	A_{f} \arrow{r}{\sigma} & A_{S(f)} \\
	\end{tikzcd}
	\]
We note that as $f$ is mapped by $\rho_{S(f)}$ to the unit group of $A_{S(f)}$, the localization universal mapping property implies that $\sigma$ is the unique map for which this diagram commutes. On the other hand, as $S(f)$ is mapped to the unit group of $A_f$ by $\rho_f$, we have that there is a unique map $\psi$ making the below diagram commutative.
	\[ 
\begin{tikzcd}
&A \arrow{ld}{\rho_f} \arrow{d}{\rho_{S(f)}} \arrow{rd}{\rho_f} &  \\
A_{f} \arrow{r}{\sigma} & A_{S(f)} \arrow[dashrightarrow]{r}{\psi} &A_f \\
\end{tikzcd}
\]
However, as the identity is the unique map from $A_f$ to $A_f$ commuting with $\rho_f$, we must have that $\psi\sigma=\id_{A_f}$. This argument can be restated symmetrically, thus showing $\psi$ and $\sigma$ to be two-sided mutual inverses and hence isomorphisms.
\end{proof}
\prt{3}\begin{prompt*}
	Define a contravariant functor $\OO_X:\DD(X)\to \cat{Ring}$ by $$D(f)\mapsto A_{S(f)}$$
	$$D(f)\subset D(g)\mapsto( A_{S(g)}\to A_{S(f)})$$
\end{prompt*}
\begin{proof}[Response]
	We show functoriality. We first note that $D(f)\subset D(g)$ implies that for any $h\in S(g)$, $D(f)\subset D(g)\subset D(h)$, so $h\in S(f)$. Thus, $S(g)\subset S(f)$, so the morphism $A_{S(g)}\to A_{S(f)}$ is achieved simply by localizing $A_{S(g)}$ at $S(f)$. We note that $A_{S(f)}\to A_{S(f)}$ is indeed the identity morphism in $\cat{Ring}$ and thus $\OO_X$ preserves identity. Finally, we suppose $D(f)\subset D(g)\subset D(h)$. Then, $A_{S(h)}\to A_{S(g)}\to A_{S(f)}$ is the composite of the localization of one set and a superset to it and hence the same map as $A_{S(h)}\to A_{S(f)}$ (this also follows immediately from a universal mapping property argument). Hence $\OO_X$ is indeed a contravariant functor.
\end{proof}
\prt{4}
\begin{proposition*}
$\OO^\#_X:\DD^\#(X)\to \cat{Ring}$ is naturally equivalent to the composition $\OO_X\circ F:\DD^\#(X)\to\DD(X)\to \cat{Ring}$
\end{proposition*}\begin{proof}
We define our natural transformation by $n_{\bullet}\in \mor_{\DD^\#(X)}(\bullet,-)$ and $m_{\bullet}\in \mor_{\cat{Ring}}(\bullet,-)$.\footnote{This feels like terrible and possibly not even well-defined notation, but I'm not sure how to express this otherwise?} We let $n_{f}=\id_{f}$ and $m_{A_{f}}=\sigma$ where $\sigma$ is the isomorphism defined in part 4b. Then, we have the following diagram showing the paths of elements:
$$
\begin{tikzcd}
 f\arrow[mapsto]{rr}{\OO^\#(X)}\arrow[mapsto]{d}{n_f}&&A_f\arrow[mapsto]{d}{m_{f}}\\
 f\arrow[mapsto]{r}{F}&D(f)\arrow[mapsto]{r}{\OO_X}&A_{S(f)}
\end{tikzcd}
$$ It is immediately clear this diagram commutes.
As all of $m_{\bullet}$ and $n_{\bullet}$ are isomorphisms, our natural transformation is a natural equivalence.
\end{proof}
\prob{5}
\prt{1}\begin{proposition*}
	Let $A$ be a ring, $X$ its spectrum, $f\in A$, $X_f$ the spectrum of $A_f$, and $\tau^*:X_f\to D(f)\subset X$ the homeomorphism of problem 1. Then $\tau^*$ induces a bijection between basic open sets $D(g)\subset D(f)$ and those $D_f(g)\subset X_f$.
\end{proposition*}
\begin{proof}
We shall show that for $U\subset X_f$ open, $\tau^*(U)$ is basic if and only if $U$ is basic. We first assume $\tau^*(U)=D(g)$ with $g\in A$. Then, as $\tau^*$ is a homeomorphism, $U=(\tau^*)^{-1}(D(g))=D_f(\tau(g))=D_f(\frac{g}{1})$. Conversely, consider $D_f(\frac{h}{f^n})$ where $n\geq 0$. We claim that $D_f(\frac{h}{f^n})=D_f(\frac{h}{1})$. Indeed, as $f^n$ is a unit in $A_f$, if $\frac{h}{f^n}$ maps to a nonzero element in $k(s)$ for any $s\in X_f$, then $f^n\frac{h}{f^n}$ is necessarily nonzero. Thus, $D_f(\frac{h}{f^n})=D_f(\frac{h}{1})=\tau(D(h))$. This completes our proof. 
\end{proof}
\prt{2}
\begin{proposition}
	The restriction of $\OO_X$ to $\DD(D(f))$ for some $f\in A$ is equivalent to $\OO_{X_f}:\DD(X_f)\to \cat{Ring}$.
\end{proposition}
\begin{proof} I'm ultimately a bit confused by this\textellipsis We can only define a natural transformation between functors between the \emph{same} two categories $\Ccal$ and $\Dcal$\textellipsis It seems like a natural interpretation to require a natural equivalence $F:\DD(D(f))\to X_f$ which commutes with $\OO_{X_f}$ and $\OO_X$, that is a natural equivalence $F$ such that the following commutes: $$
	\begin{tikzcd}
	\DD(D(f))\arrow{r}{\OO_X}\arrow{d}{F}&\cat{Ring}\\
	\DD(X_f)\arrow{ru}[swap]{\OO_{X_f}}
	\end{tikzcd}$$
	
	We let $F(D(g))=D_f(\frac{g}{1})=(\tau^*)^{-1}(D(g))$, and $F(D(g)\xhookrightarrow{}D(h))=[(\tau^*)^{-1}(D(g))\xhookrightarrow[]{}(\tau^*)^{-1}(D(h))]$ (As $\tau$ is a homeomorphism, such inclusions are preserved). As $F$ is in fact surjective (even better than essentially surjective!) in light of part a and is in its description explicitly bijective on morphisms, it is indeed a natural equivalence. Further, as $(\OO_{X_f}\circ F)(D(g))=(A_f)_g=A_g=\OO_X(D(g))$ (4b,c), it is clear our diagram commutes on objects (that it commutes on morphisms is more or less trivially checked as each morphism set has cardinality at most one). This completes our \say{proof.} 
\end{proof}
\prob{6}
\begin{prompt*}
	Find two presheaves, one of which \emph{(i)} fails locality but satisfies gluing and the other of which \emph{(ii)} satisfies locality but fails gluing.
\end{prompt*}
\begin{proof}[Response]
	\begin{enumerate}[label=\emph{(\roman*)}]
		\item We let $X$ be some Euclidean\footnote{well, I think any topological space with cardinality greater than 1 \emph{should} work, but let's stick to Euclidean so I can make my counter-example super concrete so there's less to poke holes in} space, say $\CC^n$ for $n\geq 1$ and define the \say{double-power} presheaf $P:\cat{Op}(X)\to \cat{Set}$ as follows: for an open set $U\subset X$, let $P(U)=\Pcal(\Pcal(U))$. For $V\subset U$, $S\in P(U)$, let $\evat{S}{V}{}=\{s\in S\;:\; s\subset V\}$. Then, $\evat{S}{U}{}=\{s\in S\;:\; s\subset U\}=S=\id_{\cat{Set}}(S)$, so the identity restriction axiom of presheaves is satisfied. In addition, for $S\in P(U)$, $W\subset V\subset U$, $\evat{\left(\evat{S}{V}{}\right)}{W}{}=\evat{\{s\in S\;:\;s\subset V\}}{W}{}=\{s\in S\;:\; s\subset V\cap W\}$. As $V\cap W=W$ by assumption, this is the same set as $\{s\in S\;:\; s\subset W\}=\evat{S}{W}{}$. Thus, $P$ is indeed a presheaf. We claim it satisfies the gluing sheaf axiom: we let $\{U_i\}_{i\in I}$ be some open cover of $X$ and $\{S_i\}_{i\in I}$ an element of $\prod_{i\in I}P(U_i)$ satisfying the intersection-compatibility condition of the gluing axiom. We claim that $S=\bigcup_{i\in I}S_i$ satisfies $\evat{S}{U_i}{}=S_i$ for all $i$. Indeed, that $S_i\subset \evat{S}{U_i}{}$ follows trivially from the way in which we defined restriction maps. To show the opposite containment, we suppose there exists some $\emptyset\neq W\in \left(\evat{S}{U_i}{}\setminus S_i\right)$. Then, $W\in S_j$ for some $j$, and $W\subset U_i$ as otherwise it would not be an element of $\evat{S}{U_i}{}$. Thus, $W\in \left(\evat{S_j}{U_i\cap U_j}{}\setminus \evat{S_i}{U_i\cap U_j}{}\right)$, contradicting our assumption of compatibility, and showing $P$ to satisfy the gluing sheaf axiom. However, we claim it does not satisfy the locality axiom. To see this, we let $\{B_i\}_{i\in I}$ be an open covering of $X$ by unit balls, $V$ a ball of radius exceeding unity and $S$ the singleton set $S=\{V\}$. Then, $\evat{V}{B_i}{}=\emptyset$ for all $i$ as clearly $V\centernot\subset B_i$, but $V\neq \emptyset$. This contradicts the locality sheaf axiom so indeed $P$ is not a sheaf.
		\item\emph{ (Note: I found reference to this one in Vakil, but proved its relevant properties myself)} We let $X=\CC$ and let $\BB_X$ be the \say{presheaf of bounded functions,} that is, a functor from $\DD(\CC)$ to $\cat{Ring}$ where $\BB_X(u)$ is the ring of bounded functions on $u\subset \CC$ open and $(u\subset v)\mapsto (f\mapsto \evat{f}{v}{})$, that is simple function restriction. Then, it is clear that $\BB_X$ is a presheaf, and it quite clearly satisfies the locality axiom as any non-zero function on $\CC$ certainly must restrict to a non-zero function on some set of an open cover. However, if we consider the function on $\CC$ $f(x)=x$ and an open cover $\{u_i\}_{i\in I}$ consisting of bounded sets, we have that on each $u_i$, $\evat{f}{u_i}{}$ is bounded, but $f\not \in \BB_X(\CC)$ as it is not bounded on the complex plane. Thus, $\BB_X$ fails to be a sheaf because it fails the gluing axiom.
		\end{enumerate}
\end{proof}
\end{document}