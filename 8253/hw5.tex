\documentclass[english,letter,doublesided]{article}
\usepackage{rotating}
\newcommand{\G}{\overline{C_{2k-1}}}
\usepackage[latin9]{inputenc}
\usepackage{amsmath,calligra,mathrsfs,amsfonts}
\usepackage{amssymb}
\usepackage{lmodern}
\usepackage{mathtools}
\usepackage{enumitem}
\usepackage{pgf}
\usepackage{tikz}
\usepackage{tikz-cd}
\usepackage{relsize}

\usetikzlibrary{arrows, matrix}
%\usepackage{natbib}
%\bibliographystyle{plainnat}
%\setcitestyle{authoryear,open={(},close={)}}
\let\avec=\vec
\renewcommand\vec{\mathbf}
\renewcommand{\d}[1]{\ensuremath{\operatorname{d}\!{#1}}}
\newcommand{\pydx}[2]{\frac{\partial #1}{\partial #2}}
\newcommand{\dydx}[2]{\frac{\d #1}{\d #2}}
\newcommand{\ddx}[1]{\frac{\d{}}{\d{#1}}}
\newcommand{\hk}{\hat{K}}
\newcommand{\hl}{\hat{\lambda}}
\newcommand{\ol}{\overline{\lambda}}
\newcommand{\om}{\overline{\mu}}
\newcommand{\all}{\text{all }}
\newcommand{\valph}{\vec{\alpha}}
\newcommand{\vbet}{\vec{\beta}}
\newcommand{\vT}{\vec{T}}
\newcommand{\vN}{\vec{N}}
\newcommand{\vB}{\vec{B}}
\newcommand{\vX}{\vec{X}}
\newcommand{\vx}{\vec {x}}
\newcommand{\vn}{\vec{n}}
\newcommand{\vxs}{\vec {x}^*}
\newcommand{\vV}{\vec{V}}
\newcommand{\vTa}{\vec{T}_\alpha}
\newcommand{\vNa}{\vec{N}_\alpha}
\newcommand{\vBa}{\vec{B}_\alpha}
\newcommand{\vTb}{\vec{T}_\beta}
\newcommand{\vNb}{\vec{N}_\beta}
\newcommand{\vBb}{\vec{B}_\beta}
\newcommand{\bvT}{\bar{\vT}}
\newcommand{\ka}{\kappa_\alpha}
\newcommand{\ta}{\tau_\alpha}
\newcommand{\kb}{\kappa_\beta}
\newcommand{\tb}{\tau_\beta}
\newcommand{\hth}{\hat{\theta}}
\newcommand{\evat}[3]{\left. #1\right|_{#2}^{#3}}
\newcommand{\prompt}[1]{\begin{prompt*}
		#1
\end{prompt*}}
\newcommand{\vy}{\vec{y}}
\DeclareMathOperator{\sech}{sech}
\DeclareMathOperator{\Spec}{Spec}
\DeclareMathOperator{\spec}{Spec}
\DeclareMathOperator{\spm}{Spm}
\DeclareMathOperator{\rad}{rad}
\newcommand{\mor}{\mathrm{Mor}}
\newcommand{\obj}{\mathrm{Obj}~}
\DeclarePairedDelimiter\abs{\lvert}{\rvert}%
\DeclarePairedDelimiter\norm{\lVert}{\rVert}%
\newcommand{\dis}[1]{\begin{align}
	#1
	\end{align}}
\renewcommand{\AA}{\mathbb{A}}
\newcommand{\LL}{\mathcal{L}}
\newcommand{\CC}{\mathbb{C}}
\newcommand{\DD}{\mathbb{D}}
\newcommand{\RR}{\mathbb{R}}
\newcommand{\NN}{\mathbb{N}}
\newcommand{\ZZ}{\mathbb{Z}}
\newcommand{\QQ}{\mathbb{Q}}
\newcommand{\Ss}{\mathcal{S}}
\newcommand{\OO}{\mathcal{O}}
\newcommand{\BB}{\mathcal{B}}
\newcommand{\Pcal}{\mathcal{P}}
\newcommand{\FF}{\mathscr{F}}
\newcommand{\GG}{\mathscr{G}}
\newcommand{\Fcal}{\mathcal{F}}
\newcommand{\Gcal}{\mathcal{G}}
\newcommand{\fsc}{\mathscr{F}}
\newcommand{\afr}{\mathfrak{a}}
\newcommand{\bfr}{\mathfrak{b}}
\newcommand{\cfr}{\mathfrak{c}}
\newcommand{\dfr}{\mathfrak{d}}
\newcommand{\efr}{\mathfrak{e}}
\newcommand{\ffr}{\mathfrak{f}}
\newcommand{\gfr}{\mathfrak{g}}
\newcommand{\hfr}{\mathfrak{h}}
\newcommand{\ifr}{\mathfrak{i}}
\newcommand{\jfr}{\mathfrak{j}}
\newcommand{\kfr}{\mathfrak{k}}
\newcommand{\lfr}{\mathfrak{l}}
\newcommand{\mfr}{\mathfrak{m}}
\newcommand{\nfr}{\mathfrak{n}}
\newcommand{\ofr}{\mathfrak{o}}
\newcommand{\pfr}{\mathfrak{p}}
\newcommand{\qfr}{\mathfrak{q}}
\newcommand{\rfr}{\mathfrak{r}}
\newcommand{\sfr}{\mathfrak{s}}
\newcommand{\tfr}{\mathfrak{t}}
\newcommand{\ufr}{\mathfrak{u}}
\newcommand{\vfr}{\mathfrak{v}}
\newcommand{\wfr}{\mathfrak{w}}
\newcommand{\xfr}{\mathfrak{x}}
\newcommand{\yfr}{\mathfrak{y}}
\newcommand{\zfr}{\mathfrak{z}}
\newcommand{\Dcal}{\mathcal{D}}
\newcommand{\Ccal}{\mathcal{C}}
\usepackage{graphicx}
% Swap the definition of \abs* and \norm*, so that \abs
% and \norm resizes the size of the brackets, and the 
% starred version does not.
%\makeatletter
%\let\oldabs\abs
%\def\abs{\@ifstar{\oldabs}{\oldabs*}}
%
%\let\oldnorm\norm
%\def\norm{\@ifstar{\oldnorm}{\oldnorm*}}
%\makeatother
\newenvironment{subproof}[1][\proofname]{%
	\renewcommand{\qedsymbol}{$\blacksquare$}%
	\begin{proof}[#1]%
	}{%
	\end{proof}%
}

\usepackage{centernot}
\usepackage{dirtytalk}
\usepackage{calc}
\newcommand{\prob}[1]{\setcounter{section}{#1-1}\section{}}


\newcommand{\prt}[1]{\setcounter{subsection}{#1-1}\subsection{}}
\newcommand{\pprt}[1]{{\textit{{#1}.)}}\newline}
\renewcommand\thesubsection{\alph{subsection}}
\usepackage[sl,bf,compact]{titlesec}
\titlelabel{\thetitle.)\quad}
\DeclarePairedDelimiter\floor{\lfloor}{\rfloor}
\makeatletter

\newcommand*\pFqskip{8mu}
\catcode`,\active
\newcommand*\pFq{\begingroup
	\catcode`\,\active
	\def ,{\mskip\pFqskip\relax}%
	\dopFq
}
\catcode`\,12
\def\dopFq#1#2#3#4#5{%
	{}_{#1}F_{#2}\biggl(\genfrac..{0pt}{}{#3}{#4}|#5\biggr
	)%
	\endgroup
}
\def\res{\mathop{Res}\limits}
% Symbols \wedge and \vee from mathabx
% \DeclareFontFamily{U}{matha}{\hyphenchar\font45}
% \DeclareFontShape{U}{matha}{m}{n}{
%       <5> <6> <7> <8> <9> <10> gen * matha
%       <10.95> matha10 <12> <14.4> <17.28> <20.74> <24.88> matha12
%       }{}
% \DeclareSymbolFont{matha}{U}{matha}{m}{n}
% \DeclareMathSymbol{\wedge}         {2}{matha}{"5E}
% \DeclareMathSymbol{\vee}           {2}{matha}{"5F}
% \makeatother

%\titlelabel{(\thesubsection)}
%\titlelabel{(\thesubsection)\quad}
\usepackage{listings}
\lstloadlanguages{[5.2]Mathematica}
\usepackage{babel}
\newcommand{\ffac}[2]{{(#1)}^{\underline{#2}}}
\usepackage{color}
\usepackage{amsthm}
\newtheorem{theorem}{Theorem}[section]
\newtheorem*{thm*}{Theorem}
\newtheorem{conj}[theorem]{Conjecture}
\newtheorem{corollary}[theorem]{Corollary}
\newtheorem{example}[theorem]{Example}
\newtheorem{lemma}[theorem]{Lemma}
\newtheorem*{lemma*}{Lemma}
\newtheorem{problem}[theorem]{Problem}
\newtheorem{proposition}[theorem]{Proposition}
\newtheorem*{proposition*}{Proposition}
\newtheorem*{corollary*}{Corollary}
\newtheorem{fact}[theorem]{Fact}
\newtheorem*{prompt*}{Prompt}
\newtheorem*{claim*}{Claim}
\newcommand{\claim}[1]{\begin{claim*} #1\end{claim*}}
%organizing theorem environments by style--by the way, should we really have definitions (and notations I guess) in proposition style? it makes SO much of our text italicized, which is weird.
\theoremstyle{remark}
\newtheorem{remark}{Remark}[section]
\newtheorem*{remark*}{Remark}

\theoremstyle{definition}
\newtheorem{definition}[theorem]{Definition}
\newtheorem*{definition*}{Definition}
\newtheorem{notation}[theorem]{Notation}
\newtheorem*{notation*}{Notation}
%FINAL
\newcommand{\due}{15 November 2017} 
\RequirePackage{geometry}
\geometry{margin=.7in}
\usepackage{todonotes}
\title{MATH 8253 Homework IV}
\author{David DeMark}
\date{\due}
\usepackage{fancyhdr}
\pagestyle{fancy}
\fancyhf{}
\rhead{David DeMark}
\chead{\due}
\lhead{MATH 8253}
\cfoot{\thepage}
\renewcommand{\bar}{\overline}

% %%
%%
%%
%DRAFT

%\usepackage[left=1cm,right=4.5cm,top=2cm,bottom=1.5cm,marginparwidth=4cm]{geometry}
%\usepackage{todonotes}
% \title{MATH 8669 Homework 4-DRAFT}
% \usepackage{fancyhdr}
% \pagestyle{fancy}
% \fancyhf{}
% \rhead{David DeMark}
% \lhead{MATH 8669-Homework 4-DRAFT}
% \cfoot{\thepage}

%PROBLEM SPEFICIC
\renewcommand{\hom}{\mathrm{Hom}}
\newcommand{\lint}{\underline{\int}}
\newcommand{\uint}{\overline{\int}}
\newcommand{\hfi}{\hat{f}^{-1}}
\newcommand{\tfi}{\tilde{f}^{-1}}
\newcommand{\tsi}{\tilde{f}^{-1}}
\newcommand{\PP}{\mathcal{P}}
\newcommand{\nin}{\centernot\in}
\newcommand{\seq}[1]{({#1}_n)_{n\geq 1}}
\newcommand{\Tt}{\mathcal{T}}
\newcommand{\card}{\mathrm{card}}
\newcommand{\setc}[2]{\{ #1\::\:#2 \}}
\newcommand{\idl}[1]{\langle #1 \rangle}
\newcommand{\cl}{\overline}
\newcommand{\id}{\mathrm{id} }
\newcommand{\im}{\mathrm{Im}}
\newcommand{\cat}[1]{{\mathrm{\bf{#1}}}}
%\usepackage[backend=biber,style=alphabetic]{biblatex}
%\addbibresource{algeo.bib}
\newcommand{\colim}{\varinjlim}
\newcommand{\clim}{\varprojlim}
\newcommand{\frp}{\mathop{\large {\mathlarger{\star}}}}
\newcommand{\restr}[2]{\evat{#1}{#2}{}}
\newcommand{\imp}[1]{\underline{#1}}
\newcommand{\ihm}{\imp{\hom}}
\newcommand{\him}{\ihm(\FF,\GG)}
\newcommand{\incla}{\hookrightarrow}
\newcommand{\pre}{\mathrm{pre}}
%\tikzcdset{column sep/tiny=.1cm}
\begin{document}
%\maketitle\emph{To the Grader: I've been sick and overwhelmed this week and this is the best I can do. I'm sorry you have to wade through it.}
\prob{1}\prt{1} \begin{proposition*}
	We let $F$ be an Abelian group, with $\fsc$ the skyscraper sheaf supported at $x\in X$ associated to $F$ and $\Fcal$ the constant sheaf $F$ on $\{x\}$. Then, $\fsc=\iota_*\Fcal$ where $\iota_*$ is induced by $\iota: \{x\}\hookrightarrow X$. 
\end{proposition*}
\begin{proof}
	We recall that for $U\subset X$ open, $(\iota_*\Fcal)(U)=\Fcal(\iota^{-1}(U))$. We note that \begin{equation*}
	\iota^{-1}(U)=\begin{cases}
	\{x\}&x\in U\\
	\emptyset &x\notin U
	\end{cases}\end{equation*}

Thus,
\begin{equation*}
(\iota_*\Fcal)(U)=\Fcal(\iota^{-1}(U))=\begin{cases}
F&x\in U\\
0 &x\notin U
\end{cases}=\fsc(U)\end{equation*}
\end{proof}
\prt{2}\begin{proposition*}
	We let $X$ be a scheme, $x\in X$ a closed point, $k(x)=\OO_{X,x}/\mfr_x$ its residue field, and $F$ a $k(x)$-vector space. Then, there exists a skyscraper sheaf $\Fcal$ with stalk $\Fcal_x=F$ which is quasi-coherent.
\end{proposition*}
\begin{proof}
	We note that $(\{x\},\OO_X)$ is a closed subscheme of $X$ as $i^\#:\OO_{X}\to i_*\OO_{\{x\}}=\OO_{X,x}$ is surjective. Thus, as $\Fcal=i_*F$, the direct image of the skyscraper sheaf on $\{x\}$, by part 1, $\Fcal$ is quasi-coherent.   
\end{proof}
%
%%
%
%
\prob{2}
We begin with a lemma; the proposition will follow as a corollary.
\begin{lemma*}
	We let $A=k[t]$ for some field $k$ and $E\subset k[t]$ a multiplicatively closed subset which is finitely generated as a semigroup. We let $A_E:=A[E^{-1}]$. Then, $k(t)$ is not of finite rank as an $A_E$-module (via the natural inclusion $A_E\hookrightarrow k(t)$).
\end{lemma*}
\begin{subproof}
	We note that $k(t)$ is has irredundant generating set as a $k[t]$-algebra consisting of all elements of the form $1/p(t)$ where $p(t)$ is prime. We note that for any field $k$, the set of prime elements of $k[t]$ is an infinite set.\footnote{As $kt]$ is a UFD, this follows from Euclid's argument for the infinitude of the primes.} Thus, $A_E$ is a finitely generated $k[t]$-subalgebra of $k(t)$, which is infinitely generated, so if $k(t)$ is a finitely-generated $A_E$-algebra it must be a finitely generated $k[t]$-algebra, contradicting the infinitude of prime elements.
\end{subproof}
\begin{corollary*}
	The skyscraper sheaf $\fsc$ on $X=\AA_k^1$ supported at the generic point with stalk $k(t)$ is not locally of finite rank and hence not coherent. 
\end{corollary*}
\begin{proof}
	We note that any open set in $X$ is of the form $D(E)$ where $E\subset A$. A priori, $E$ need not be a finite set, but we note that $D(E)=D(\idl{E})$, and as $k[t]$ is Noetherian, we have that $E$ has some finite generating set $F$. Hence, $D(E)=D(F)$, so $\OO_X(D(E))=A[F^{-1}]$ for some finite set $F\subset A$. Thus, by the lemma, we have that for any open set $U=D(E)\subset X$, $k(t)=\fsc(U)$ is an infinite-rank $\OO_X(U)$-module.
\end{proof}
%
%
%
%
\prob{3}
\begin{proposition*}
	We let $X$ be a scheme and $\Fcal$, $\Gcal$ quasicoherent $\OO_X$-modules. Then, $\Fcal \otimes \Gcal$ is quasicoherent.
\end{proposition*}
\begin{proof}
	By definition, we have that there exists an affine open cover $\{U_i\}_{i\in I}$ (resp. $\{U'_j\}$) of $X$ such that on each $U_i$ (resp. $U'_j$), $\restr{\Fcal}{U_i}\cong M_i^\sim$ (resp. $\restr{\Gcal}{U_j'}\cong M_j'^{}\sim$) for some $\OO_X(U_i)$-module $M_i$ (resp. $\OO_X(U_j')$-module $M_j'$). We let $W_{ij}$ be the open subscheme $U_i\cap U'_j$, and let $\{V_{ijk}\}_{k\in K_{ij}}$ be an affine open cover of $W_{ij}$ for some indexing set $K_{ij}$. We note that as $V_{ijk}\subset U_i$ is an open subscheme of an affine open set on which $\Fcal$ is associated, we have that $\Fcal$ is associated on $V_{ijk}$ with a similar statement holding for $\Gcal$. Hence, $\Fcal$ and $\Gcal$ are both associated on each $V_{ijk}$, and thus their tensor product is as well. As $\bigcup_{(i,j)\in I\times J} \{V_{ijk}\}_{k\in K_{ij}}$ is an affine open cover of $X$, we have completed our proof.
\end{proof}
%
%
%
\prob{4}
\prt{1}\begin{proposition*}
	A scheme $X$ is quasi-compact if and only if it is a finite union of affine open subsets.
\end{proposition*}
\begin{proof}
	($\implies$) We assume $X$ is quasi-compact. By definition of scheme, there exists an affine open set $U_x\ni x$ for all $x\in X$. Then, $\{U_x\}_{x\in X}$ is an open cover of $X$ so there exists some finite subcover $\{U_{x_i}\}_{i=1}^n$.
	
	($\impliedby$) It is a completely trivial exercise in point-set topology that any topological space which can be written as a finite union of quasi-compact sets is itself quasi-compact. As any affine scheme (and hence any affine open subset) is quasi-compact, the proof follows immediately.   
\end{proof}


\prt{2}
\begin{definition}
	\label{def1} A morphism $\pi:X\to Y$ of schemes is quasi-compact if $\pi^{-1}(U)$ is quasi-compact for any quasi-compact $U\subset Y$.
\end{definition}
\begin{definition}
	\label{def2} A morphism $\pi:X\to Y$ of schemes is quasi-compact if $\pi^{-1}(U)$ is quasi-compact for any open affine $U\subset Y$.
\end{definition}
\begin{proposition*}
	Definitions \ref{def1} and \ref{def2} are equivalent.
\end{proposition*}
\begin{proof}
As all open affine sets are quasi-compact, it is immediately clear that definition \ref{def1} implies \ref{def2}. On the other hand, we suppose $\pi$ satisfies definition \ref{def2} and that $U\subset Y$ is quasi-compact. Then, we may cover $U$ by affine open sets $\{W_i\}_{i\in I}$ and have that there exists a finite subcover $\{W_i\}_{i=1}^n$. Then, $\pi^{-1}(U)=\bigcup_{i=1}^n \pi^{-1}(W_i)$, and by assumption each $\pi^{-1}(W_i)$ is quasi-compact. Part a of this question implies the proposition. 
\end{proof}
%
%
%
\prob{5}
%
%
%
\prob{6}
\begin{proposition*}
We let $X$ be a topological space, $U\subset X$ with inclusion $j:U\hookrightarrow X$, and $\Fcal$ a sheaf of sets on $X$. Then, $j^{-1}\Fcal$ coincides with $\restr{\Fcal}{U}$.
\end{proposition*}
\begin{proof}
	We recall that $(j^{-1}\Fcal)^{\pre}(U)=\clim_{V\supset U}\Fcal(V)$. We recall as well that sheaves are recoverable from their stalks via sheafification\textemdash in particular if the stalks of presheaf $\Gcal$ coincide with the stalks of sheaf $\Fcal$, then $\Gcal^+=\Fcal.$ We let $p\in X$ and present the following diagram in which the black arrows commute by definition, with explanation of dashed arrows to follow.
	
	$$
	\begin{tikzcd}[row sep=scriptsize, column sep=tiny]
	&\Fcal(V_\alpha)\arrow[rrrrrr]\arrow[ddddrr]\arrow[dddddrrrrrrrrrr]&&&&&&\Fcal(V_\beta)\arrow[ddddllll,crossing over]\arrow[dddddrrrr]&&&&&&\Fcal(V_\gamma)\arrow[rrrrrr]\arrow[dddddll]&&&&&&\Fcal(V_\delta)\arrow[ddddllll]\arrow[dddddllllllll]&\\%1-2
	&&&&&&&&&&&&&&&&&&&\\%1-3
	&&&&&&&&&&&&&&&&&&&\\%2-1
	&&&&&&&&&&&&&&&&&&&\\%2-2
	&&&(j^{-1}\Fcal)^{\pre}(U_a)\arrow[dddrrrrrr]\arrow[drrrrrrrr,dashrightarrow,swap,"\exists!g_a"]&&&&&&&&&&&&(j^{-1}\Fcal)^{\pre}(U_b)\arrow[dllll,dashrightarrow,"\exists!g_b"]\arrow[from=uuuull,crossing over]\arrow[dddllllll, bend left =10]&&&&&\\%2-3
	&&&&&&&&&&&\Fcal_p\arrow[ddll,bend right=15,dashrightarrow, near end, swap, "\exists! f"]&&&&&&&&\\%3-1
	&&&&&&&&&&&&&&&&&&&\\%3-2
	&&&&&&&&&(j^{-1}\Fcal)_p\arrow[uurr,bend right=15,dashrightarrow,near end, swap, "\exists! h"]&&&&&&&&&&%3-3
	\end{tikzcd}$$
	We note that each $\Fcal(V_\alpha)$ has a map $\Fcal(V_\alpha)\to (j^{-1}\Fcal)^\pre(U_a)\to(j^{-1}\Fcal)_p$ commuting with $\Fcal(V_\alpha \to \Fcal(V_\beta)$ by definition of colimit. Thus, $(j^{-1}\Fcal)_p$ is a co-cone for $\Fcal(V)$ with $V\ni p$, inducing the map $f$. We also note that $\{\Fcal(V)\;:\;U\subset V\}$ is a subco-cone of $\{\Fcal(V)\;:\;p\in V\}$, inducing the maps $g_a,g_b$ in the diagram. Then, the maps $g_a,g_b$ establish $\Fcal_p$ as a co-cone for $(j^{-1}\Fcal)^{\pre}(U)$ for all $U\ni p$, inducing the map $h$. By universality of the colimit construction, we have that $f$ and $h$ are mutual isomorphisms.   
	
\end{proof}

\end{document}