% Exam Template for UMTYMP and Math Department courses
%
% Using Philip Hirschhorn's exam.cls: http://www-math.mit.edu/~psh/#ExamCls
%
% run pdflatex on a finished exam at least three times to do the grading table on front page.
%
%%%%%%%%%%%%%%%%%%%%%%%%%%%%%%%%%%%%%%%%%%%%%%%%%%%%%%%%%%%%%%%%%%%%%%%%%%%%%%%%%%%%%%%%%%%%%%

% These lines can probably stay unchanged, although you can remove the last
% two packages if you're not making pictures with tikz.
\documentclass[11pt]{exam}
\RequirePackage{amssymb, amsfonts, amsmath, mathtools, latexsym, verbatim, xspace, setspace}
\RequirePackage{tikz, pgflibraryplotmarks}

% By default LaTeX uses large margins.  This doesn't work well on exams; problems
% end up in the "middle" of the page, reducing the amount of space for students
% to work on them.
\usepackage[margin=1in]{geometry}


% Here's where you edit the Class, Exam, Date, etc.
\newcommand{\class}{Math 1271 - Lectures 010 and 030 }
\newcommand{\term}{Fall 2017}
\newcommand{\examnum}{Quiz 1II}
\newcommand{\examdate}{09/21/17}
\newcommand{\timelimit}{25 Minutes}
\DeclarePairedDelimiter\abs{\lvert}{\rvert}%
% For an exam, single spacing is most appropriate
\singlespacing
% \onehalfspacing
% \doublespacing

% For an exam, we generally want to turn off paragraph indentation
\parindent 0ex
\def\changemargin#1#2{\list{}{\rightmargin#2\leftmargin#1}\item[]}
\let\endchangemargin=\endlist 
\renewcommand{\d}[1]{\ensuremath{\operatorname{d}\!{#1}}}
\newcommand{\dydx}[2]{\frac{\d #1}{\d #2}}
\newcommand{\ddx}[1]{\frac{\d{}}{\d{#1}}}
\begin{document} 

% These commands set up the running header on the top of the exam pages
\pagestyle{head}
\firstpageheader{}{}{}
\runningheader{\class}{\examnum\ - Page \thepage\ of \numpages}{\examdate}
\runningheadrule

\begin{flushright}
\begin{tabular}{p{2.8in} r l}
\textbf{\class} & \textbf{Name (Print):} & \makebox[2in]{\hrulefill}\\
\textbf{\term} &&\\
\textbf{\examnum} &&\\
\textbf{\examdate} &&\\
\textbf{Time Limit: \timelimit} & Section & \makebox[2in]{\hrulefill}
\end{tabular}\\
\end{flushright}
\rule[1ex]{\textwidth}{.1pt}
This quiz will cover sections 3.4-6 in Stewart.
You may \textit{not} use your books, notes, graphing calculator, phones or any other internet devices on this exam. You are \textbf{required} to show your work on each problem on this quiz. \\
\hspace*{12cm}\begin{minipage}[t]{2.3in}
\vspace{0pt}
%\cellwidth{3em}
\gradetablestretch{2}
\vqword{Problem}
\addpoints % required here by exam.cls, even though questions haven't started yet.	
\gradetable[v]%[pages]  % Use [pages] to have grading table by page instead of question

\end{minipage}
%\newpage % End of cover page

%%%%%%%%%%%%%%%%%%%%%%%%%%%%%%%%%%%%%%%%%%%%%%%%%%%%%%%%%%%%%%%%%%%%%%%%%%%%%%%%%%%%%
%
% See http://www-math.mit.edu/~psh/#ExamCls for full documentation, but the questions
% below give an idea of how to write questions [with parts] and have the points
% tracked automatically on the cover page.
%
%
%%%%%%%%%%%%%%%%%%%%%%%%%%%%%%%%%%%%%%%%%%%%%%%%%%%%%%%%%%%%%%%%%%%%%%%%%%%%%%%%%%%%%


\begin{questions}

% Basic question
\vspace*{-150pt}
\addpoints\question
\begin{changemargin}{0pt}{137.4803pt}
	Compute the following derivatives
	\begin{parts}
		\part[1] $\displaystyle{\ddx{x}\tan\ln(x)}$\vspace{2pt}
		\part[1] $\displaystyle{\ddx{x} \cos^{-1}\left({\sqrt{x^3+e^x}}\right)}$\vspace{2pt}
		\part[2] $\displaystyle{\ddx{x} x^{x^x}}$
\end{parts}
\end{changemargin}	

% Question with parts
\newpage
\addpoints
\question[3]
Write the equation for the tangent line to the curve $C$ defined by the equation $y^3(y-4)=x(x^3-5)$ at the point $(0,4)$
\addpoints\vspace{280pt}
\question[3]
Find $y'$ for $y=\ln(\cos^2(x)+y^2)$

%\noaddpoints

%\addpoints


\end{questions}
\end{document}