\documentclass[english]{article}
\newcommand{\G}{\overline{C_{2k-1}}}
\usepackage[latin9]{inputenc}
\usepackage{amsmath}
\usepackage{amssymb}
\usepackage{lmodern}
\usepackage{mathtools}
\usepackage{enumitem}
%\usepackage{natbib}
%\bibliographystyle{plainnat}
%\setcitestyle{authoryear,open={(},close={)}}
\let\avec=\vec
\renewcommand\vec{\mathbf}
\renewcommand{\d}[1]{\ensuremath{\operatorname{d}\!{#1}}}
\newcommand{\pydx}[2]{\frac{\partial #1}{\partial #2}}
\newcommand{\dydx}[2]{\frac{\d #1}{\d #2}}
\newcommand{\ddx}[1]{\frac{\d{}}{\d{#1}}}
\newcommand{\hk}{\hat{K}}
\newcommand{\hl}{\hat{\lambda}}
\newcommand{\ol}{\overline{\lambda}}
\newcommand{\om}{\overline{\mu}}
\newcommand{\all}{\text{all }}
\newcommand{\valph}{\vec{\alpha}}
\newcommand{\vbet}{\vec{\beta}}
\newcommand{\vT}{\vec{T}}
\newcommand{\vN}{\vec{N}}
\newcommand{\vB}{\vec{B}}
\newcommand{\vX}{\vec{X}}
\newcommand{\vx}{\vec {x}}
\newcommand{\vn}{\vec{n}}
\newcommand{\vxs}{\vec {x}^*}
\newcommand{\vV}{\vec{V}}
\newcommand{\vTa}{\vec{T}_\alpha}
\newcommand{\vNa}{\vec{N}_\alpha}
\newcommand{\vBa}{\vec{B}_\alpha}
\newcommand{\vTb}{\vec{T}_\beta}
\newcommand{\vNb}{\vec{N}_\beta}
\newcommand{\vBb}{\vec{B}_\beta}
\newcommand{\bvT}{\bar{\vT}}
\newcommand{\ka}{\kappa_\alpha}
\newcommand{\ta}{\tau_\alpha}
\newcommand{\kb}{\kappa_\beta}
\newcommand{\tb}{\tau_\beta}
\newcommand{\hth}{\hat{\theta}}
\newcommand{\evat}[3]{\left. #1\right|_{#2}^{#3}}
\newcommand{\prompt}[1]{\begin{prompt*}
		#1
\end{prompt*}}
\newcommand{\vy}{\vec{y}}
\DeclareMathOperator{\sech}{sech}
\DeclarePairedDelimiter\abs{\lvert}{\rvert}%
\DeclarePairedDelimiter\norm{\lVert}{\rVert}%
\newcommand{\dis}[1]{\begin{align*}
	#1
	\end{align*}}
\newcommand{\LL}{\mathcal{L}}
\newcommand{\RR}{\mathbb{R}}
\newcommand{\NN}{\mathbb{N}}
\newcommand{\ZZ}{\mathbb{Z}}
\newcommand{\QQ}{\mathbb{Q}}
\newcommand{\Ss}{\mathcal{S}}
\newcommand{\BB}{\mathcal{B}}
\usepackage{graphicx}
% Swap the definition of \abs* and \norm*, so that \abs
% and \norm resizes the size of the brackets, and the 
% starred version does not.
%\makeatletter
%\let\oldabs\abs
%\def\abs{\@ifstar{\oldabs}{\oldabs*}}
%
%\let\oldnorm\norm
%\def\norm{\@ifstar{\oldnorm}{\oldnorm*}}
%\makeatother
\newenvironment{subproof}[1][\proofname]{%
	\renewcommand{\qedsymbol}{$\blacksquare$}%
	\begin{proof}[#1]%
	}{%
	\end{proof}%
}

\usepackage{centernot}
\usepackage{dirtytalk}
\usepackage{calc}
\newcommand{\prob}[1]{\setcounter{section}{#1-1}\section{}}
\newcommand{\prt}[1]{\setcounter{subsection}{#1-1}\subsection{}}
\newcommand{\pprt}[1]{{\textit{{#1}.)}}\newline}
\renewcommand\thesubsection{\alph{subsection}}
\usepackage[sl,bf,compact]{titlesec}
\titlelabel{\thetitle.)\quad}
\DeclarePairedDelimiter\floor{\lfloor}{\rfloor}
\makeatletter

\newcommand*\pFqskip{8mu}
\catcode`,\active
\newcommand*\pFq{\begingroup
	\catcode`\,\active
	\def ,{\mskip\pFqskip\relax}%
	\dopFq
}
\catcode`\,12
\def\dopFq#1#2#3#4#5{%
	{}_{#1}F_{#2}\biggl(\genfrac..{0pt}{}{#3}{#4}|#5\biggr
	)%
	\endgroup
}
\def\res{\mathop{Res}\limits}
% Symbols \wedge and \vee from mathabx
% \DeclareFontFamily{U}{matha}{\hyphenchar\font45}
% \DeclareFontShape{U}{matha}{m}{n}{
%       <5> <6> <7> <8> <9> <10> gen * matha
%       <10.95> matha10 <12> <14.4> <17.28> <20.74> <24.88> matha12
%       }{}
% \DeclareSymbolFont{matha}{U}{matha}{m}{n}
% \DeclareMathSymbol{\wedge}         {2}{matha}{"5E}
% \DeclareMathSymbol{\vee}           {2}{matha}{"5F}
% \makeatother

%\titlelabel{(\thesubsection)}
%\titlelabel{(\thesubsection)\quad}
\usepackage{listings}
\lstloadlanguages{[5.2]Mathematica}
\usepackage{babel}
\newcommand{\ffac}[2]{{(#1)}^{\underline{#2}}}
\usepackage{color}
\usepackage{amsthm}
\newtheorem{theorem}{Theorem}[section]
%\newtheorem*{theorem*}{Theorem}[section]
\newtheorem{conj}[theorem]{Conjecture}
\newtheorem{corollary}[theorem]{Corollary}
\newtheorem{example}[theorem]{Example}
\newtheorem{lemma}[theorem]{Lemma}
\newtheorem*{lemma*}{Lemma}
\newtheorem{problem}[theorem]{Problem}
\newtheorem{proposition}[theorem]{Proposition}
\newtheorem*{prop*}{Proposition}
\newtheorem*{corollary*}{Corollary}
\newtheorem{fact}[theorem]{Fact}

\newtheorem*{claim*}{Claim}
\newcommand{\claim}[1]{\begin{claim*} #1\end{claim*}}
%organizing theorem environments by style--by the way, should we really have definitions (and notations I guess) in proposition style? it makes SO much of our text italicized, which is weird.
\theoremstyle{remark}
\newtheorem{remark}{Remark}[section]

\theoremstyle{definition}
\newtheorem{definition}[theorem]{Definition}
\newtheorem{notation}[theorem]{Notation}
\newtheorem*{notation*}{Notation}

%FINAL
\newcommand{\due}{3 October 2017} 
\RequirePackage{geometry}
\geometry{margin=.7in}
\usepackage{todonotes}
\title{Worksheet for 10/3\textemdash David's Solutions}
\author{David DeMark}
\date{\due}
\usepackage{fancyhdr}
\pagestyle{fancy}
\fancyhf{}
\rhead{Groupwork 10/3/17}
\chead{\due}
\lhead{MATH 1271-012\&016}
\cfoot{\thepage}
% %%
%%
%%
%DRAFT

%\usepackage[left=1cm,right=4.5cm,top=2cm,bottom=1.5cm,marginparwidth=4cm]{geometry}
%\usepackage{todonotes}
% \title{MATH 8669 Homework 4-DRAFT}
% \usepackage{fancyhdr}
% \pagestyle{fancy}
% \fancyhf{}
% \rhead{David DeMark}
% \lhead{MATH 8669-Homework 4-DRAFT}
% \cfoot{\thepage}

%PROBLEM SPEFICIC

\newcommand{\lint}{\underline{\int}}
\newcommand{\uint}{\overline{\int}}
\newcommand{\hfi}{\hat{f}^{-1}}
\newcommand{\tfi}{\tilde{f}^{-1}}
\newcommand{\tsi}{\tilde{f}^{-1}}
\newcommand{\PP}{\mathcal{P}}
\newcommand{\nin}{\centernot\in}
\newcommand{\seq}[1]{({#1}_n)_{n\geq 1}}
\usepackage{array}
\newcolumntype{M}[1]{>{\centering\arraybackslash}m{#1}}
\newcolumntype{N}{@{}m{0pt}@{}}
\input ArtNouvc.fd
\newcommand*\initfamily{\usefont{U}{ArtNouvc}{xl}{n}}
\begin{document}
	\maketitle

\prob{1} Let $f(x)$ and $g(z)$ be continuous and differentiable functions. \prt{1} 
\textbf{Find formulas} (in terms of $f$, $f'$, $g$, $g'$, $x$, $z$, $x_0$, and $z_0$) \textbf{for $T_f(x)$, the tangent line to the graph $y=f(x)$ taken at $x_0$, and $T_g(z)$, the tangent line to the graph $y=g(z)$ taken at $z_0$.} (these should be more or less the same formula, of course\textellipsis)\\
\begin{proof}[Response]
This is basically an application of point-slope form! Let's write that out for $T_f(x)$ with slope $m$ and point $(w_0,y_0)$\textellipsis 
$$T_f(x)-y_0=m(x-w_0)$$
Now, we need to figure out (a) what our slope is and (b) what our point is. We want $T_f(x)$ to give the formula for the \emph{tangent line} of the curve $y=f(x)$ at $x=x_0$\textemdash i.e. we want $T_f(x)$ to \say{hit} $y=f(x)$ when $x=x_0$. The derivative tells us the slope of the tangent line, so our slope $m$ is $f'(x_0)$. We also only know of one point on $T_f(x)$: it \say{hits} $y=f(x)$ when $x=x_0$. That means that we know the point $(x_0,f(x_0))$ must be on the graph $y=T_f(x)$, so $w_0=x_0$ and $y_0=f(x_0)$. We now get the formula:
$$T_f(x)-f(x_0)=f'(x_0)(x-x_0)$$
or, moving everything to one side to give a formula for $T_f$:
$$T_f(x)=f'(x_0)(x-x_0)+f(x_0).$$
The formula for $T_g$ is found completely analogously:
$$T_g(z)=g'(z_0)(z-z_0)+g(z_0).$$
\end{proof} 
\prt{2}\textbf{Let $z_0=f(x_0)$. Adjust your formula for $T_g(z)$ appropriately.}
\begin{proof}[Response]
	Okay! Literally all this part is asking is to replace every $z_0$ with a $f(x_0)$--we are simply declaring $z_0$ to be that.
	$$T_g(z)=g'(z_0)(z-z_0)+g(z_0)=g'(f(x_0))\left(z-f(x_0)\right)+g(f(x_0))$$
\end{proof}
\noindent\textbf{ Write out the formula for the composition $F(x)=T_g(T_f(x))$} (of course, $T_f(x)$ is still the tangent at $x_0$)
\begin{proof}[Response]
	Now, all we are being asked to do is substitute $z=T_f(x)$ in the formula for $T_g$ to find $F(x)=T_g(T_f(x))$.
	\begin{align*}
	F(x)=	T_g(T_f(x))&=g'(f(x_0))\left(T_f(x)-f(x_0)\right)+g(f(x_0))\\
		&=g'(f(x_0))\left(\left(f'(x_0)(x-x_0)+f(x_0)\right)-f(x_0)\right)+g(f(x_0))
	\end{align*}
	That is correct, but very ugly. The $f(x_0)$'s cancel, which makes things considerably better.
	\begin{align*}
	F(x)=f'(x_0)g'(f(x_0))(x-x_0)+g(f(x_0))
	\end{align*}
	\end{proof}

\prt{3}\textbf{Find the slope of $F(x)$. This should look familiar\textemdash what is it?? What do you think I'm trying to illustrate by putting this question on here?}
\begin{proof}[Response]
	So! $F(x)$ is a \textbf{\emph{linear}} function. If I give you a function of the form $y=a(x-x_0)+b$, what is its slope? That's right, it's $a$. Here, we can do the same thing to find that the slope of $F(x)$ is $f'(x_0)g'(f(x_0))$. This should look familiar\textellipsis that's right it's the chain rule!
	
	Remember how on tuesday, we looked at an animation of zooming into the function until it looks identical to the tangent line? That is precisely what's going on here! The point is, near $x_0$ and $z_0$, the tangent lines $T_f$ and $T_g$ are similar enough to our original function that they are \emph{basically} interchangeable with the real thing. So, if we want the slope of the tangent line of the composition (that is, the tangent line for $(g\circ f) (x)$ at $x=x_0$, it is enough to compose the tangent lines at $x_0$ and $f(x_0)$! This is almost always harder to do than just actually use the chain rule, but I think it gives some neat intuition for what the chain rule is actually getting at--and a way to derive the formula without writing out the limit-definition of a derivative once!!
\end{proof}

I'll leave the rest of the worksheet for y'all to work out.
\prob{2}\textbf{Compute the following derivatives. Write your answer in a form that \emph{feels} simplified to you.} (that is, if you find yourself expanding a 149-term polynomial, something has gone horribly \emph{horribly} wrong).
\prt{1} $$\ddx{\theta}{\cos(\sin(\theta))}=$$\vspace{5cm}
\prt{2}$$\ddx{x}(x^3+x+1)^{50}=$$\vspace{5cm}
\prt{3}$$\ddx{x}e^{\frac{x+3}{2\sin(x)}}$$\vspace{5cm}

\prob{3} \textbf{Use the \emph{definition of the derivative} to find $f'(x)$ where $f(x)=\sin x$.} You may use the identity $\sin(x+y)=\sin(x)\cos(y)+\cos(x)\sin(y)$ and the limit identities found on pages 191-192.
\end{document}
