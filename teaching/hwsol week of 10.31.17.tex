\documentclass[english]{article}
\newcommand{\G}{\overline{C_{2k-1}}}
\usepackage[latin9]{inputenc}
\usepackage{amsmath}
\usepackage{amssymb}
\usepackage{lmodern}
\usepackage{mathtools}
\usepackage{enumitem}
\usepackage{relsize}

%\usepackage{natbib}
%\bibliographystyle{plainnat}
%\setcitestyle{authoryear,open={(},close={)}}
\let\avec=\vec
\renewcommand\vec{\mathbf}
\renewcommand{\d}[1]{\ensuremath{\operatorname{d}\!{#1}}}
\newcommand{\pydx}[2]{\frac{\partial #1}{\partial #2}}
\newcommand{\dydx}[2]{\frac{\d #1}{\d #2}}
\newcommand{\ddx}[1]{\frac{\d{}}{\d{#1}}}
\newcommand{\hk}{\hat{K}}
\newcommand{\hl}{\hat{\lambda}}
\newcommand{\ol}{\overline{\lambda}}
\newcommand{\om}{\overline{\mu}}
\newcommand{\all}{\text{all }}
\newcommand{\valph}{\vec{\alpha}}
\newcommand{\vbet}{\vec{\beta}}
\newcommand{\vT}{\vec{T}}
\newcommand{\vN}{\vec{N}}
\newcommand{\vB}{\vec{B}}
\newcommand{\vX}{\vec{X}}
\newcommand{\vx}{\vec {x}}
\newcommand{\vn}{\vec{n}}
\newcommand{\vxs}{\vec {x}^*}
\newcommand{\vV}{\vec{V}}
\newcommand{\vTa}{\vec{T}_\alpha}
\newcommand{\vNa}{\vec{N}_\alpha}
\newcommand{\vBa}{\vec{B}_\alpha}
\newcommand{\vTb}{\vec{T}_\beta}
\newcommand{\vNb}{\vec{N}_\beta}
\newcommand{\vBb}{\vec{B}_\beta}
\newcommand{\bvT}{\bar{\vT}}
\newcommand{\ka}{\kappa_\alpha}
\newcommand{\ta}{\tau_\alpha}
\newcommand{\kb}{\kappa_\beta}
\newcommand{\tb}{\tau_\beta}
\newcommand{\hth}{\hat{\theta}}
\newcommand{\evat}[3]{\left. #1\right|_{#2}^{#3}}
\newcommand{\prompt}[1]{\begin{prompt*}
		#1
\end{prompt*}}
\newcommand{\vy}{\vec{y}}
\DeclareMathOperator{\sech}{sech}
\DeclarePairedDelimiter\abs{\lvert}{\rvert}%
\DeclarePairedDelimiter\norm{\lVert}{\rVert}%
\newcommand{\dis}[1]{\begin{align}
	#1
	\end{align}}
\newcommand{\LL}{\mathcal{L}}
\newcommand{\RR}{\mathbb{R}}
\newcommand{\NN}{\mathbb{N}}
\newcommand{\ZZ}{\mathbb{Z}}
\newcommand{\QQ}{\mathbb{Q}}
\newcommand{\Ss}{\mathcal{S}}
\newcommand{\BB}{\mathcal{B}}
\usepackage{graphicx}
% Swap the definition of \abs* and \norm*, so that \abs
% and \norm resizes the size of the brackets, and the 
% starred version does not.
%\makeatletter
%\let\oldabs\abs
%\def\abs{\@ifstar{\oldabs}{\oldabs*}}
%
%\let\oldnorm\norm
%\def\norm{\@ifstar{\oldnorm}{\oldnorm*}}
%\makeatother
\newenvironment{subproof}[1][\proofname]{%
	\renewcommand{\qedsymbol}{$\blacksquare$}%
	\begin{proof}[#1]%
	}{%
	\end{proof}%
}

\usepackage{centernot}
\usepackage{dirtytalk}
\usepackage{calc}
\newcommand{\prob}[1]{\setcounter{section}{#1-1}\section{}}


\newcommand{\prt}[1]{\setcounter{subsection}{#1-1}\subsection{}}
\newcommand{\pprt}[1]{{\textit{{#1}.)}}\newline}
\renewcommand\thesubsection{\alph{subsection}}
\usepackage[sl,bf,compact]{titlesec}
\titlelabel{\thetitle.)\quad}
\DeclarePairedDelimiter\floor{\lfloor}{\rfloor}
\makeatletter

\newcommand*\pFqskip{8mu}
\catcode`,\active
\newcommand*\pFq{\begingroup
	\catcode`\,\active
	\def ,{\mskip\pFqskip\relax}%
	\dopFq
}
\catcode`\,12
\def\dopFq#1#2#3#4#5{%
	{}_{#1}F_{#2}\biggl(\genfrac..{0pt}{}{#3}{#4}|#5\biggr
	)%
	\endgroup
}
\def\res{\mathop{Res}\limits}
% Symbols \wedge and \vee from mathabx
% \DeclareFontFamily{U}{matha}{\hyphenchar\font45}
% \DeclareFontShape{U}{matha}{m}{n}{
%       <5> <6> <7> <8> <9> <10> gen * matha
%       <10.95> matha10 <12> <14.4> <17.28> <20.74> <24.88> matha12
%       }{}
% \DeclareSymbolFont{matha}{U}{matha}{m}{n}
% \DeclareMathSymbol{\wedge}         {2}{matha}{"5E}
% \DeclareMathSymbol{\vee}           {2}{matha}{"5F}
% \makeatother

%\titlelabel{(\thesubsection)}
%\titlelabel{(\thesubsection)\quad}
\usepackage{listings}
\lstloadlanguages{[5.2]Mathematica}
\usepackage{babel}
\newcommand{\ffac}[2]{{(#1)}^{\underline{#2}}}
\usepackage{color}
\usepackage{amsthm}
\newtheorem{theorem}{Theorem}[section]
\newtheorem*{theorem*}{Theorem}
\newtheorem{conj}[theorem]{Conjecture}
\newtheorem{corollary}[theorem]{Corollary}
\newtheorem{example}[theorem]{Example}
\newtheorem{lemma}[theorem]{Lemma}
\newtheorem*{lemma*}{Lemma}
\newtheorem{problem}[theorem]{Problem}
\newtheorem*{problem*}{Problem}
\newtheorem{proposition}[theorem]{Proposition}
\newtheorem*{proposition*}{Proposition}
\newtheorem*{corollary*}{Corollary}
\newtheorem{fact}[theorem]{Fact}
\newtheorem*{prompt*}{Prompt}
\newtheorem*{claim*}{Claim}
\newcommand{\claim}[1]{\begin{claim*} #1\end{claim*}}
%organizing theorem environments by style--by the way, should we really have definitions (and notations I guess) in proposition style? it makes SO much of our text italicized, which is weird.
\theoremstyle{remark}
\newtheorem{remark}{Remark}[section]

\theoremstyle{definition}
\newtheorem{definition}[theorem]{Definition}
\newtheorem{notation}[theorem]{Notation}
\newtheorem*{notation*}{Notation}
%FINAL
\newcommand{\due}{31 October 2017} 
\RequirePackage{geometry}
\geometry{margin=.7in}
\usepackage{todonotes}
\title{Selected Solutions for \S 4.[5,7,8]}
\author{David DeMark}
\date{\due}
\usepackage{fancyhdr}
\pagestyle{fancy}
\fancyhf{}
\rhead{David DeMark}
\chead{\due}
\lhead{MATH 1271 sec 012 \& 016}
\cfoot{\thepage}
% %%
%%
%%
%DRAFT

%\usepackage[left=1cm,right=4.5cm,top=2cm,bottom=1.5cm,marginparwidth=4cm]{geometry}
%\usepackage{todonotes}
% \title{MATH 8669 Homework 4-DRAFT}
% \usepackage{fancyhdr}
% \pagestyle{fancy}
% \fancyhf{}
% \rhead{David DeMark}
% \lhead{MATH 8669-Homework 4-DRAFT}
% \cfoot{\thepage}

%PROBLEM SPEFICIC

\newcommand{\lint}{\underline{\int}}
\newcommand{\uint}{\overline{\int}}
\newcommand{\hfi}{\hat{f}^{-1}}
\newcommand{\tfi}{\tilde{f}^{-1}}
\newcommand{\tsi}{\tilde{f}^{-1}}
\newcommand{\PP}{\mathcal{P}}
\newcommand{\nin}{\centernot\in}
\newcommand{\seq}[1]{({#1}_n)_{n\geq 1}}
\newcommand{\Tt}{\mathcal{T}}
\newcommand{\card}{\mathrm{card}}
\newcommand{\setc}[2]{\{ #1\::\:#2 \}}
\newcommand{\Fcal}{\mathcal{F}}
\newcommand{\cbal}{\overline{B}}
\newcommand{\Ccal}{\mathcal{C}}
\newcommand{\Dcal}{\mathcal{D}}
\newcommand{\cl}{\overline}
\newcommand{\id}{\mathrm{id}}
\newcommand{\intr}{\mathrm{int}}
\renewcommand{\hom}{\mathrm{Hom}}
\newcommand{\vect}{\mathrm{Vect}}
\newcommand{\Top}{\mathrm{Top}}
\renewcommand{\top}{\Top}
\newcommand{\hTop}{\mathrm{hTop}}
\newcommand{\set}{\mathrm{Set}}
\newcommand{\frp}{\mathop{\large {\mathlarger{\star}}}}
\newcommand{\ondt}{1_{\cdot}}
\newcommand{\onst}{1_{\star}}
\begin{document}
	\maketitle

	\section*{4.7-- \#37)}
	\begin{problem*}
	A piece of wire 10 m long is cut into two pieces. One is bent into a square, and the other is bent into an equilateral triangle. How should the wire be cut so that the total area enclosed is (a) maximized? (b) minimized?
	\end{problem*}
\begin{proof}[Response]
Set-up: Let's let the square have side-length $s_1$ and the triangle have side-length $s_2$. Recall that the area of an equilateral triangle with side-length $s_2$ is $\frac{\sqrt 3}{4}s_2^2$ (if you don't happen to remember this, try drawing a line from a vertex to the middle of the side opposite, then use special right triangles to find the area of each of the two triangles you cut it into!). Then, the area of the two shapes combined is \begin{equation}\label{are1}A=s_1^2+\frac{\sqrt{3}}{4}s_2^2.\end{equation} In order to min/maximize $A$, we should first relate $s_1$ and $s_2$ so that $A$ is a function of a \emph{single} variable. Our \textbf{{constraint}} is that we only have 10 m of wire to work with, so the perimeter of the square and the perimeter of the triangle can only sum to 10 m\textemdash i.e. $$4s_1+3s_2=10\hspace{3em}\text{ with }s_1,s_2>0$$. This relates $s_1$ and $s_2$; in particular, we can solve for $s_1$ and get $$s_1=\frac{10-3s_2}{4}.$$ Note that both sides must be greater than or equal to zero, so in particular $0\leq s_2\leq \frac{10}{3}$. Now, we can substitute our new form for $s_1$ into \eqref{are1}:
\begin{equation}
\label{are2}	A(s_2)=\left(\frac{10-3s_2}{4}\right)^2+\frac{\sqrt{3}}{4}s_2^2
\end{equation}
From now on, let's drop the subscript and rename $s_2$ to $s$ to make things simpler. Let's also expand \eqref{are2}.

\begin{align}
	A(s)&=\left(\frac{10-3s}{4}\right)^2+\frac{\sqrt{3}}{4}s^2\\
A(s)	&=\frac{4\sqrt{3}+9 }{16}s^2-\frac{15}{4}s+\frac{25}{4}\label{are3}
\end{align}
So, with constraint $0\leq s\leq \frac{10}{3}$, this means we need to max/minimize \eqref{are3} with $s$ in the interval $[0,\frac{10}{3}]$. First let's check for critical points by taking a derivative and solving for $s_0$ such that $A(s_0)=0$:
\begin{align*}
A'(s)&=\frac{4\sqrt{3}+9}{8}s-\frac{15}{4}\\
0&=\frac{4\sqrt{3} +9}{8}s_0-\frac{15}{4}\\
\implies s_0&= \frac{30}{4\sqrt{3}+9}
\end{align*} 
Note that $0\leq \frac{30}{4\sqrt{3}+9}<\frac{30}{9}=\frac{10}{3}$, so $s_0$ is in our interval. $A''(s)=\frac{4\sqrt{3}+9}{8}$ is a constant function at a positive constant, so we can conclude by the second-derivative test that $s_0$ is a \textbf{local minimum.} In fact, because $A(s)$ is a quadratic (i.e. its graph is a parabola), we can conclude that $s_0$ is the $x$-coordinate for the \emph{global} minimum of $A(s)$. Thus, our answer to part (b) is that there is a local minimum at $(s_0,A(s_0))=\left(\frac{30}{4\sqrt{3}+9},\frac{25}{11} \left(3 \sqrt{3}-4\right)\right)\approx(1.88345,2.71853)$.
%\footnote{Incidentally, on a problem like this, I guess it would be sort of impractical to answer in exact form, so I guess I should walk back my whole thing on decimals a bit\textellipsis} 
This corresponds to cutting the wire into two pieces, one of length $3s_0=\frac{90}{4\sqrt{3}+9}$ and the other of length $10-3s_0$. Next, testing the endpoints of our interval for a maximum, we have that $A(0)=\frac{25}{4}$, and $A(\frac{10}{3})=\frac{25}{3\sqrt{3}}$. By squaring, we can see that $(3\sqrt{3})^2=27>16=4^2$, so we can conclude $3\sqrt{3}>4$, or $\frac{25}{3\sqrt{3}}<\frac{25}{4}$, so the maximum on the interval is $(0,\frac{25}{4})$. Remember that $s$ was the length of our triangle--since $s=0$, this shows that the area of the two shapes is maximized when the entire length of wire encloses a square with perimeter $10$.
\end{proof}	\section*{4.8--\#15) }\begin{problem*}
Approximate the negative root of $e^x=4-x^2$ to six decimal places.
\end{problem*}
\textbf{Note: For this problem, and problems of this sort, I am \emph{suspending} the rule regarding decimal expansions.}
\begin{proof}[Response]
Let's rephrase this a bit. Let $f(x)=x^2+e^x-4$. Then, a negative root of $f$ corresponds to one of the equation in the problem. I'd imagine that the negative root is probably pretty close to $x=-2$ as $(-2)^2-4=0$ which isn't that far off from $e^{-2}$. Let's use that then. First, let's compute $f'(x)$:
\begin{equation}
	f'(x)=2x+e^x
\end{equation}
We let $x_0=-2$ as discussed. Then, the tangent line to $f$ at $x_0$ is $$T_0(x)=f'(-2)(x+2)+f(-2)=(e^{-2}-4)(x+2)+(e^{-2}).$$
Now, we solve $T_0(x_1)=0$ to find our $x_1$:
\begin{align*}0&=(e^{-2}-4)(x_1+2)+(e^{-2})\\
\implies x_1&=\frac{-e^{-2}}{e^{-2}-4}-2\approx -1.9649814
\end{align*}
Now, let's repeat that process a few more times \begin{align*}x_2=\frac{-f(x_1)}{f'(x_1)}+x_1&\approx -1.9646356\\
x_3=\frac{-f(x_2)}{f'(x_2)}+x_2&\approx -1.9646356\\
\end{align*}
Amazing! We've only done three iterations, and we've already approximated it to eight decimal places! Our final answer (to six places) for the solution $x_r$ is $$x_r\approx-1.9646356$$
\end{proof}
\end{document}
