% Exam Template for UMTYMP and Math Department courses
%
% Using Philip Hirschhorn's exam.cls: http://www-math.mit.edu/~psh/#ExamCls
%
% run pdflatex on a finished exam at least three times to do the grading table on front page.
%
%%%%%%%%%%%%%%%%%%%%%%%%%%%%%%%%%%%%%%%%%%%%%%%%%%%%%%%%%%%%%%%%%%%%%%%%%%%%%%%%%%%%%%%%%%%%%%

% These lines can probably stay unchanged, although you can remove the last
% two packages if you're not making pictures with tikz.
\documentclass[11pt]{exam}
\RequirePackage{amssymb, amsfonts, amsmath, mathtools, latexsym, verbatim, xspace, setspace}
\RequirePackage{tikz, pgflibraryplotmarks}

% By default LaTeX uses large margins.  This doesn't work well on exams; problems
% end up in the "middle" of the page, reducing the amount of space for students
% to work on them.
\usepackage[margin=1in]{geometry}


% Here's where you edit the Class, Exam, Date, etc.
\newcommand{\class}{Math 1271 - Lectures 010 and 030 }
\newcommand{\term}{Fall 2017}
\newcommand{\examnum}{Quiz 1}
\newcommand{\examdate}{09/07/17}
\newcommand{\timelimit}{25 Minutes}
\DeclarePairedDelimiter\abs{\lvert}{\rvert}%
% For an exam, single spacing is most appropriate
\singlespacing
% \onehalfspacing
% \doublespacing

% For an exam, we generally want to turn off paragraph indentation
\parindent 0ex
\def\changemargin#1#2{\list{}{\rightmargin#2\leftmargin#1}\item[]}
\let\endchangemargin=\endlist 

\begin{document} 

% These commands set up the running header on the top of the exam pages
\pagestyle{head}
\firstpageheader{}{}{}
\runningheader{\class}{\examnum\ - Page \thepage\ of \numpages}{\examdate}
\runningheadrule

\begin{flushright}
\begin{tabular}{p{2.8in} r l}
\textbf{\class} & \textbf{Name (Print):} & \makebox[2in]{\hrulefill}\\
\textbf{\term} &&\\
\textbf{\examnum} &&\\
\textbf{\examdate} &&\\
\textbf{Time Limit: \timelimit} & Section & \makebox[2in]{\hrulefill}
\end{tabular}\\
\end{flushright}
\rule[1ex]{\textwidth}{.1pt}

You may \textit{not} use your books, notes, graphing calculator, phones or any other internet devices on this exam.\\

You are \textbf{required} to show your work on each problem on this quiz. If you are unable to demonstrate your answer in full rigor, supporting evidence may possibly be redeemed for partial credit.\\
\hspace*{12cm}\begin{minipage}[t]{2.3in}
\vspace{0pt}
%\cellwidth{3em}
\gradetablestretch{2}
\vqword{Problem}
\addpoints % required here by exam.cls, even though questions haven't started yet.	
\gradetable[v]%[pages]  % Use [pages] to have grading table by page instead of question

\end{minipage}
%\newpage % End of cover page

%%%%%%%%%%%%%%%%%%%%%%%%%%%%%%%%%%%%%%%%%%%%%%%%%%%%%%%%%%%%%%%%%%%%%%%%%%%%%%%%%%%%%
%
% See http://www-math.mit.edu/~psh/#ExamCls for full documentation, but the questions
% below give an idea of how to write questions [with parts] and have the points
% tracked automatically on the cover page.
%
%
%%%%%%%%%%%%%%%%%%%%%%%%%%%%%%%%%%%%%%%%%%%%%%%%%%%%%%%%%%%%%%%%%%%%%%%%%%%%%%%%%%%%%


\begin{questions}

% Basic question
\vspace*{-130pt}
\addpoints\question\begin{changemargin}{0}{4.85cm} Find the real numbers within the interval $I$ at which $f$ is \emph{not} continuous. State whether $f$ is continuous from the right, the left, or neither. Justify with words and/or a graph of $y=f(x)$.
\begin{parts}
\part[1] $\displaystyle{f(x)=\frac{1}{x^2+5x-14}}$\hfill $I=(-\infty,\infty)$.
\part[1] $\displaystyle{f(x)=\frac{\cos{\frac{x}{2}}}{\sin{2x}}}$\hfill $I=[-\pi,\pi]$
\part[1] $\displaystyle{f(x)=\begin{cases}3^x&x\leq 2\\-3x+15&2<x\leq 4\\\sqrt{x}&x>4
	\end{cases}}$\hfill $I=(-3,8)$
\end{parts}
\end{changemargin}
% Question with parts
\newpage
\addpoints
\question[4] Sketch a graph of a function $f(x)$ satisfying the following properties:\\
$\lim_{x\to-\infty}f(x)=-1$,~ $\lim_{x\to -1^+}f(x)=3$,~ $\lim_{x\to 0}=\infty$,~ $\lim_{x\to 2}f(x)=4$,~ $f(2)=3$,~ $\lim_{x\to \infty} f(x)=1$,~ $f$ has only two points of discontinuity in the real numbers. 

\vskip70mm
\addpoints
\question
%\noaddpoints
Evaluate the following limits. State explicitly which limit laws you are employing: \begin{parts}
\part[1] $\displaystyle{\lim_{x\to \infty} \frac{\cos^2(x)}{x^2-3}}$ \vspace{1.5in}
\part[1] $\displaystyle{\lim_{x\to \infty} \frac{3x^2-2x+1}{2x^2+1}}$
\vspace{1.5in}
\part[1] $\displaystyle{\lim_{x\to -\infty} \frac{\abs{3x^3-1}}{x^3+2x^2}}$ 
\end{parts}
%\addpoints


\end{questions}
\end{document}