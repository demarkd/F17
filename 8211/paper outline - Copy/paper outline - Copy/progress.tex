\documentclass[12pt,letter]{article}
\usepackage{genpape}
\usepackage[style=alphabetic,doi=false,isbn=false,url=false]{biblatex}
\usepackage{dirtytalk}
\bibliography{trop} 
\newcommand{\gr}{\mathrm{Gr}}
\newcommand{\grkn}[2]{\mathrm{Gr}_{#1,#2}}
\newcommand{\trop}{\mathrm{Trop}}
\title{Tropical Grassmannians and Clusters: Associating $\trop^+\gr_{3,6}$ with cluster algebras of type $D_4$.}
\date{9 October 2017}
\author{David DeMark}

\begin{document}
\maketitle\setcounter{section}{-2}
\section{Progress Report}
To this point, my primary activity towards the completion of this paper has been towards gaining my own bearings on the material. I read through \cite{WiSp05}, \cite{He09} and of course \cite{BCL16} in order to determine what material would be pre-requisite to understanding theorem 4.1 of \cite{BCL16}, and have since undertaken a thorough study of the relevant chapters of \cite{MaSt15}. This process is still underway as I was wholly unfamiliar with large parts of the theory required here (especially that of polyhedral geometry and Gr\"obner basis theory), and what is below represents an outline of the prerequisite material as well as some motivating exposition on Grassmannians and total positivity in general (this may ultimately find itself in the introduction by the 13th). My work on the computational side is thus far in its infancy; I've determined that I will likely be utilizing \verb|gfan| and \verb|polymake|, so my progress thus far has been confined to figuring out how to install those\textemdash in the coming days I'm hoping to gain sufficient comfort with these (as well as possibly \verb|homology| and \verb|macaulay2| per the recommendation of \cite{He09}) to begin my work on actually computing the combinatorial structure of $F_{3,6}$. 
\section{Introduction}
\section{Preliminaries \&~Basic Definitions}

\subsection{Grassmannians}
Let $\KK$ be a field fixed over the duration of this subsection. The \emph{$k,n$-Grassmannian over $\KK$} (denoted $\gr_{k,n}(\KK)$ or just $\gr_{k,n}$ when $\KK$ is clear from context) can be thought of as the $k$-dimensional subspaces through the origin in the $\KK$-vector space $\KK^n$. There are several ways of defining the Grassmannian one can refer to, including one \cite[\S1.2]{khat} in which $\gr_{k,n}$ is actually viewed the set of such $k$-dimensional subspaces, which is then topologized via a surjection from the Stiefel manifold of $k$-touples of orthonormal vectors in $\KK^n$. Here, we present the definition of the grassmannian most common in algebraic geometry,
%(roughly following \cite[\S6]{harris}),
with the na\"ive understanding of above as inspiration.

A $k$-dimensional subspace $S$ of $\KK^n$ can be described by a basis, which we represent here as a full-rank $k\times n$ matrix $P$. This description is not unique however, indeed $S$ is invariant under the action of $GL(S)$. This points to another definition of $\gr_{k,n}$ as a quotient $GL_k(\KK^n)/GL(S)$, which gives $\gr_{k,n}$ a smooth manifold structure if $GL_k(\KK^n)$ may be taken as a lie group. Instead, we consider the $n\choose k$ $k\times k$ minors of $P$, which we write as $p_{T}$ for $T\in {[n]\choose k}$. This gives projective coordinates known as \emph{Pl\"ucker coordinates} for $\gr_{k,n}$ in $\PP_\KK^N:=\PP_{\KK}^{{n\choose k}-1}$. However, an arbitrary point in $\PP_\KK^N$ need not necessarily correspond to a $k$-dimensional subspace in $\KK^n$, indeed there are homogenous relations in $\KK[p_T\::\:T\in {[n]\choose k}]$ known as \emph{P\"ucker relations} which the minors of any $k\times n$ matrix satisfy, including for instance the three-term relations $$p_{T'\cup\{ij\}}p_{T'\cup\{kl\}}-p_{T'\cup\{ik\}}p_{T'\cup\{jl\}}+p_{T'\cup\{il\}}p_{T'\cup\{jk\}}$$
where $T'\in {[n]\choose k-2}$ and $i,j,k,l\in [n]\setminus T'$ are distinct. These relations characterize $\gr_{k,n}$ completely as a variety, which we formalize in the following definition.
\begin{definition}
	$\gr_{k,n}$ is the projective variety in $\PP_{\KK}^{N}$ defined by the ideal  $I_{k,n}\subset \KK[p_T]$ which is generated by the homogenous Pl\"ucker relations.
\end{definition}

Moreover, our description of the Grassmannian by Pl\"ucker coordinates enables us to define its totally positive part, which shall be our primary concern here. We let $\KK=\RR$. Then, the totally positive part of $\gr_{k,n} (\RR)$, here denoted $\gr_{k,n}^+(\RR)$ is the subset of $\gr_{k,n}(\RR)$ where (some presentation of) the Pl\"ucker coordinates $(p_T)$ are all positive real numbers. This set has been of great algebro-geometric and combinatorial interest in recent years.
\subsection{Polyhedral Geometry \& Gr\"obner Bases}
In order to properly discuss the Grassmannian, its totally positive part, and the tropicalization of each, we shall give a few definitions from polyhedral geometry which we shall refer to throughout the rest of this paper.
\begin{definition}
	\ddi{define cone}
\end{definition}
\begin{definition}
	\ddi{define fan}
\end{definition}
We shall also require some basic notions from Gr\"obner basis theory. This will allow us to discuss Gr\"obner complexes, a polyhedral complex corresponding to a given homogenous ideal which shall be critical to our discussion of the tropical Grassmannian.
\begin{definition}
	\ddi{initial segments}
\end{definition}
\ddi{define Gr\"obner bases \& Gr\"obner complex here}
\subsection{Tropical Varieties and Tropicalization}
\label{deftrop}
In general, tropical algebraic geometry is the study of the \emph{tropical semiring} $(\RR,\oplus,\odot)$, where $a\oplus b:=\min\{a,b\}$ and $a\odot b:=a+b$. We often may discuss tropical geometry without explicit reference to $\oplus$ and $\otimes$ via a process called \emph{tropicalization}, which gives a correspondence between a variety and its \say{tropicalized} counterpart. In this section, we largely follow \cite[\S3]{MaSt15}. 
\ddi{Define Puiseaux series at some point?}
We let $S=\KK[x_1,\hdots,x_n]$ where $\KK$ is an algebraically closed field with nontrivial valuation with splitting $\phi:\Gamma_{\mathrm{val}}\to \KK^*$ by $w\mapsto t^w$.\todo{Figure out if we actually need this} A familiar example which shall be central to the rest of this paper is the field of \emph{Puiseaux series in $\CC$,} $\Ccal:=\acl{\CC(t)}=\bigcup_{n\geq 1}\CC((t^{1/n}))$. We let $f(x)=\sum_{\vec{u}\in \NN^{n}}c_{\vec{u}}x^{\vec{u}}\in \KK[x]$ We may then define the \emph{tropicalization} of $f$ as a function $\mathrm{trop}(f):\RR^{n}\to \RR$ by \begin{equation} \mathrm{trop}(f)(\vec{w})=\min_{u\in \NN^{n+1}} \{\mathrm{val}(c_{\vec{u}})+\vec{w}\cdot\vec{u}\;:\;\vec{u}\in \NN^{n}\}
\end{equation}
Intuitively, what tropicalization \say{does} is \say{translate} the coefficients of $f$ to $\RR$ with the valuation map and evaluate $f$ at $\vec{u}$ with tropical operations substituted in for their classical counterparts. Now, the graph of $\trop(f)$ over $\RR^{n}$ is a piecewise linear \say{tent} with \say{(tent) poles}\footnote{Yes, this is confusing terminology; we use it only colloquially and only right now with the caution that we are \emph{not} referring to poles in the sense of analysis.} where $\trop(f)$ fails to be differentiable, that is to say where the minimum in its definition is achieved at least twice. We define $\Tcal(f)$, the \emph{tropical hypersurface associated to $f$} as precisely those points in $\RR^{n}$ at which $\trop(f)$ fails to be differentiable. When $n=2$, $\Tcal(f)$ is an embedding of a connected graph into $\RR^2$, pointing towards the central motif of tropical algebraic geometry: transforming data related to smooth varieties into combinatorial data.

Pushing this idea further, 
 \subsection{Clusters}
\ddi{Come back to this later\textemdash need to define at least type $D_4$.}
\section{The ``Speyer-Williams Fan" $F_{3,6}$}
\section{Computing the rays of $F_{3,6}$}
\nocite{*}
\printbibliography
\end{document}
