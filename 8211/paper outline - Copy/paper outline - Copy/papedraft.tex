\documentclass[12pt,letter]{article}
\usepackage{genpape}
\usepackage[style=alphabetic,doi=false,isbn=false,url=false]{biblatex}
\usepackage{dirtytalk}
\bibliography{trop} 
\newcommand{\gr}{\mathrm{Gr}}
\newcommand{\grkn}[2]{\mathrm{Gr}_{#1,#2}}
\newcommand{\trop}{\mathrm{Trop}}
\newcommand{\web}{\mathrm{Web}}
\newcommand{\prad}{\mathrm{Prod}}
\newcommand{\plu}{Pl\"ucker~}
\newcommand{\gro}{Gr\"obner~}
\newcommand{\swH}{Speyer--Williams}
\newcommand{\val}{\mathrm{val}}
\newcommand{\res}[1]{\overline{#1}}
\newcommand{\kk}{\mathbbm{k}}
\newcommand{\ins}{\mathrm{in}}
\newcommand{\proj}{\mathrm{proj}}
\newcommand{\spec}{\mathrm{spec}}
\title{Tropical Grassmannians and the Speyer--Williams Fan}
\date{9 October 2017}
\author{David DeMark}

\begin{document}
\maketitle
%\setcounter{section}{-2}
%\section{Progress Report}
%To this point, my primary activity towards the completion of this paper has been towards gaining my own bearings on the material. I read through \cite{WiSp05}, \cite{He09} and of course \cite{BCL16} in order to determine what material would be pre-requisite to understanding theorem 4.1 of \cite{BCL16}, and have since undertaken a thorough study of the relevant chapters of \cite{MaSt15}. This process is still underway as I was wholly unfamiliar with large parts of the theory required here (especially that of polyhedral geometry and Gr\"obner basis theory), and what is below represents an outline of the prerequisite material as well as some motivating exposition on Grassmannians and total positivity in general (this may ultimately find itself in the introduction by the 13th). My work on the computational side is thus far in its infancy; I've determined that I will likely be utilizing \verb|gfan| and \verb|polymake|, so my progress thus far has been confined to figuring out how to install those\textemdash in the coming days I'm hoping to gain sufficient comfort with these (as well as possibly \verb|homology| and \verb|macaulay2| per the recommendation of \cite{He09}) to begin my work on actually computing the combinatorial structure of $F_{3,6}$. 
\setcounter{section}{-1}
\section{Introduction} Cluster algebras were introduced in the first years of the current century in part to reflect the mutative structure which had been observed in studies of total positivity. 
Among the most well-known examples of cluster structure arising in applications of total positivity to familiar contexts is that of the Grassmannian $\gr_{2,n}$, which parametrizes two-dimensional planes through the origin in $n$-dimensional linear space. As we shall expand upon momentarily, \todo{Oh christ how will we ever reconcile this with \S1.1} elements of the (2,n)-Grassmannian are distinguished by \emph{Pl\"ucker coordinates,} that is, the projective coordinates in $\PP^{{n \choose 2}-1}$ given by the $n \choose 2$ $2\times 2$ minors of a given $2\times n$ matrix whose rows give a basis for the 2-dimensional plane in question [unfinished]
\section{Preliminaries \&~Basic Definitions}

\subsection{Grassmannians}
Let $\KK$ be a field fixed over the duration of this subsection. The \emph{$k,n$-Grassmannian over $\KK$} (denoted $\gr_{k,n}(\KK)$ or just $\gr_{k,n}$ when $\KK$ is clear from context) can be thought of as the $k$-dimensional subspaces through the origin in the $\KK$-vector space $\KK^n$. There are several ways of defining the Grassmannian one can refer to, including one \cite[\S1.2]{khat} in which $\gr_{k,n}$ is actually viewed the set of such $k$-dimensional subspaces, which is then topologized via a surjection from the Stiefel manifold of $k$-touples of orthonormal vectors in $\KK^n$. Here, we present the definition of the grassmannian most common in algebraic geometry,
%(roughly following \cite[\S6]{harris}),
with the na\"ive understanding of above as inspiration.

A $k$-dimensional subspace $S$ of $\KK^n$ can be described by a basis, which we represent here as a full-rank $k\times n$ matrix $P$. This description is not unique however, indeed $S$ is invariant under the action of $GL(S)$. This points to another definition of $\gr_{k,n}$ as a quotient $GL_k(\KK^n)/GL(S)$, which gives $\gr_{k,n}$ a smooth manifold structure if $GL_k(\KK^n)$ may be taken as a lie group. Instead, we consider the $n\choose k$ $k\times k$ minors of $P$, which we write as $p_{T}$ for $T\in {[n]\choose k}$. This gives projective coordinates known as \emph{Pl\"ucker coordinates} for $\gr_{k,n}$ in $\PP_\KK^N:=\PP_{\KK}^{{n\choose k}-1}$. However, an arbitrary point in $\PP_\KK^N$ need not necessarily correspond to a $k$-dimensional subspace in $\KK^n$, indeed there are homogenous relations in $\KK\left[p_T\::\:T\in {[n]\choose k}\right]$ known as \emph{P\"ucker relations} which the minors of any $k\times n$ matrix satisfy, including for instance the three-term relations $$p_{T'\cup\{ij\}}p_{T'\cup\{kl\}}-p_{T'\cup\{ik\}}p_{T'\cup\{jl\}}+p_{T'\cup\{il\}}p_{T'\cup\{jk\}}$$
where $T'\in {[n]\choose k-2}$ and $i,j,k,l\in [n]\setminus T'$ are distinct. These relations characterize $\gr_{k,n}$ completely as a variety, which we formalize in the following definition.
\begin{definition}
	$\gr_{k,n}$ is the $k(n-k)$-dimensional projective variety in $\PP_{\KK}^{N}$ defined by the ideal  $I_{k,n}\subset \KK[p_T]$ which is generated by the homogenous Pl\"ucker relations.
\end{definition}

Moreover, our description of the Grassmannian by Pl\"ucker coordinates enables us to define its totally positive part, which shall be our primary concern here. We let $\KK=\RR$. Then, the totally positive part of $\gr_{k,n} (\RR)$, here denoted $\gr_{k,n}^+(\RR)$ is the subset of $\gr_{k,n}(\RR)$ where (some presentation of) the Pl\"ucker coordinates $(p_T)$ are all positive real numbers. Analogously, in $\Rcal$, the totally positive (k,n)-Grassmannian $\gr_{k,n}^+(\Rcal)$ is the subset of $\gr_{k,n}(\Rcal)$ for which the coefficient of the lowest-degree term in each Pl\"ucker coordinate is positive. These subvarieties~\ddf{wait is that the right terminology?} have been of great algebro-geometric and combinatorial interest in recent years.
\section{Cluster Algebras and the Associahedron}
\ddi{Brief description. Example: Whichever polygon triangulation is type 2,6}
\subsection{Polyhedral Geometry \& Gr\"obner Bases}
In order to properly discuss the Grassmannian, its totally positive part, and the tropicalization of each, we shall give a few definitions from polyhedral geometry which we shall refer to throughout the rest of this paper.

\begin{definition}
A \emph{cone} $C$ is a subset of $\RR^n$ such that for any finite set $S\subset  C$, all subtraction-free linear combinations of elements in $S$ are elements of $C$. A \emph{polyhedral cone} is a finitely generated cone $C$; that is a subset of $\RR^d$ with the property that there exists some $\vec{s}_1,\hdots,\vec{s}_k\in C$ such that for all $\vec{x}\in C$, there exist $a_1,\hdots,a_k\in \RR^+$ such that $\vec{x}=\sum_{i=1}^k a_i\vec{s}_i$. 
\end{definition}

\begin{definition}
	A \emph{polyhedral complex} $\Sigma$ is a collection of polyhedra containing the empty polyhedron such that for any polyhedron $P\in \Sigma$, each face $F\subset \partial P$ is in $\Sigma$, and the intersection of any two polyhedra in $\Sigma$ is a common face of each. The \emph{support} or \emph{underlying point set} of $\Sigma$ is $\abs{\Sigma}=\bigcup_{P\in \Sigma}P$. We say a polyhedron $P\in \Sigma$ is maximal if it is not a face of another element of $\Sigma$. The \emph{dimension} of $\Sigma$ is the supremum of the dimensions of all elements of $\Sigma$, and we say $\Sigma$ is \say{pure $d$-dimensional} if all maximal polyhedra are dimension-$d$.
\end{definition}
\begin{definition}
	The \emph{face poset} $\Pcal(\Sigma)$ of a polyhedral complex $\Sigma$ is the graded poset in which the nodes are polyhedra of $\Sigma$ and $P$ covers $Q$ if $Q$ is a maximal proper face of $P$. $\Pcal(\Sigma)$ is graded by dimension; we refer to each rank level by $\Pcal_i(\Sigma)$ containing all $i$-dimensional faces of $\Sigma$. Two polyhedral complexes are \emph{combinatorially equivalent} if their face posets coincide. Considering the empty face to be dimension $-1$, the \emph{$f$-vector} of $\Sigma$ is $(\#\Pcal_{-1}(\Sigma),\#\Pcal_{0}(\Sigma),\hdots,\#\Pcal_{d}(\Sigma))$. 
\end{definition}
\begin{definition}
	A \emph{fan}  $F$ is a polyhedral complex in which each element is a cone. 
	%We say $F$ is \emph{complete} if $F$ is $n$-dimensional and $\abs{F}=\RR^n$
\end{definition}
We shall also require some basic notions from Gr\"obner basis theory. This will allow us to discuss Gr\"obner complexes, a polyhedral complex corresponding to a given homogenous ideal which shall be critical to our discussion of the tropical Grassmannian.

We let $S=\KK[x_1,\hdots,x_n]$ where $\KK$ is the field of \emph{Puiseaux series in $\CC$ or $\RR$,} $\Ccal:=\acl{\CC(t)}=\bigcup_{n\geq 1}\CC((t^{1/n}))$ and $\Rcal:=\bigcup_{n\geq 1}\RR((t^{1/n}))$ respectively. We equip $\Rcal$ and $\Ccal$ with $\val:\KK\to \QQ$ where $\val(\sum_{w\in \QQ}a_wt^w)=\min_{w\in \QQ}\{w\;:\;a_w\neq 0\}$. We note that $\val$ \emph{splits,} that is, $\val(t^w)=w$. We denote the residue field of $\KK$ by $\kk$ and denote the image of $a\in \KK$ in $\kk$ by $\res{a}$.
\begin{definition}
	We let $\vec{w}\in \RR^{n}$ be fixed and $f=\sum_{\vec{u}\in \NN^n}c_ux^\vec{u}\in S$. We let $W:=\trop(f)(\vec{w})$. The \emph{initial segment} of $f$ w/r/t $\vec{w}$ is $$\ins_{\vec{w}}(f)=\sum_{\val({c_\vec{u}})+\vec{w}\cdot\vec{u}=W}\res{c_\vec{u}t^{-\val({c_\vec{u}})}}x^\vec{u}\in \kk[x_1,\hdots,x_n]$$                                              Similarly, for an ideal $I=\idl{f_1,\hdots,f_n}\subset S$, we define the initial segment of $I$ w/r/t/ $\vec{w}$  as $\ins_\vec{w}(I)=\idl{\ins_\vec{w}(f_1),\hdots,\ins_\vec{w}(f_n)}\subset \kk[x_1,\hdots,x_n]$.                         
\end{definition}
We may now define a central concept to tropical algebraic geometry.
\begin{definition}[Definition/Proposition]
	We let $I\subset S$ be an ideal and $\vec{w}\in \RR^n$. We define a set $$C_I[\vec{w}]:=\{\vec{w}'\in \RR^n\;:\;\ins_{\vec{w}'}(I)=\in_w(I)\}.$$ We denote its closure under the Euclidean topology as $\acl{C_I[\vec{w}]}$. $\acl{C_I[\vec{w}]}$ is a polyhedron whose lineality space contains $\RR\vec{1}$. If $\in_\vec{w}(I)$ is not a monomial ideal, then there is some $\vec{w}'\in \RR^n$ such that $\acl{C_I[w]} $ is a proper face of $\acl{C_I[\vec{w'}]}$. We define the \gro complex as $\Sigma(I):=\{C_{I}[\vec{w}]\}_{\vec{w}\in \RR^n}$; it follows from the observation above that $\Sigma(I)$ is a polyhedral complex.
\end{definition}
\begin{example}
\ddi{\gro example for PL(2,6)}
\end{example}
\subsection{Tropical Varieties and Tropicalization}
\label{deftrop}
In general, tropical algebraic geometry is the study of the \emph{tropical semiring} $(\RR,\oplus,\odot)$, where $a\oplus b:=\min\{a,b\}$ and $a\odot b:=a+b$. We often may discuss tropical geometry without explicit reference to $\oplus$ and $\otimes$ via a process called \emph{tropicalization}, which gives a correspondence between a variety and its \say{tropicalized} counterpart. In this section, we largely follow \cite[\S3]{MaSt15}. 
\ddi{recall puiseaux....}
 We let $f(x)=\sum_{\vec{u}\in \NN^{n}}c_{\vec{u}}x^{\vec{u}}\in \KK[x]$ We may then define the \emph{tropicalization} of $f$ as a function $\mathrm{trop}(f):\RR^{n}\to \RR$ by \begin{equation} \mathrm{trop}(f)(\vec{w})=\min_{u\in \NN^{n+1}} \{\mathrm{val}(c_{\vec{u}})+\vec{w}\cdot\vec{u}\;:\;\vec{u}\in \NN^{n}\}
\end{equation}
Intuitively, what tropicalization \say{does} is \say{translate} the coefficients of $f$ to $\RR$ with the valuation map and evaluate $f$ at $\vec{u}$ with tropical operations substituted in for their classical counterparts. Now, the graph of $\trop(f)$ over $\RR^{n}$ is a piecewise linear \say{tent} with \say{(tent) poles}\footnote{Yes, this is confusing terminology; we use it only colloquially and only right now with the caution that we are \emph{not} referring to poles in the sense of analysis.} where $\trop(f)$ fails to be differentiable, that is to say where the minimum in its definition is achieved at least twice. We define $\Tcal(f)$, the \emph{tropical hypersurface associated to $f$} as precisely those points in $\RR^{n}$ at which $\trop(f)$ fails to be differentiable. When $n=2$, $\Tcal(f)$ is an embedding of a connected graph into $\RR^2$, pointing towards the central motif of tropical algebraic geometry: transforming data related to smooth varieties into combinatorial data.

Pushing this idea further, we may define the tropical variety of an ideal $I$ as $$\Vcal(I):=\bigcap_{f\in I}\Tcal(f).$$
This is the definition which shall serve as our intuitive understanding of tropical varieties. From the fundamental theorem of tropical algebraic geometry, we may also take the definition of $\Vcal(I)$ to be (i) the set of all vectors $\vec{w}\in \RR^n$ with $\ins_\vec{w}(I)\neq \idl{1}$, or (ii) letting $X=V(I)\subset (\KK^*)^n$, the closure of the image of the coordinate-wise application of the $\val$ map on $X$. We now a slightly modified theorem which shall shed light on the structure of the tropical Grassmannian. \begin{theorem}[Structure Theorem for Tropical Varieties]\label{struc}
If $V(I)$ is an irreducible $d$-dimensional variety in $(\KK^*)^n$, then $\Vcal(I)$ is the support of a pure dimension-$d$ rational polyhedral complex. In particular, if $I$ has a generating set $F$ with \emph{constant coefficients} in the sense that $\val{a_\vec{u}}=0$ for any coefficient $a_\vec{u}$ of $f\in F$, $\Vcal(I)$ is a pure dimension-$d$ fan in $\RR^{n}$.
\end{theorem}
In particular, $\Vcal(I)$ is a subcomplex of the Gr\"obner complex $\Sigma(I_\proj)$.
We use the third definition of $\Vcal(I)$ to define the totally positive part of a tropical variety as $\Vcal^+(I)=\acl{\val(V(I)\cap (\KK^+)}$, that is the closure of the image of the restriction of the valuation map to the totally positive part of the classical variety $V(I)$. Speyer and Williams \cite{WiSp05} prove that a point $\vec{w}\in \Vcal(I)$ lies in $\Vcal^+(I)$ if and only if $\ins_\vec{w}(I)$ contains no nonzero elements of $\RR^+[x_1,\hdots,x_n]$.

\section{Total Positivity, $\trop \left(\gr^+_{k,n}\right)$ and the ``Speyer--Williams Fan" $F_{k,n}$}
\subsection{The Tropical Grassmannian}\label{tropr}
In this section, we work largely from Speyer and Sturmfels' paper \cite{SpSt04} of the same title as well as \cite[\S4.3]{MaSt15}. Theorem \ref{struc} implies that $\trop \left(\gr_{k,n}(\Rcal)\right)$ is a pure $k(n-k)$-dimensional polyhedral fan in $\Rcal^{N}$. Its cones have a common intersection which may be parametrized by the map $\trop \phi: \RR^n\to \trop(\gr_{k,n}(\Rcal))$ which takes $(i_1,\hdots i_n)$ to the ${n\choose k}$-vector which for $K={j_1,j_2,\hdots,j_k}\in {[n]\choose k}$ has $K$-coordinate $i_{j_1}+i_{j_2}+\hdots +i_{j_k}$. $\trop\phi$ is in fact an injection, and thus its image is a pure $n$-dimensional cone. We may also consider $\phi:(\KK^+)^n\to \gr_{k,n}(\KK)$, the de-tropicalization of $\trop \phi$, which maps $(i_1,\hdots i_n)$ to the ${n\choose k}$-vector with $K$-coordinate $i_{j_1}i_{j_2}\hdots i_{j_k}$. Consider the $(\KK^*)^n$-action on $\gr_{k,n}(\KK)$, in which an element $(\lambda_1,\hdots,\lambda_n)\in(\KK^*)^n$ acts on a $k\times n$ matrix $A$ representing a point $P\in \gr_{k,n}(\KK)$ by multiplying each column $i$ by $\lambda_i$. This takes the \plu coordinate $p_K$ to $\left(\prod_{i\in K}\lambda_i\right)p_k$. Then, $\gr_{k,n}(\KK)/\phi((\KK^*)^n)$ is as a set the orbits of the $(\KK^*)^n$-action we have just described. In the next section, we shall discuss a bijective parametrization of this quotient.  
\subsection{Parametrizing the classical Grassmannian and its quotients}
Postnikov has given an explicitly combinatorial parametrization of the Grassmannian \cite{postnikov2006total}, which is generalized by Speyer--Williams \cite{WiSp05} in their study of tropical total positivity, which we follow along with from here. In this  Postnikov's method associates the directed graph $\web_{k,n}$ with $\gr^+_{k,n}$, where $\web_{k,n}$ is the directed graph obtained from a $k\times (n-k)$ grid with rows indexed $1,\hdots,k$ and columns indexed $(k+1),\hdots n$ by adding left- and down-facing arrows to each vertex (as well as sources labeled by $[k]$ along the right side of the grid and sinks labeled by $[n]\setminus [k]$ along the bottom with all labellings increasing clockwise). Each edge is given weighting $x_e$, and a path (compatible with the orientation of the graph) $e_1e_2\hdots e_m$ is associated to the monomial $\prad_p(x)=\prod_{i=1}^mx_{e_{i}}$. For a set of paths $S$, we let $\prad_S(x)=\prod_{p\in S}\prad_p(x)$. We then let $A_{n,k}$ be the $k\times n$ matrix with entries $a_{ij}(x)=(-1)^{i+1}\sum\prad_p(x)$ summing over all paths from the source at vertex $i$ to the sink at vertex $j$. We let $K\in {[n]\choose k}$ and let $\mathrm{Path(K)}$ be the set of tuples of pairwise vertex-disjoint paths with sinks in $K$ and sources in its complement. An application of the familiar  Gessel-Viennot trick for determinental calculations (for an exposition thereof see e.g. \cite[\S2.7]{stan}) yields the following result:\begin{proposition}
	$P_K(x):=p_K(A_{k,n}(x))=\sum_{S\in \mathrm{Path}(k)} \prad_S(x)$. 
\end{proposition}  
Then, substituting elements of $\RR^+$ for the $2k(n-k)$ weight variables $x_e$ gives the Pl\"ucker coordinates of element of $\gr_{3,6}^+(\RR)$.
 This gives a map $\Phi_0:(\RR^+)^{2k(n-k)}\to \gr_{3,6}^+(\RR)$, which, as it turns out, is surjective but due to obvious dimension concerns, is not injective. Speyer and Williams refine this map by replacing the weighting scheme as follows:
  rather than weighting by \emph{edges}, we weight by \emph{regions}, which are defined as follows: inner regions are the maximal connected components of the complement of an embedding of $\web_{k,n}$ into $\RR^2$, while outer regions are those which would satisfy the definition of inner region if we added edges between each source/sink $i$ and $i+1$, but are not inner regions.
  These regions are weighted by monomials $z_r$ in $\{x_e\}$ and $\{x_e^{-1}\}$, with each counterclockwise-oriented edge $e$ bordering region $r$ contributing $x_e$ and each counterclockwise edge $f$ contributing $x_f^{-1}$. Then, one may check that indeed the path monomials $\prad_p(x)$ may be viewed as monomials in the $z_r$, giving a map $\Phi_1:(\RR^+)^{k(n-k)}\to \gr_{k,n}^+(\RR)$.
  Speyer and Williams explicitly construct an inverse map $\Psi$ to $\Phi_1$, showing bijectivity. 
  A key feature of the map $\Psi$ is that each coordinate map (indexing $\RR^{k(n-k)})$ by $[k]\times [j]$) $\Psi_{i,j}$ is the ratio of two monomials in the \plu coordinates $p_{K}$. Thus, the composition $\Psi\Phi_1:(\RR^+)^{k(n-k)}\to(\RR^+)^{k(n-k)}$ expresses each region variable $x_r$ as a ratio of monomials in $P_K(x)$. In particular, the multiset formed from the indices of the numerator of $x_R$ and that formed from the denominator coincide if and only if $R$ is an inner region. Thus, the composition of $\Psi$ with the map $\phi$ of section \ref{tropr} fixes the $x_r$ corresponding to the\ddf{orbits?} inner regions, while acting transitively on the $x_r$ corresponding to the outer regions. This observation gives a bijective map $\Phi_2:(\RR^+)^{(n-k-1)(k-1)}\to\gr^+_{k,n}(\RR) /\phi(\RR^+)^n$ by lifting $c\in (\RR^+)^{(n-k-1)(k-1)}$ to a point $\tilde{c}\in (\RR^+)^{k(n-k)}$ agreeing on the coordinates corresponding to the inner regions, applying $\Phi_1$, then going down to the corresponding point in the quotient. Speyer and Williams also prove that the results above still apply when $\RR^+$ is replaced with $\Rcal^+$, establishing a bijection between $(\Rcal^+)^{(k-1)(n-k-1)}$ and $\gr_{k,n}^+(\Rcal)/\phi(\Rcal^+)^n$. Then, the tropicalization of $\Phi_2$ gives a surjective map $\trop\phi_2:\RR^{(k-1)(n-k-1)}\to \trop\gr_{k,n}^+/\trop\phi(\RR^+)^n$.
  
  \subsection{The Speyer--Williams Fan} We have now built up the machinery to define the central object of Speyer and Williams' study:
  \begin{definition}
  	The Speyer--Williams fan $F_{k,n}$ is the complete fan in $\RR^{(k-1)(n-k-1)}$ which has as its maximal cones the domains of linearity for $\trop\phi_2$. 
  \end{definition}
Speyer and Williams then show the following results:
\begin{theorem}[\cite{WiSp05}, \S 5-7]~
\begin{enumerate}
	\item The fan $F_{2,n}$ is combinatorially equivalent to the \emph{Stanley--Pitman} fan $F_{n-3}$, which has structure determined by the set of plane binary trees with $n-1$ leaves
	\item The fan $F_{3,6}$ has $f$-vector $(16, 66, 98, 48)$, which very nearly coincides with the $f$-vector $(16,66,100,50)$ of the fan normal to the type-$D_4$ generalized associahedron associated to the cluster algebra of type $D_4$. The discrepancy in the latter two coordinates can be explained by noting that two of the cones in $F_{3,6}$ are \say{cones over a bipyramid} which, when subdivided, yield a refined polyhedral complex with $f$-vector coinciding with that of the fan normal to the type-$D_4$ generalized associahedron.
	\item The fan $F_{3,7}$ has $f$-vector $(42, 392, 1463, 2583, 2163, 693)$. There exists a refinement of $F_{3,7}$ which establishes a polyhedral complex with $f$-vector coinciding with that of the fan normal to the type-$E_6$ generalized associahedron, $(42, 399, 1547, 2856, 2499, 833)$.
\end{enumerate}
\end{theorem}
 
In later work, Brodsky, Ceballos, and Labb\'e \cite{BCL16} make the connection between $F_{3,6}$ and type-$D_4$ cluster algebras more precise by giving an explicit bijection between combinatorial types of tropical planes in tropical projective space $\TT\PP^5$, which are realized by $\trop\gr_{3,6}$, and clusters in the cluster algebra of type $D_4$ 
  
\begin{example}
	\ddi{compute fan of $F_{2,6}$}
\end{example}


%We have now built up the machinery we need to define the Speyer--Williams fan $F_{k,n}$. 
%\section{Computing the $f$-vector of $F_{3,6}$}
\nocite{*}
\printbibliography
\end{document}
