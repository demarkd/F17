\documentclass[12pt,letter]{article}
\usepackage{genpape}
\usepackage[style=alphabetic,doi=false,isbn=false,url=false]{biblatex}
\usepackage{dirtytalk}
\bibliography{trop} 
\newcommand{\gr}{\mathrm{Gr}}
\newcommand{\grkn}[2]{\mathrm{Gr}_{#1,#2}}
\newcommand{\trop}{\mathrm{Trop}}
\newcommand{\web}{\mathrm{Web}}
\newcommand{\prad}{\mathrm{Prod}}
\newcommand{\plu}{Pl\"ucker~}
\newcommand{\gro}{Gr\"obner~}
\newcommand{\swH}{Speyer--Williams}
\newcommand{\val}{\mathrm{val}}
\newcommand{\res}[1]{\overline{#1}}
\newcommand{\kk}{\mathbbm{k}}
\newcommand{\ins}{\mathrm{in}}
\newcommand{\proj}{\mathrm{proj}}
\newcommand{\spec}{\mathrm{spec}}
\title{Peer Review for Greg Michel's ``Koszul Algebras"}
\date{19 November 2017}
\author{David DeMark}

\begin{document}
\maketitle\section*{Summary of Comments}
In \emph{Koszul Algebras,} Greg builds up the necessary machinery from multilinear and homological algebra to define and briefly discuss the work's titular objects. He first defines quadratic algebras as a certain class of quotients of the tensor algebra on a $k$-vector space $V$ and gives the polynomial algebra as a basic example. He also introduces the quantum plane. These two structures shall serve as his running examples throughout the piece. He then dedicates a substantial amount of space to a discussion of projective resolutions and cohomology, with a perspective rooted in the ambient algebraic structure of each object in the piece, the tensor algebra. He gives several computational examples of both concepts, each taking place within the context of his two running examples. Also in this section, he defines  $\mathrm{Ext}$, an essential component to the notion of cohomology.  In the final section of the piece, he defines Koszul algebras, presents a theorem giving several equivalent definitions, and demonstrates that the quantum plane is indeed Koszul.

The piece excels in its presentation of projective resolutions and Greg's choice of examples. The presentation of projective resolutions is highly readable, and the examples in that subsection serve well to illustrate the concept. More broadly, the two running examples provide a perfect canvas with which to illustrate each of the essential notions introduced, with the polynomial algebra serving as the \say{easy} example and the quantum plane the \say{interesting} example. All of the computations in the piece are exceptionally easy to follow, and there is no portion of the piece which begs the question of what it is doing there. However, the latter portion of \S2 feels slightly blown through\textemdash I would have appreciated approximately as thorough a treatment of cohomology as that of projective resolutions\textemdash and the third section feels slightly thin. In particular, while Koszul algebras are clearly and thoroughly defined, it is left to the reader to determine why exactly they are an interesting or useful definition. An extension to another context hinting at their importance or utility could go a long way towards completing the final draft of the paper. Finally, this review is incomplete without acknowledgment of the momentous honor and blessing it was to review a paper written by international \emph{Magic: The Gathering} celebrity Gregory Michel.

\section*{Suggested Revisions}
\begin{itemize}
	\item \emph{Example 1.2} It may make sense to mention that in general, this construction is known as the \say{symmetric algebra of $V$.}
	\item \emph{Example 1.3} \say{\textellipsis generated by two generators\textellipsis} is an awkward phrasing\textemdash perhaps 'generators' could be replaced by 'indeterminates?'
	\item \emph{\S2, introductory remarks} While you use the notion of a \say{projective resolution over [an algebra]} repeatedly in the coming sections, you never define it\textemdash while defining projective resolutions in general seems like an opportune time. Further, it may make some sense to put that definition in a definition environment.
	\item \emph{\S2, introductory remarks} In the final displayed equation of the first page, 'pd' should be in \verb|mathrm| style.
	\item \emph{\S2, remarks proceeding example 2.1} It may serve to clarify what exactly $A$ is\textemdash perhaps with the phrase \say{\textellipsis which says that a module over a Noetherian connected $k$-algebra $A$ is projective\textellipsis} 
	\item \emph{Example 2.1} Is exact complex the right terminology? Would long exact sequence be more appropriate? I'm genuinely not sure.
	\item \emph{Example 2.2}  The kernel of $\epsilon$ is \emph{not} $\mathrm{span}_k\{x,y\}$, it's the ideal generated by them\textemdash which is distinct from their span here. 
	\item \emph{Example 2.2} In the paragraph following the second displayed exact sequence, you multiply an ordered pair representing a column vector by a row-vector, which is a bit confusing\textemdash the only fix I can think of is biting the bullet and writing out $(-qy,x)$ as a column vector. 
	\item \emph{Definition 2.3} The dual of a projective resolution is never quite satisfactorily defined\textemdash is it just application of $\hom_{k[x]}(-,k)$, or could another contravariant functor take its place?
	\item \emph{Example 2.4} Consider expanding on \say{There is a well defined multiplication on $\mathrm{Ext}_A(k,k)$.}
	\item \emph{Definition 3.1} Is there a reason such a resolution is called \emph{minimal?}
	\item \emph{Example 3.5} $R$ and $R^\bot$ both need $\mathrm{span}$ before their brackets.  
\end{itemize}
%\nocite{*}
%\printbibliography
\end{document}
